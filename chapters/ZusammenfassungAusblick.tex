\chapter{Zusammenfassung und Ausblick}
In dieser Projektarbeit wurde die Umsetzung einer mobilen Applikation zur zentralen Verwaltung WG-typischer Aufgaben
erl�utert. Dabei wurde mit Hilfe der Entwicklungsumgebung \textit{Android Studio} und der Programmiersprache
\textit{Java} eine Applikation implementiert, die auf allen Smartphones mit Android 4.0 oder h�heren Versionen des
Betriebssystems lauff�hig sind. Die Daten der Nutzer werden in einer externen, �ber eine Internetverbindung
erreichbaren, Datenbank gespeichert. Bei jedem Start der App werden die Daten des Benutzers aus der Datenbank geladen und bei jedem Beenden der App wieder in diese Datenbank geschrieben.\\

Die im Vorfeld get�tigten umfangreichen Vorbereitungen, von Use-Cases-Digrammen,
�ber Mockups bis hin zum Lastenheft, haben uns im Nachhinein in der Implementierungsphase sehr gut weitergeholfen. 
Insbesondere durch das Lastenheft waren wir zu jederzeit in der Lage, den �berblick �ber die Funktionen und deren 
Anforderungen zu behalten. Damit hatten wir zur finalen Version der Applikation eine gute Referenz zum Vergleichen.
\\

Die Datenbankverbindung hat im Vergleich zu anderen Anforderungen zurecht die h�chste Aufmerksamkeit bekommen. 
Bereits in der ersten Vorbesprechung war uns klar, dass es bei der Arbeit mit externen Quellen zu verschiedensten
Fehlern kommen kann. Die Entwicklung eines Prototypen vor Beginn der Implementierung der App, hat uns vermutlich im
sp�teren Verlauf erheblich viel Zeit und �rger erspart.
\\

Leider gestaltete sich der Einstieg in die Programmierung f�r die Plattform \textit{Android} schwieriger als erwartet.
Am Anfang gab es Kompatitiblit�tssprobleme bei der Konfiguration von \textit{Eclipse} mit dem Plugin \textit{Android
Developer Tools}, weshalb wir kurzerhand auf die Standalone Variante\textit{ Android Studio }umgestiegen sind. Doch auch
bei dieser Software geh�ren w�hrend Konfiguration und Implementierung verwirrende Mechaniken, sich widersprechende
Anleitungen und undeutliche Fehlermeldungen fast zur Tagesordnung. Die Tatsache, dass lediglich ein Drittel des Teams erste grundlegende Erfahrungen in der Programmierung f�r \textit{Android} sammeln konnte, hat die Situation w�hrend der Implementierungsphase weiter erschwert. F�r weitere Projekte w�re zu Beginn ein Teammeeting zu empfehlen, in dem alle Mitglieder auf den gleichen Stand der Technik und des Wissens gebrachten werden.
\\

Eine Ver�ffentlichung in den \textit{Google Play Store}, dem damit verbundenen erh�hten Arbeitsaufwand der Qualit�tssicherung, dem Support der Nutzer und der Erwartung der Nutzer auf immer weitere Steigerung des Umfangs, ist zum jetzigen Zeitpunkt nicht geplant.

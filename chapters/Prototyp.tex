\chapter{Prototyp}\label{chp:Prototyp}
Um vermeidbare Verz�gerungen in der sp�teren Entwicklung m�glichst auszuschlie�en, wurde der Einsatz von Prototypen entschieden. 
Dazu werden die ben�tigten Komponenten genauer betrachtet und auf ihre Umsetzbarkeit hin untersucht. 
Bei einer Android-App, die auf Inhalte aus einer Datenbank zugreift, 
ist das die Schnittstelle zur Datenbank, beziehungsweise die Netzwerkverbindung.\\
Da eine direkte Verbindung zum Datenbankserver aus dem Android Betriebssystem nicht m�glich ist,
wurde die Authentifizierung des Benutzers als weitere Problemstelle markiert.
Den die Verwendung von Datenbankbenutzern f�llt dadurch weg und muss andersweitig gel�st werden.
Um bei diesen kritischen Stellen nicht in bredouille zu geraten, wurde daf�r ein Prototyp entwicklt.

\section{Implementierung}\label{sect:ImplPrototyp}
Als eine einfache und schnelle Art der Umsetzung haben wir uns f�r die PHP-Variante entschieden.
Dabei werden die Daten von einem PHP-Skript aus der Datenbank gelesen und in eine JSON-Datenformat gebracht.
Dieses Objekt wird in einem HTTP-Paket an die App �bertragen.
In der App sorgt dann ein JSON-Parser f�r das Auslesen der Daten, welche anschlie�end direkt verwertet werden oder
zun�chst in der lokalen SQLite-Datenbank vorgehalten werden. \\
Alternativ zu dieser L�sung w�re ein RESTful Web Service gewesen. 
Dieser L�sungsansatz w�re jedoch mit einem gr��eren programmiertechnischen Aufwand, sowie einer umfangreicheren
Serverkonfiguration verbunden gewesen. \\
Da beide Schwerpunkte in einem Prototyp getestet wurden, wird auf eine weitere Trennung verzichtet.
\pagebreak

\subsection{Datenbankstruktur}\label{subsect:DBStruktPrototyp}
Begonnen wurde die Umsetzung mit der Definition der Datenbankstruktur sowie deren Umsetzung.
Dazu wurden zwei einfache Tabellen angelegt. Die Tabelle \verb!tp_test!\ref{sql:tptest} wird f�r die Schreib- und Lesevorg�nge
verwendet. Um alle n�tigen Datentypen testen zu k�nnen wurden verschiedene Spalten verwendet. 
Dadurch konnte auch der sp�tere Einsatz besser simuliert werden.

\begin{figure} %[hbtp]
  \centering
  \begin{minipage}{1.0\textwidth} \small
    \begin{lstlisting}[language=sql]
      CREATE TABLE IF NOT EXISTS `test_tp` (
          `uid` VARCHAR(23) NOT NULL,
          `msg` TEXT,
          `nmbr` INT(11) DEFAULT NULL,
          `created_at` TIMESTAMP NOT NULL DEFAULT CURRENT_TIMESTAMP
      ) ENGINE=MyISAM DEFAULT CHARSET=utf8;
    \end{lstlisting}
  \end{minipage}
  \caption{Aufbau der Tabelle tp\_test} 
  \label{sql:tptest}
\end{figure}

Um die Benutzerverwaltung f�r den Prototypen zu simulieren wurden au�erdem noch eine Tabelle \verb!users!
\ref{sql:usersTest} angelegt.
Darin wurden Informationen zu den Benutzern hinterlegt. Zum Beispiel eine eindeutige ID, sowie Vor- und Nachname.
Zum Authentifizieren wurde die E-Mailadresse und ein beliebiges Passwort verwendet.
Um das Passwort nicht im Klartext zu speichern, wurde es zusammen mit einem Salt als Hash-Wert abgelegt.

\begin{figure} %[hbtp]
  \centering
    \begin{minipage}{1.0\textwidth} \small
      \begin{lstlisting}[language=sql]
        CREATE TABLE IF NOT EXISTS `users` (
          `uid` int(11) NOT NULL AUTO_INCREMENT,
          `unique_id` varchar(23) NOT NULL,
          `firstname` varchar(50) NOT NULL,
          `lastname` varchar(50) NOT NULL,
          `username` varchar(20) NOT NULL,
          `wgId` int(11) NOT NULL,
          `email` varchar(100) NOT NULL,
          `encrypted_password` varchar(80) NOT NULL,
          `salt` varchar(10) NOT NULL,
          `created_at` datetime DEFAULT NULL,
          PRIMARY KEY (`uid`)
        ) ENGINE=MyISAM  DEFAULT CHARSET=utf8 AUTO_INCREMENT=5 ;
        \end{lstlisting}
    \end{minipage}
  \caption{Aufbau der Tabelle users}
  \label{sql:usersTest}
\end{figure}

  %Nachdem die Struktur der Datenbank stand, wurde die \kill

\subsection{PHP-Skripte}\label{subsect:PHPrototyp}
Der Aufbau der PHP-Schnittstelle ist simpel umgesetzt, da nicht viele Funktionen f�r den Prototyp ben�tigt werden.
Trotzdem wurde auf eine �bersichtliche Datei- und Ordnerstruktur, als auch auf einen modularen Aufbau geachtet.
Wie in Abbildung \ref{dirtree:phpPrototyp} \nameref{dirtree:phpPrototyp} zu sehen, wurden der Aufbau zun�chst in zwei Kategorien aufgeteilt.
Funktionen, die direkt auf der Datenbank ausgef�hrt werden, sowie das Verbinden und Bereitstellen des Datenbankobjekts
�bernehmen, sind im Ordner \verb!php/include! untergebracht.
Alle weiteren Funktionen die aus der App heraus erreichbar sein sollen, befinden sich im Hauptordner \verb!php!.

\begin{figure} %[hbtp]
  \centering
  \begin{minipage}{1.0\textwidth} \small
    \dirtree{%
      .1 php.
        .2 include.
          .3 config.php \ldots{} \begin{minipage}[t]{5cm}
                                Verbindungsparameter f�r die Datenbank{.}
                               \end{minipage}.
          .3 DB\_Connect.php \ldots{}
                               \begin{minipage}[t]{5cm}
                                 Funktionen zum Aufbau und dem Bereitstellen der Datenbankverbindung{.}
                               \end{minipage}.
          .3 DB\_Functions.php \ldots{} \begin{minipage}[t]{5cm}
                                  Methoden f�r Datenbankoperationen; zum Beispiel getTable{.}
                                \end{minipage}.
        .2 db.php \ldots{} \begin{minipage}[t]{5cm}
                                  Verarbeitung des HTTP-Requests, auswerten der Anfrage, sammeln der Daten und
                                  abschlie�end Daten verpacken und an Client zur�cksenden{.}
                                \end{minipage}.
        .2 login.php \ldots{} \begin{minipage}[t]{5cm}
                                  Authentifizieren des Benutzers{.}
                                \end{minipage}.
    }
  \end{minipage}
  \caption{Struktur der PHP-Skripte im Dateisystem}
  \label{dirtree:phpPrototyp}
\end{figure}

Auf die einzelnen Skripte, sowie die Funktionen der Methoden, wird im Kapitel \ref{section:App} noch
detailiert eingegangen, da diese nahezu identisch �bernommen wurden.

\subsection{Prototyp-App}\label{subsect:PrototypApp}


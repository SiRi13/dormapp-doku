\section{Risiken}\index{Risiken}
Risiken k�nnen den Ablauf in einer Softwarearchitektur erheblich beeinflussen. Jedes Risiko, welche vor der
Implementierung nicht ber�cksichtigt wurde, kann beim Benutzer des Systems ein unkontrollierbares Verhalten verursachen.
Um dies bestm�glich zu vermeiden haben wir uns bereits fr�hzeitig Gedanken um M�gliche Risiken gemacht. In einem
Brainstorming kamen so eine Menge an Risikopunkten zusammen, die zu einem sp�teren Zeitpunkt Probleme machen k�nnten. In
unserem Lastenheft, werden zu jeder Anforderung alle Risiken aufgelistet.
\\
\begin{itemize}
	\item \textbf{Blackboard}
		\begin{itemize}
			\item \textit{Anforderung F1 - Notiz hinzuf�gen}
				\begin{itemize}
					\item Benutzer bricht Vorgang ab
					\item Speichern nicht m�glich
					\item Synchronisieren nicht m�glich
					\\
				\end{itemize}
			\item \textit{Anforderung F2 - Notiz bearbeiten}
				\begin{itemize}
					\item Benutzer bricht Vorgang ab
					\item Speichern nicht m�glich
					\item Synchronisieren nicht m�glich
					\\
				\end{itemize}
			\item \textit{Anforderung F3 - Notiz l�schen}
				\begin{itemize}
					\item Ungewolltes L�schen
					\\
				\end{itemize}
			\item \textit{Anforderung NF4 - Design}			
				\begin{itemize}
					\item Un�bersichtliches Layout
					\item Missverst�ndliche Abl�ufe
					\\
				\end{itemize}
		\end{itemize}
\end{itemize}

\begin{itemize}
	\item \textbf{Putzplan}
		\begin{itemize}
			\item \textit{Anforderung F5 - Aufgabe erledigen}
				\begin{itemize}
					\item Fehlerhafte Synchronisierung
					\item Falsche Aufgabe ausgew�hlt
					\\
				\end{itemize}
			\item \textit{Anforderung NF6 - Design}
				\begin{itemize}
					\item Namen des Verantwortlichen zu Lang und/oder in nicht darstellbarem Zeichensatz
					\item Rhythmus funktioniert nicht wie Benutzer es erwartet
					\\
				\end{itemize}
		\end{itemize}
\end{itemize}

\begin{itemize}
	\item \textbf{Einkaufsliste}
		\begin{itemize}
			\item \textit{Anforderung F7 - Artikel hinzuf�gen}
				\begin{itemize}
					\item Fehler beim Speichern in die Datenbank
					\item Der Artikel steht bereits in der Einkaufsliste, allerdings mit anderer Buchstabenformatierung (Gro�- und Kleinschreibung, Bindestrich, etc.)
					\\
				\end{itemize}
			\item \textit{Anforderung F8 - Artikel l�schen}
				\begin{itemize}
					\item Fehler beim Speichern in die Datenbank
					\item Der Artikel existiert bereits nicht mehr in der Datenbank, wird aber in der Applikation auf dem Endger�t noch angezeigt.
					\\
				\end{itemize}
			\item \textit{Anforderung F9 - Artikel gekauft}
				\begin{itemize}
					\item Fehler beim speichern in die Datenbank
					\item Der Artikel existiert bereits nicht mehr in der Datenbank, wird aber in der Applikation auf dem Endger�t noch angezeigt.
					\\
				\end{itemize}
			\item \textit{Anforderung NF10 - Design}
				\begin{itemize}
					\item Sortierung der Artikel
					\item Un�bersichtliche Aufz�hlung
					\item Schlecht leserliche Schlichtart und Schriftgr��e
					\\
				\end{itemize}
		\end{itemize}
\end{itemize}

\begin{itemize}
	\item \textbf{Kalender}
		\begin{itemize}
			\item \textit{Anforderung F11 - Kalendereintrag eintragen}
				\begin{itemize}
					\item Fehler beim Speichern in die Datenbank
					\item Titel ist zu lang
					\item Datum liegt in der Vergangenheit
					\item Datum liegt weit in der Zukunft (>50 Jahre)
					\\
				\end{itemize}
			\item \textit{Anforderung F12 - Kalendereintrag bearbeiten}
				\begin{itemize}
					\item Fehler beim Speichern in die Datenbank
					\item Titel ist zu lang
					\item Datum liegt in der Vergangenheit
					\item Datum liegt weit in der Zukunft (>50 Jahre)
					\\
				\end{itemize}
			\item \textit{Anforderung F13 - Kalendereintrag l�schen}
				\begin{itemize}
					\item Fehler beim speichern in die Datenbank
					\item Kalendereintrag existiert nicht mehr in Datenbank, wird aber noch auf dem Endger�t angezeigt
					\\
				\end{itemize}
			\item \textit{Anforderung NF14 - Design}
				\begin{itemize}
					\item Benutzer ben�tigt zu lange um sich mit dem Design zurecht zu finden
					\item Zu wenig Platz um alle Kalendereintragungen anzuzeigen
					\item Titel zu lang f�r Gesamtansicht
					\\
				\end{itemize}
		\end{itemize}
\end{itemize}

\begin{itemize}
	\item \textbf{Web Login}
		\begin{itemize}
			\item \textit{Anforderung F15 - Web Login}
				\begin{itemize}
					\item E-Mailadresse existiert nicht
					\item E-Mailadresse und/oder Passwort falsch
					\item Passwort vergessen
					\item Verbindung zum Server kann nicht hergestellt werden
					\\
				\end{itemize}
			\item \textit{Anforderung F16 - Eingabekontrolle}
				\begin{itemize}
					\item\textit{ Eingabe falscher Daten}
					\\
				\end{itemize}
			\item \textit{Anforderung F17 - Passwort vergessen}
				\begin{itemize}
					\item Missbrauch von Daten
					\\
				\end{itemize}
			\item \textit{Anforderung NF18 - Design}
				\begin{itemize}
					\item Design Gesetze nicht eingehalten (Gesetz der N�he, Gleichheit, Geschlossenheit, etc.)
					\item Gr��e der Textfelder nicht optimal
					\\
				\end{itemize}
		\end{itemize}
\end{itemize}


\begin{itemize}
	\item \textbf{Web Oberfl�che}
		\begin{itemize}
			\item \textit{Anforderung F19 - Benutzerverwaltung}
				\begin{itemize}
					\item Datenbankverbindung schl�gt fehl
					\\
				\end{itemize}
			\item \textit{Anforderung F20 - Terminkalenderverwaltung}
				\begin{itemize}
					\item Datenbankverbindung
					\\
				\end{itemize}
			\item \textit{Anforderung F21 - Putzplanverwaltung}
				\begin{itemize}
					\item Datenbankverbindung
					\\
				\end{itemize}
			\item \textit{Anforderung F22 - Systemeinstellungen}
				\begin{itemize}
					\item Datenbankverbindung
					\\
				\end{itemize}
			\item \textit{Anforderung NF23 - Design}
				\begin{itemize}
					\item Un�bersichtlich
					\\
				\end{itemize}
		\end{itemize}
\end{itemize}
Die mit Abstand anf�lligste Funktion unserer App wird die Verbindung zur Datenbank auf einem externen Server sein. Dabei kann zu jedem Zeitpunkt ein Fehler auftreten, der vom System m�glichst gut behandelt werden soll. Darum hat die Verbindung zur Datenbank mehr Aufmerksamkeit bekommen und wir haben uns daf�r entschieden einen lauff�higen Prototypen zu entwickeln. Wir haben uns bereits jetzt so stark mit der Mechanik besch�ftigt, um anstelle eines einfachen Wegwerf-Prototypen, einen im sp�teren Verlauf der Implementierung weiter zu verwendenden Prototypen entwerfen k�nnen.
Folgende Risiken k�nnen bei unserer Arbeit mit einer extern erreichbaren Datenbank auftreten:
\begin{itemize}
	\item Server nicht erreichbar
	\item Verbindungsabbruch durch Zusammenbruch der Internetverbindung
	\item Benutzer killt die App w�hrend der aktiven Nutzung der Verbindung
\end{itemize}
Eine genaue Erl�uterung der Implementierung des gesamten Prototypen finden Sie im folgenden Kapitel \ref{chp:Prototyp} \nameref{chp:Prototyp}.

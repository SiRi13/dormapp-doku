\section{Einkaufsliste}\label{section:Einkaufsliste}
Die Einkaufsliste ist eine vom Benutzer sortierte Liste von Einkaufsgegenst�nden. Es k�nnen von jedem WG Mitglied Gegenst�nde hinzugef�gt und entfernt werden. Abschlie�end soll dadurch die Abrechnung vereinfacht werden, da den
gekauften Artikel jeweils der K�ufer, als auch der bezahlte Preis zugeordnet werden kann.

\subsection{Implementierung}\label{subSec:EkLImpl}
Wie die Aufgaben wurde auch f�r die Einkaufliste eine Klasse \texttt{ShoppingListFragment} von der Klasse
\verb!Fragment! abgeleitet. Da hierbei bis einschlie�lich zum Einsatz des \texttt{ShoppingListAdapter} analog zum
\texttt{ChorePlanFragment} vorgegangen wurde, ist eine erneute Ausf�hrung nicht notwendig. Im Vergleich dazu wurde
jedoch ein \verb!AutoCompleteTextView! verwendet. Da meist die gleichen Artikel hinzugef�gt werden m�ssen, ist der
Einsatz der Autovervollst�ndigung hier eine gro�e Erleichterung. Dazu werden zun�chst die gew�nschten Vorschl�ge aus der
Datenbank in ein \verb!String-Array! geladen (vgl. Zeile \ref{line:getGroceries}). Danach wird ein \verb!ArrayAdapter!
erzeugt, dem das \verb!Array! mit den Vorschl�gen, sowie ein Layout �bergeben werden. Dieser Adapter wird dem
vorbereiteten \verb!AutoCompleteTextView! auf dem Einkaufslisten-\verb!Fragment! gesetzt. Als weitere Erleichterung
wurde ein \verb!Listener! zur automatischen Eingabebest�tigung implementiert (Zeile \ref{line:onEditorListener}),
welcher beim Hinzuf�gen von Waren zum Einsatz kommt, die noch nicht als Vorschlag verf�gbar hinterlegt sind.
Dazu kommt der \texttt{TextView.OnEditorActionListener()} zum Einsatz und f�ngt das Dr�cken der Tasten auf der Tastatur
ab und �berpr�ft ob es sich dabei um die \verb!Senden!-Taste handelt (siehe Zeile \ref{line:actionIdVergleich}).
In diesem Fall wird die Weitergabe des Events unterbrochen und die \textsl{handleItemAddAction(\ldots{})} aufgerufen
(vgl. Zeile \ref{line:handleItemAdd}, der der eingegebene Text �bergeben wird und diesen als neue Auswahl zur Verf�gung
stellt, sowie ein Dialog �ffnet mit dem die ben�tigte Anzahl f�r die Einkaufsliste �bergeben werden kann.\\

\noindent
\begin{minipage}{\linewidth}
\begin{lstlisting}[caption=ShoppingListFragment.java, captionpos=below, label=java:ShoppingListFrag,
language=java, escapechar=|]
[|\ldots{}|]
final String[] grocieries = dbHandler.getGrocieries(); |\label{line:getGroceries}|
ArrayAdapter<String> adapter = new ArrayAdapter<String>(getActivity(),
    android.R.layout.simple_list_item_1, grocieries);
final AutoCompleteTextView editTextNew =
    (AutoCompleteTextView) rootView.findViewById(R.id.autoCompleteShoppingListNewItem);
editTextNew.setAdapter(adapter);

editTextNew.setOnEditorActionListener(new TextView.OnEditorActionListener() { |\label{line:onEditorListener}|
  @Override
  public boolean onEditorAction(TextView v, int actionId, KeyEvent event) {
    boolean handled = false;
    if (actionId == EditorInfo.IME_ACTION_SEND) { |\label{line:actionIdVergleich}|
      handleItemAddAction(v.getText().toString()); |\label{line:handleItemAdd}|
      handled = true;
    }
    return handled;
} });
[|\ldots{}|]
\end{lstlisting}  
\end{minipage}      

Dieser Teil der App konnte ohne weitere Probleme gel�st werden, weswegen auf die hier �blichen Paragraphen \textsc{Probleme}
und \textsc{L�sungen} verzichtet wird.


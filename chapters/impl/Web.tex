\section{Weboberf�che}\label{section:WebUI}
Der Administrator einer WG muss in der Lage sein, die Eigenschaften der WG zu konfigurieren und die Funktionen zu verwalten. Wir haben uns dazu entschieden, eine extern jederzeit erreichbare Weboberfl�che daf�r bereit zu stellen. Wir h�tten uns genauso gut f�r die Implementierung in die Android App entscheiden k�nnen, haben uns jedoch bewusst dagegen entschieden. Wir m�chten mit den unterschiedlichen Programmiersprachen und den daraus resultierenden Herangehensweisen eine m�glichst gro�es Vielfalt der Informatik widerspiegeln und die Arbeiten auf die Mitglieder verteilen, die bisher noch keine Erfahrung mit der Programmierung f�r die Android Plattform sammeln konnten.\\
Im folgenden Kapitel wird die Weboberfl�che mit Hilfe von Codesnippets erl�utert.

\subsection{Umsetzung}\label{subSec:WebUmsetz}
Als Programmiersprache haben wir uns f�r die sehr beliebte und weit verbreitete Skriptsprache PHP entschieden. PHP bietet durch den Einfluss von Java, C++ und Perl einen leichten Einstieg f�r diejenigen, die bereits erste Erfahrungen mit einer der Programmiersprachen sammeln konnten. Ausserdem bietet PHP die einfache Umsetzung von dynamischen Webseiten und eine sehr gute Unterst�tzung von Datenbankverbindungen. Mit PHP ist automatisch sichergestellt, dass die Weboberfl�che von jedem g�ngigen Browser aus in deren Desktop- sowie Mobilversion genutzt werden kann.\\
\begin{description}
\item Die Weboberfl�che gliedert sich in neun Seiten:
\begin{itemize}
	\item admin.php
	\item benutzer.php
	\item blackboard.php
	\item einkaufsliste.php
	\item login.php
	\item logout.php
	\item putzplan.php
	\item regist.php
	\item system.php
\end{itemize}
\end{description}
Jede dieser PHP Dateien stellt eine Seite der Weboberfl�che dar. Jede Datei beinhaltet gew�hnlichen HTML Code f�r die Anzeige im Browser und PHP Code f�r den dynamischen Teil der Datenabfrage von der Datenbank.\\
Mehrzeilige Abfragen in PHP sind in eigene Dateien ausgelagert. So ist z.B. die Abfrage zum l�schen eines WG Mitglieds in der Datei  \textit{benutzer\textunderscore edit\textunderscore delete.php} zu finden. So verfahren wir mit allen mehrzeiligen Abfragen um die �bersichtlichkeit zu erh�hen.\\
\begin{description}
\item Die Weboberfl�che mit ihren neun Seiten inklusive aller ausgelagerten PHP Abfragen gegliedert nach Seite:
\begin{itemize}
	\item admin.php
	\item benutzer.php
	\begin{itemize}
		\item benutzer\textunderscore edit\textunderscore delete.php
		\item benutzer\textunderscore script.php
		\item benutzer\textunderscore update.php
	\end{itemize}
	\item blackboard.php
	\begin{itemize}
		\item blackboard\textunderscore add.php
		\item blackboard\textunderscore delete\textunderscore edit.php
		\item blackboard\textunderscore update.php
	\end{itemize}
	\item einkaufsliste.php
	\begin{itemize}
		\item einkaufsliste\textunderscore add.php
		\item einkaufsliste\textunderscore delete\textunderscore edit.php
		\item waren\textunderscore add\textunderscore delete.php
		\item waren\textunderscore add.php
	\end{itemize}
	\item login.php
	\begin{itemize}
		\item login\textunderscore script.php
	\end{itemize}
	\item logout.php
	\item putzplan.php
	\begin{itemize}
		\item putzplan\textunderscore add.php
		\item putzplan\textunderscore delete\textunderscore edit.php
		\item putzplan\textunderscore update.php
	\end{itemize}
	\item regist.php
	\begin{itemize}
		\item regist\textunderscore script.php
	\end{itemize}
	\item system.php
	\begin{itemize}
		\item system\textunderscore delete\textunderscore edit.php
		\item system\textunderscore update.php
	\end{itemize}
\end{itemize}
\end{description}
Text Text Text zur Erkl�rung zur Gliederung (add, delete-edit, update)



\begin{description}
\item Dazu kommen noch:
\begin{itemize}
	\item style.css
	\item wg\textunderscore app\textunderscore aktuell.sql
\end{itemize}
\end{description}
\section{Weboberf�che}\label{section:WebUI}
Der Administrator einer WG muss in der Lage sein, die Eigenschaften der WG zu konfigurieren und die Funktionen zu verwalten. Wir haben uns dazu entschieden, eine extern jederzeit erreichbare Weboberfl�che daf�r bereit zu stellen. Wir h�tten uns genauso gut f�r die Implementierung in die Android App entscheiden k�nnen, haben uns jedoch bewusst dagegen entschieden. Wir m�chten mit den unterschiedlichen Programmiersprachen und den daraus resultierenden Herangehensweisen eine m�glichst gro�es Vielfalt der Informatik widerspiegeln und die Arbeiten auf die Mitglieder verteilen, die bisher noch keine Erfahrung mit der Programmierung f�r die Android Plattform sammeln konnten.\\
Im folgenden Kapitel wird die Weboberfl�che mit Hilfe von Codesnippets erl�utert.

\subsection{Umsetzung}\label{subSec:WebUmsetz}
Als Programmiersprache haben wir uns f�r die sehr beliebte und weit verbreitete Skriptsprache PHP entschieden. PHP bietet durch den Einfluss von Java, C++ und Perl einen leichten Einstieg f�r diejenigen, die bereits erste Erfahrungen mit einer der Programmiersprachen sammeln konnten. Ausserdem bietet PHP die einfache Umsetzung von dynamischen Webseiten und eine sehr gute Unterst�tzung von Datenbankverbindungen. Mit PHP ist automatisch sichergestellt, dass die Weboberfl�che von jedem g�ngigen Browser aus in deren Desktop- sowie Mobilversion genutzt werden kann.\\
\begin{description}
\item Die Weboberfl�che gliedert sich in neun Seiten:
\dirtree{%
.0 /.
.2 admin.php.
.2 benutzer.php.
.2 blackboard.php.
.2 einkaufsliste.php.
.2 login.php.
.2 logout.php.
.2 putzplan.php.
.2 regist.php.
.2 system.php.
}
\end{description}
Jede dieser PHP Dateien stellt eine Seite der Weboberfl�che dar. Jede Datei beinhaltet gew�hnlichen HTML Code f�r die Anzeige im Browser und PHP Code f�r den dynamischen Teil der Datenabfrage von der Datenbank.\\
Mehrzeilige Abfragen in PHP sind in eigene Dateien ausgelagert. So ist z.B. die Abfrage zum l�schen eines WG Mitglieds in der Datei  \textit{benutzer\textunderscore edit\textunderscore delete.php} zu finden. So verfahren wir mit allen mehrzeiligen Abfragen um die �bersichtlichkeit zu erh�hen.\\
\begin{figure}
\begin{description}
\item Daraus ergibt sich folgende Struktur f�r die Weboberfl�che mit ihren Seiten inklusive aller ausgelagerten PHP Abfragen:
\dirtree{%
.1 /.
.2 admin.php.
.2 benutzer.php.
.3 benutzer\textunderscore aktivierung.php.
.3 benutzer\textunderscore edit\textunderscore delete.php.
.3 benutzer\textunderscore script.php.
.3 benutzer\textunderscore update.php.
.2 blackboard.php.
.3 blackboard\textunderscore add.php.
.3 blackboard\textunderscore delete\textunderscore edit.php.
.3 blackboard\textunderscore update.php.
.2 einkaufsliste.
.3 einkaufsliste\textunderscore add.php.
.3 einkaufsliste\textunderscore delete\textunderscore edit.php.
.3 waren\textunderscore add\textunderscore delete.php.
.3 waren\textunderscore add.php.
.2 login.
.3 login\textunderscore script.php.
.2 logout.php.
.2 putzplan.php.
.3 putzplan\textunderscore add.php.
.3 putzplan\textunderscore delete\textunderscore edit.php.
.3 putzplan\textunderscore unteraufgaben\textunderscore add.php.
.3 putzplan\textunderscore unteraufgaben\textunderscore update.php.
.3 putzplan\textunderscore update.php.
.2 regist.php.
.3 regist\textunderscore script.php.
.2 system.php.
.3 system\textunderscore delete\textunderscore edit.php.
.3 system\textunderscore update.php.
.2 style.css.
}
\end{description}
\end{figure}
Alle Eingaben die der Benutzer in der Weboberfl�che machen kann, werden durch Textfeder oder Checkboxen erfasst. Wir verwenden f�r alle zu �bertragenden Daten die Methode POST, die die Daten f�r Benutzer der Seiten unsichtbar im Rumpf des HTTP-Requests sendet. Die Methode GET, die die Daten f�r alle sichtbar an die URI anh�ngt, k�nnte theoretisch ebenfalls genutzt werden. Jedoch w�rde das in unserem Fall ein erh�htes Sicherheitsrisiko darstellen. Darum haben wir bis auf wenige ID's auf die Methode GET verzichtet.\\ \\
\begin{figure}
Auf jeder HTML Seite und in jedem PHP Skript wird zu Beginn �berpr�ft, ob der Nutzer im System angemeldet ist. Falls die Pr�fung fehlschl�gt, wird ihm mit dem Hinweis \textit{Bitte loggen Sie sich erst ein!} die Anzeige der Seite verwehrt.
\begin{lstlisting}[language=PHP, caption=Login des Benutzers �berpr�fen, captionpos=below, label=WEBLogin]
if(!isset($_SESSION["email"]))
{
	echo("<a href=\"login.php\" />Bitte loggen sie sich erst ein!</a>");
	exit;
}
\end{lstlisting}
\end{figure}

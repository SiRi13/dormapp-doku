\section{Weboberf�che}\label{section:WebUI}
Der Administrator einer WG muss in der Lage sein, die Eigenschaften der WG zu konfigurieren und die Funktionen zu verwalten. Wir haben uns dazu entschieden, eine extern jederzeit erreichbare Weboberfl�che daf�r bereit zu stellen. Wir h�tten uns genauso gut f�r die Implementierung in die Android App entscheiden k�nnen, haben uns jedoch bewusst dagegen entschieden. Wir m�chten mit den unterschiedlichen Programmiersprachen und den daraus resultierenden Herangehensweisen eine m�glichst gro�es Vielfalt der Informatik widerspiegeln und die Arbeiten auf die Mitglieder verteilen, die bisher noch keine Erfahrung mit der Programmierung f�r die Android Plattform sammeln konnten.\\
Im folgenden Kapitel wird die Weboberfl�che mit Hilfe von Codesnippets erl�utert.

\subsection{Umsetzung}\label{subSec:WebUmsetz}
Als Programmiersprache haben wir uns f�r die sehr beliebte und weit verbreitete Skriptsprache PHP entschieden. PHP bietet durch den Einfluss von Java, C++ und Perl einen leichten Einstieg f�r diejenigen, die bereits erste Erfahrungen mit einer der Programmiersprachen sammeln konnten. Ausserdem bietet PHP die einfache Umsetzung von dynamischen Webseiten und eine sehr gute Unterst�tzung von Datenbankverbindungen. Mit PHP ist automatisch sichergestellt, dass die Weboberfl�che von jedem g�ngigen Browser aus in deren Desktop- sowie Mobilversion genutzt werden kann.\\
\begin{description}
\item Die Weboberfl�che gliedert sich in 11 Seiten:
\begin{itemize}
	\item admin.php
	\item benutzer.php
	\item blackboard.php
	\item einkaufsliste.php
	\item login.php
	\item logout.php
	\item admin.php
	\item putzplan.php
	\item regist.php
	\item admin.php
	\item system.php
\end{itemize}
\end{description}
Jede dieser PHP Dateien beinhaltet gew�hnlichen HTML Code um dem Browser die Daten pr�sentieren und PHP Code f�r den dynamischen Teil der Datenabfrage von der Datenbank.
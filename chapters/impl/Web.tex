\section{Weboberfl�che}\label{section:WebUI}
Der Administrator einer WG muss in der Lage sein, die Eigenschaften der WG zu konfigurieren und die Funktionen zu verwalten. Wir haben uns dazu entschieden, eine extern jederzeit erreichbare Weboberfl�che daf�r bereit zu stellen. Wir h�tten uns genauso gut f�r die Implementierung in die Android App entscheiden k�nnen, haben uns jedoch bewusst dagegen entschieden. Wir m�chten mit den unterschiedlichen Programmiersprachen und den daraus resultierenden Herangehensweisen eine m�glichst gro�e Vielfalt der Informatik widerspiegeln und die Aufgaben fair auf die Teammitglieder verteilen, die bisher noch keine Erfahrung mit der Programmierung f�r die Android Plattform sammeln konnten.\\
Im folgenden Kapitel wird die Struktur der Weboberfl�che mit Hilfe von einzelnen Ausz�gen des Quellcodes erl�utert.

\subsection{Umsetzung}\label{subSec:WebUmsetz}
Als Programmiersprache haben wir uns f�r die sehr beliebte und weit verbreitete Skriptsprache PHP entschieden. PHP bietet durch den Einfluss von Java, C++ und Perl einen leichten Einstieg f�r diejenigen, die bereits erste Erfahrungen mit einer der Programmiersprachen sammeln konnten. Ausserdem bietet PHP die einfache Umsetzung von dynamischen Webseiten und eine sehr gute Unterst�tzung von Datenbankverbindungen. Mit PHP ist automatisch sichergestellt, dass die Weboberfl�che von jedem g�ngigen Browser aus in deren Desktop- sowie Mobilversion angezeigt werden kann.\\
\begin{description}
\item Die Weboberfl�che gliedert sich in neun Seiten:
\dirtree{%
.0 /.
.2 admin.php.
.2 benutzer.php.
.2 blackboard.php.
.2 einkaufsliste.php.
.2 login.php.
.2 logout.php.
.2 putzplan.php.
.2 regist.php.
.2 system.php.
}
\end{description}
Jede dieser PHP Dateien stellt eine Seite der Weboberfl�che dar. Jede Datei beinhaltet gew�hnlichen HTML Code f�r die Anzeige im Browser und PHP Code f�r den dynamischen Teil der Datenabfrage von der Datenbank.\\
Mehrzeilige Abfragen in PHP sind in externe Dateien ausgelagert. So ist z.B. die Abfrage zum l�schen eines WG Mitglieds in der Datei  \textit{benutzer\textunderscore edit\textunderscore delete.php} zu finden. So verfahren wir mit allen weiteren Abfragen. Dies soll die �bersichtlichkeit erh�hen.\\
\begin{figure}
\begin{description}
\item Daraus ergibt sich folgende Struktur f�r die Weboberfl�che mit ihren Seiten inklusive aller ausgelagerten PHP Abfragen:
\dirtree{%
.1 /.
.2 admin.php.
.2 benutzer.php.
.3 benutzer\textunderscore aktivierung.php.
.3 benutzer\textunderscore edit\textunderscore delete.php.
.3 benutzer\textunderscore script.php.
.3 benutzer\textunderscore update.php.
.2 blackboard.php.
.3 blackboard\textunderscore add.php.
.3 blackboard\textunderscore delete\textunderscore edit.php.
.3 blackboard\textunderscore update.php.
.2 einkaufsliste.
.3 einkaufsliste\textunderscore add.php.
.3 einkaufsliste\textunderscore delete\textunderscore edit.php.
.3 waren\textunderscore add\textunderscore delete.php.
.3 waren\textunderscore add.php.
.2 login.
.3 login\textunderscore script.php.
.2 logout.php.
.2 putzplan.php.
.3 putzplan\textunderscore add.php.
.3 putzplan\textunderscore delete\textunderscore edit.php.
.3 putzplan\textunderscore unteraufgaben\textunderscore add.php.
.3 putzplan\textunderscore unteraufgaben\textunderscore update.php.
.3 putzplan\textunderscore update.php.
.2 regist.php.
.3 regist\textunderscore script.php.
.2 system.php.
.3 system\textunderscore delete\textunderscore edit.php.
.3 system\textunderscore update.php.
.2 style.css.
}
\end{description}
\end{figure}
Alle Eingaben die der Benutzer in der Weboberfl�che machen kann, werden durch Textfelder oder Checkboxen erfasst. Wir verwenden f�r alle zu �bertragenden Daten die Methode POST. Damit werden die Daten f�r die Benutzer unsichtbar im Rumpf des HTTP-Requests gesendet. Die Methode GET, die die Daten f�r alle sichtbar an die URI anh�ngen w�rde, k�nnte theoretisch ebenfalls genutzt werden. Jedoch w�rde das in unserem Fall ein erh�htes Sicherheitsrisiko darstellen. Darum verzichten wie, f�r bis auf wenige ID's, auf die Methode GET.\\ \\

Auf jeder HTML Seite und in jedem PHP Skript wird zu Beginn �berpr�ft, ob der Nutzer im System angemeldet ist. Falls die Pr�fung fehlschl�gt, wird ihm mit dem Hinweis \textit{Bitte loggen Sie sich erst ein!} die Anzeige der Seite verwehrt und das PHP Script wird mit \textit{exit;} gestoppt.

\begin{figure}
\begin{lstlisting}[language=PHP, caption=Login des Benutzers �berpr�fen, captionpos=below, label=WEBLogin]
if(!isset($_SESSION["email"]))
{
	echo("<a href=\"login.php\" />Bitte loggen sie sich erst ein!</a>");
	exit;
}
\end{lstlisting}
\end{figure}

Ist der Benutzer eingeloggt, wird das PHP Skript nicht gestoppt, sondern weiter verarbeitet. Mit einem \textit{require\textunderscore once 'db\textunderscore inc.php} wird die Datei db\textunderscore inc.php eingebunden und ausgef�hrt. Das passiert nur, wenn die Datei nicht schon im vorhergehenden Teil des Codes  eingebunden wurde.

\begin{figure}
\begin{lstlisting}[language=PHP, caption=Verbindung aufbauen\, falls noch nicht geschehen, captionpos=below,
label=OpenConn]
require_once 'db_inc.php';	
\end{lstlisting}
\end{figure}

Im Anschluss k�nnen die �ber die Methoden POST und GET �bergebenden Variablen abgefragt und zwischengespeichert werden. In dem Beispiel wird mit einer if-Abfrage gepr�ft, ob eine Variable "einkaufsliste\textunderscore id" an das PHP Skript geliefert wurde. Falls dies der Fall ist, wird der Inhalt der Variablen lokal zwischengespeichert.

\begin{figure}
\begin{lstlisting}[language=PHP, caption=Variable \textit{einkaufsliste\textunderscore id} lokal zwischenspeichern,
captionpos=below, label=Variable]
if(isset($_POST['einkaufsliste_id']))
{
	$einkaufsliste_id = $_POST['einkaufsliste_id'];
}
\end{lstlisting}
\end{figure}
 

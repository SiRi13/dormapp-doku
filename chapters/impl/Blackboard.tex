\section{Blackboard}\label{section:Blackboard}
Eine einfache M�glichkeit alle WG-Mitglieder zu erreichen bietet ein Blackboard auf dem jeder Nachrichten hinterlassen
kann. Obwohl der Einsatz von Zugriffsbeschr�nkungen auf die Nachrichten leicht umsetzbar w�re,
wurde bewusst darauf verzichtet, um die Eigentschaften eines physikalischen Blackboards gerecht zu werden.

\subsection{Implementierung}\label{subSec:BbImpl}
Die Klassenhierarchie der Fragmente ist aus den vorhergehenden Beispielen schon bekannt und wurde auch in beim
\texttt{BlackboardFragment} beibehalten. Dabei wird in der �berschriebenen Methode \textsl{onCreateView(\ldots{})} durch
ein SQL-Statement die Nachrichten aus der \verb!SQLite!-Datenbank gelesen, welche in einem \verb!Cursor! vorgehalten
werden. Da es sich bei dem Zeilenlayout der \verb!ListView! um kein Standard-Layout handelt, muss zun�chst der
\verb!Cursor! schrittweise durchgearbeitet werden und die Werte in eine \verb!HashMap! �bertragen werden. Auf eine
\verb!HashMap! wurde zur�ckgegriffen, um beim L�schen des Eintrags einfach �ber die \textit{BlackboardId} an die
Nachricht zu gelangen und aus der Liste zu l�schen, das spart einen direkten Datenbankzugriff f�r das L�schen und
weitere Zugriffe beim Aktualisieren der Liste, sowie dem etwaigen Wiederherstellen der Nachricht.
�ber eine \verb!EditText!-Feld kann die neue, mehrzeilige Nachricht eingegeben werden und durch den
\textit{+}-Button dem schwarzen Brett hinzugef�gt werden. Das L�schen der einzelnen Nachrichten kann durch ein Klick auf
das L�schen-Symbol ausgel�st werden und ist durch die \verb!UndoBar!-Funktion revidierbar.
Zum Bearbeiten wurde ein \verb!onLongClickListener! an die \verb!View! der Zeile gebunden. Wird dieser ausgel�st, so
generiert er einen Dialog mit dem aktuellen Inhalt der Nachricht und zeigt diesen an. Gespeichert wird dann die
Nachricht mit der ID des Bearbeiters. Eine Erweiterung um die \verb!UndoBar!-Funktion sollte hier noch erg�nzt werden.

\subsection{Probleme}\label{subSec:BbProbl}
Bei der Umsetzung des Blackboards kam es beim Anzeigen der Nachrichten zu einem komplexen Problem, was zun�chst nicht
nachvollziehbar war. In der \verb!ListView! war zwar die Anzahl der Eintr�ge richtig, aber der Anfang der Liste wurde am 
Ende der Liste, also bei Zeilen die ausserhalb des anf�nglich darstellbaren Bereichs lagen, wiederholt.

\subsection{L�sungen}\label{subSec:BbSol}
Um die Datenquelle als Fehler auszuschlie�en wurde zun�chst ein \verb!DISTINCT! in das \verb!SQL!-Statement eingef�gt.
Das bewirkt, dass doppelte Eintr�ge ausgefiltert werden. Das war aber nicht die Ursache des Problems, da die erwarteten
Daten im \verb!Cursor! waren und auch an den \texttt{BlackboardAdapter} weitergegeben wurden. Somit lies sich das
Problem auf die Anzeige, beziehungsweise das die Vorbereitung der Daten zu Anzeige, eingrenzen. Demnach muss sich der
Fehler im erw�hnten \texttt{BlackboardAdapter} befinden. Nachdem weitere Gedanken �ber die Funktion des Adapters gemacht
wurden, kam die Erkenntnis, dass die Zeile \ref{line:WrongLineBbAdap} im alten Quelltext \ref{java:BbAdapOld} nicht
funktionieren kann, sobald es mehr Eintr�ge gibt, als auf anhieb anzeigbar sind.

\begin{lstlisting}[float, language=java, caption=Alte BlackboardAdapter.java, captionpos=below, label=java:BbAdapOld,
escapechar=|]
[|\ldots{}|]
@Override
public View getView(int position, View convertView, ViewGroup parent) { |\label{line:getView}|
  if (convertView &=&  null) { |\label{line:WrongLineBbAdap}|
    final BlackboardMessage bbMsg =
        (BlackboardMessage) blackboardMessages.values().toArray()[position];
  [|\ldots{}|] }
}
[|\ldots{}|]
\end{lstlisting}

Mit der \verb!if!-Abfrage, ob die �bergebene \verb!View! noch \verb!null! ist, wird verhindert, dass wenn die
\verb!ListView! gescrollt wird, die neue Zeile �berschrieben werden kann. In der \verb!ListView! befinden sich n�mlich
immer gleich viele \verb!Views! als Zeilen, vorrausgesetzt dass mehr Eintr�ge dargestellt werden sollen als auf den
sichtbaren Bereich passen. Tritt der Fall ein dass neue Zeilen angezeigt werden m�ssen, sprich es wird gescrollt, werden
die angezeigten \verb!View!s der \textsl{getView(int position, View convertView, ViewGroup parent)} (vgl. Zeile
\ref{line:getView} im \nameref{java:BbAdapOld}~\ref{java:BbAdapOld}) als \textit{converView} �bergeben. Somit kann diese
Variable nicht \verb!null! sein und die Bedingung der \verb!if!-Abfrage ist falsch. Demzufolge k�nnen die alten 
Daten nicht in den vorhandenen \verb!View!s durch die neuen ersetzt werden und die selbe Nachricht wird noch mal
angezeigt. Die Wiederholung des Anfangs wird dadurch erzeugt, dass die Zeilen-\verb!View!s wiederverwendet werden, die
aus dem angezeigten Bereich geschoben werden, also die zuvor erste \verb!View!.\\*
Durch das Entfernen der \verb!if!-Abfrage wurde die erw�nschte Funktion erreicht und das Scrollen der Liste war
m�glich.

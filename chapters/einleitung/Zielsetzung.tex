\section{Zielsetzung}\index{Zielsetzung}
Einer der WG-Mitbewohner erkl�rt sich bereit die Aufgaben des WG-Adminstrators zu �bernehmen. Dieser WG-Administrator registriert sich und legt dabei eine neue WG an. Zu seinen Aufgaben geh�rt unter anderem die Verwaltung der Mitbewohner sowie das Pflegen des Putzplans. Neue Mitbewohner m�ssen vom WG-Administrator per E-Mail in die WG eingeladen werden. F�r den Putzplan definiert er Aufgaben die in einem einstellbaren Rhythmus wiederholt werden und w�hlt zu jeder Aufgabe einen Mitbewohner aus. Nun beginnt der Rhythmus zu laufen und die Aufgaben wechseln nach Erledigung automatisch zu der n�chsten Person aus der WG. Auf dem Schwarzen Brett k�nnen Eintr�ge angezeigt, erstellt und gel�scht werden. In der Einkaufsliste k�nnen Artikel hinzugef�gt und entfernt werden. Wurde ein Artikel gekauft, kann derjenige den Artikel als gekauft markieren.
Alle �nderungen eines Mitbewohners ist f�r alle anderen Mitglieder der WG nach einer kurzen Synchronisation sichtbar.
Alle Informationen einer WG werden serverseitig in einer Datenbank und clientseitig auf dem Smartphone des Benutzers gespeichert. Bei jedem Start der App wird ein Datenabgleich der auf Client- und Serverseite gespeicherten Informationen durchgef�hrt und alle voneinander abweichenden Daten auf einer �bersichtsseite dem Benutzer als ,,Neu`` aufgelistet.
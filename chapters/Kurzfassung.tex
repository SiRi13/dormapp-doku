\kurzfassung

%% deutsch
\paragraph*{}

Mit der steigenden Anzahl von Studierenden an den Bildungseinrichtungen Deutschlands erfreuen sich die Wohngemeinschaften immer gr��erer Beliebtheit.
Mit den vielen neuen Studenten vergr��ern sich gleichzeitig die Wohngemeinschaften und die Probleme die mit solch zusammengew�rfelten Personengruppen einhergehen. 
Viele Studierende kennen das Problem einer chaotischen Zettelwirtschaft in K�che, Bad und Flur.\\
In dieser Projektarbeit wird die Problematik der Verwaltung von Haushaltsaufgaben in einer Wohngemeinschaft mit meist jungen Erwachsenen behandelt.
Ziel ist das oftmals auftretende Chaos (durch Zettelwirtschaft und schlechter Kommunikation) m�glichst gering zu halten 
und die anstehenden Aufgaben und Termine jederzeit klar definiert jedem Mitbewohner zug�nglich zu machen.
Dieses Ziel soll durch eine mobile Applikation f�r Smartphones auf Basis des von Google entwickelten Betriebssystems Android erreicht werden.


%% englisch
\paragraph*{}
\begin{comment}
This project thesis displays the often problematic administration of the various household tasks within a flatsharing student community.
Objective is to reduce the mostly chaotic atmosphere (be it due to jumble of bits of paper or bad communication) by defining due tasks and deadlines and publishing them to every flatmate in an easy manner.
This should be reached by using a mobile application based in an operation system from Google, called ?Android?. Carrier medium should be a common smartphone.
The application out every roommate into position to manage and create different notes and calender appointments. Furthermore it is possible to manage a shopping list where articles can easily be added, deleted or marked.  In addition to this shows
\end{comment}

Due to increasing numbers of students on German universities, living communities getting more and more important and
popular. This results in larger communities and therefore more problems. Chaotic pin boards and mess all over the place
is well known by every student.
\\
This project deals with the issue of managing household chores in those living communities mainly inhabited by young
adults. Diminishing and containing nuisance by untidy pin boards and poor communication, as well as providing upcoming
chores and tasks for everybody on demand are the main goals. For meeting those goals we decided to develop and implement
an mobile application for Google Android based smartphones.

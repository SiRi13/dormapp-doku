\chapter{Implementierung von \textsl{\textbf{DorMApp}}}\label{chp:Impl}
Die App wurde zum Gro�teil in der aktuell von Google empfohlenen Umgebung Android Studio 
umgesetzt. Zu Beginn fand die Entwicklung noch in Eclipse mit dem entsprechenden Plug-In
statt. Jedoch erschwerten die Bugs und die Beh�bigkeit der Eclipse-IDE den z�gigen 
Fortschritt. Aus diesem Grund wurde nach dem Legen des Grundsteins das Projekt auf die
neue Entwicklungsumgebung migriert, wo es auch fertiggestellt wurde. \\*

\section{Entwicklungsgrundlagen}\label{section:ImplGrundlagen}
Durch die Prototyp-Enwicklung konnte zu Beginn der Implementierung schon auf einige Bausteine zur�ckgegriffen werden.
Zun�chst wurde die Kommunikation mit der Datenbank nicht ver�ndert. Somit ruft die App weiterhin ein PHP-Skript auf,
welches dann die Tabelleninhalte ausliest, aufbereitet und im \emph{JSON}-Format zur�cksendet. 
Die von der \textsl{\textbf{DorMApp}}-App ben�tigten Skripte, sowie der Zugriff auf die Datenbank werden in den Kapiteln
\nameref{chp:Datenbank}~\ref{chp:Datenbank} und \nameref{chp:PhpDbKomm}~\ref{chp:PhpDbKomm} genauer erl�utert. Sollte
es das Verst�ndnis eines Sachverhaltes fordern, so wird auf das entsprechende Kapitel vorgegriffen und mit in die
Erkl�rung aufgenommen.

\paragraph*{Externe Bibliotheken}\label{para:ExtLibs}
Um Entwicklungszeit zu sparen, wurden vier externe Bibliotheken eingesetzt. Um eine bessere Sicherheit zu erreichen und
die Logindaten der Benutzer unverschl�sselt im \verb!SharedPreferences!-Bereich der App abzulegen, wurde die Bibliothek
\emph{secure-preferences} \cite{scottyab:14} eingebunden. Diese Bibliothek stellt eine Schnittstelle f�r die
eigentlichen \verb!SharedPreferences! bereit, die zwar die gleichen Methoden bereitstellt, aber alle Daten
verschl�sselt im App-Speicherbereich ablegt. Das bringt, neben der gleichen Verwendung, den Vorteil, dass die vom
Android-Betriebssystem bereitgestellten \verb!SharedPreferences! weiterhin f�r unsensible Daten genutzt werden kann.
\\*
Weiterhin wurde auf die Bibliothek \emph{UndoBar} \cite{soarcn:13} zur�ckgegriffen. Dadurch wird ein einfache und
schnelle M�glichkeit zum Erstellen der \texttt{Toasts} mit \verb!Undo!-Button integriert. Neben der visuellen Komponente
stellt \emph{UndoBar} auch noch die Logik hinter dieser intuitiven Art der Benutzerf�hrung dem Entwickler zur Verf�gung.
Bei Bedarf kann nach dem Verschwinden des \texttt{Toasts} direkt die �nderung speichern oder, wenn der Benutzer die
�nderung r�ckg�ngig machen will, bei Buttonklick den vorigen Zustand wiederherstellen. 
\\*
Zu Testzwecken wurden die Bibliotheken \emph{android-test-kit} \cite{atk:13}, die auch als \emph{Espresso}-Testkit
bekannt ist, und \emph{mockwebserver} \cite{okhttp:14} in das Projekt aufgenommen. In dieser Kombination lassen sich
ohne gro�en Aufwand die Abl�ufe auf der GUI testen, sowie durch gestellte Serverantworten, die Verarbeitung der Daten in
verschiedenen Testf�llen testen. 
\\*
Abgesehen von diesen externen L�sungen, flossen noch Teile der Android-Samples \cite{google:14} mit in die Entwicklung
ein und eine anf�ngliche Hilfestellung bei der Entwicklung bot die Seite \emph{learn2crack.com} \cite{Learn2Crack:13}
konsultiert, die eine Vielzahl an Beispielprojekten und L�sungsans�tzen aufweist.

\begin{comment}
\paragraph{Projektstruktur}\label{para:ProjektStrukt}
Da das Projekt in die neue Entwicklungsumgebung f�r Android, dem \textbf{Android Studio}, migriert wurde, haben sich
einige �nderungen an der Projekt- und Dateistruktur ergeben. In Abbildung \ref{dirtree:AndroidStudio} kann man erkennen,
dass es im Vergleich zur Projektstruktur von \textbf{Eclipse} \ref{dirtree:dbProto} einige Unterschiede gibt. Zum einen
ist die Anzahl der Ordner in den Projekten auf drei geschrumpft, was die �bersichtlichkeit enorm 
\end{comment}

\section{App}\label{section:App}
Dieses Kapitel umfasst die Implementierung der App, auf der die weiteren Teile aufbauen. Dabei wird n�her auf die
service-�hnliche Struktur und deren Umsetzung, sowie die sichere Speicherung von Benutzerinformationen eingegangen. 
Anfangs wird der Ablauf beim ersten Start, den folgenden Starts sowohl mit als auch ohne angemeldetem
Benutzer beschrieben.
\\*
Da eine Verwendung der App ohne Anmeldung, sprich ohne personalisierte Daten, nicht vorgesehen ist, wird zun�chst der
App-Speicher auf vorhandene Login-Informationen �berpr�ft. Aufgrund der Annahme, dass es sich um die erste Verwendung
nach der Installation handelt, k�nnen noch keine Benutzerdaten vorhanden sein und es wird direkt in die
\textsl{LoginActivity} weitergeleitet, die ohne Anmeldung nicht verlassen werden kann. Hat sich der Benutzer erfolgreich
authentifiziert, startet die Synchronisierung aller Tabellen, um die aktuellsten Daten zu erhalten. Ein Beenden ohne
Logout hat zur Folge, dass E-Mailadresse und Passwort �ber die \texttt{Secure-Preferences}-Schnittstelle verschl�sselt
im Speicherbereich abgelegt werden. Dies erfolgt in der \textsl{onPause()}-Methode, um den Verlust der Daten beim
Zerst�ren der App aufgrund von Ressourcenknappheit zu verhindern. Beim darauffolgenden Start wird in der
\textsl{onResume()}-Methode gepr�ft, ob verschl�sselte Credentials im Speicher hinterlegt sind. Nach dem Auslesen werden
damit die aktuellen Bewegungsdaten aus der Datenbank abgerufen und lokal gespeichert. Hat sich der Benutzer vor
dem Beenden der App abgemeldet, wurden beim Schlie�en keine Informationen im App-Speicher abgelegt. Dadurch landet der
Benutzer, wie beim Erststart, wieder in der Loginmaske.

\subsection{Probleme}\label{subsec:AppProbl}
Neben denen in Kapitel \ref{chp:Prototyp} - \nameref{chp:Prototyp} gel�sten Problemen, die schon vor dem Beginn der Implementierung erkannt 
wurden, sind auch w�hrend der Umsetzung einige Stolperfallen aufgetreten. Dazu geh�rt zun�chst die Herausforderung, die
Logininformationen so sicher wie m�glich zu speichern. Des Weiteren sollte die Server- und Datenbankkommunikation
m�glichst unabh�ngig vom Darstellungsteil der App gehalten werden.

\paragraph*{Unsichere App-Speicher}\label{para:unsafeStorage}
Da der \verb!SharedPreference!-Speicher nicht verschl�sselt ist und durch einfache Mittel ausgelesen werden kann, d�rfen
dort keine sensiblen Daten ohne weiteres gespeichert werden. Die sicherste Art w�re es, die Logininformationen nicht zu
speichern, was aber dazu f�hren w�rde, dass sich der Benutzer bei jedem Start der App neu anmelden m�sste. Das ist dem
Benutzer aber unter keinerlei Umst�nden zumutbar.

\paragraph*{Datenbank-Service}\label{para:dbService}
Das Trennen der Oberfl�che von der Datenhaltung hat mehrere Vorteile und wurde deswegen in diesem Projekt
umgesetzt. Zun�chst ist es dadurch m�glich die Kommunikation zwischen App und Datenbank zu �ndern, ohne dass die
Oberfl�che angepasst werden muss. Weiterhin ist es dadurch einfacher die Umsetzung auf verschiedene Entwickler
aufzuteilen, da keine st�ndige R�cksprache n�tig ist, sondern nur zu Beginn die Schnittstelle bereits definiert sein muss.

\subsection{L�sungen}\label{subsec:AppSol}
Zur L�sung der vorangegangenen Probleme, sind nachfolgend kurze Ausz�ge aus dem Quelltext mit einer knappen 
Erl�uterung der Funktion.

\paragraph*{Secure-Preferences}\label{para:secPrefs}
Wie zuvor erw�hnt, ist der App-Speicher nicht verschl�sselt und somit eigentlich f�r die Ablage von Passw�rtern
ungeeignet. Die L�sung dieses Problems bringt der Einsatz einer Verschl�sselung beim Schreiben der \verb!SharedPreferences!.
Dabei kommt die schon erw�hnte Bibliothek \emph{Secure-Preferences} ins Spiel, wodurch die Informationen vor dem
Schreiben in den App-Speicher verschl�sselt werden. Dabei wird einfach die schon vorhandene \verb!SharedPreferences!-Funktion
von Android mit einer extra Schnittstelle dazwischen verwendet, die beim Schreiben ver- und beim Lesen entschl�sselt.
Die Verwendung der \emph{Secure-Preferences} ist einfach und analog zur Verwendung der
Standard-\verb!SharedPreferences!. Das physikalische erstellen der Datei auf dem Datentr�ger geschieht durch das
Instanziieren der Klasse \texttt{SecurePreferences} im \verb!Context! der App. Wie auch bei den \verb!SharedPreferences!
kann auch hier optional ein eigener Name f�r die Datei �bergeben werden, ist hier jedoch nicht notwendig. Hat sich ein
Benutzer angemeldet und wird die App geschlossen, wird die Funktion \textsl{storeCredentials(\ldots{})}
(Zeile \ref{line:secPrefsStore}) aufgerufen und die Referenz zum \emph{SecurePreferences}-Objekt sowie die
\texttt{AuthCredentials} �bergeben. Damit �berhaupt Daten geschrieben werden k�nnen, muss zun�chst ein \texttt{Editor}-Objekt
erstellt werden, was durch \verb!.edit()! in Zeile \ref{line:secPrefsEdit} geschieht. Damit einen eventuell verwaister Eintrag
keine Probleme bereitet, wird der Speicher in der folgenden Zeile sicherheitshalber komplett gel�scht. Danach werden die
Attribute aus dem \texttt{AuthCredentials}-Objekt ausgelesen und zusammen mit einem eindeutigen Bezeichner durch den
\verb!.put(\ldots{})!-Befehl dem \verb!Editor! zum Speichern �bergeben (vgl. Zeile \ref{line:secPrefsPut} ff). Um die
Daten nun physikalisch zu schreiben, wird auf dem \verb!Editor! \verb!.commit()! (vgl. Zeile
\ref{line:secPrefsCommit})
ausgef�hrt. Ausgelesen werden die Daten einfach in der umgekehrten Reihenfolge, mit dem einzigen Unterschied, dass
hierzu kein \verb!Editor! ben�tigt wird. Wird zum Beispiel die App gestartet, ruft die
\textsl{onResume()}-Methode die Methode \textsl{loggedInUser(\ldots{})} in Zeile \ref{line:secPrefsLogged} auf.
Darin wird nach dem Sicherstellen ob Zugangsdaten vorhanden sind, die einzelnen Schl�ssel-Wert-Paare wieder ausgelesen
(siehe Zeile \ref{line:secPrefsGet} ff). Ist keiner der Werte \verb!null!, werden sie in einem \texttt{AuthCredentials}
verpackt zur�ckgegeben.

\begin{figure}[H]
  \centering
  
  \begin{lstlisting}[language=java,escapechar=|]
[|\ldots{}|]
public void resetCredentials(final SecurePreferences secPrefs) {
  Editor secPrefEditor = secPrefs.edit();
  secPrefEditor.clear();
  secPrefEditor.commit();
}
public static void storeCredentials(final SecurePreferences secPrefs,
  AuthCredentials _creds) { |\label{line:secPrefsStore}|
  Editor secPrefEditor = secPrefs.edit(); |\label{line:secPrefsEdit}|
  secPrefEditor.clear();
  secPrefEditor.putString(EnumSqLite.KEY_UID.getName(), 
      _creds.getUid()); |\label{line:secPrefsPut}|
  secPrefEditor.putString(EnumSqLite.KEY_PASSWORD.getName(),
      _creds.getPassword());
  secPrefEditor.putString(EnumSqLite.KEY_EMAIL.getName(), 
      _creds.getEmail());
  secPrefEditor.commit(); |\label{line:secPrefsCommit}|
}
public static AuthCredentials loggedInUser(final SecurePreferences secPrefs) { |\label{line:secPrefsLogged}|
  String uid = null, uname = null, upassword = null, email = null;
  if (!secPrefs.getAll().isEmpty()) {
    uid = secPrefs.getString(EnumSqLite.KEY_UID.getName(), null); |\label{line:secPrefsGet}|
    upassword = secPrefs.getString(EnumSqLite.KEY_PASSWORD.getName(), null);
    email = secPrefs.getString(EnumSqLite.KEY_EMAIL.getName(), null);
  }
  if (uid != null & upassword != null & email != null ) {
    AuthCredentials creds = new AuthCredentials(uid, email, upassword); |\label{line:secPrefsAuth}|
    return creds;
  }
  return null;
}
[|\ldots{}|]
  \end{lstlisting}
  \caption{Verwendung von Secure-Preferences}
  \label{java:UsingSecPrefs}
\end{figure}

Hat man schon mit den \verb!SharedPreferences! gearbeitet, kann man klar die analoge Vorgehenseweise erkennen.
Obwohl die Entwickler keine hunderprozentige Sicherheit garantieren k�nnen und m�chten, ist es dennoch dem
unverschl�sselten Ablegen vorzuziehen.

\paragraph{Messenger-Klasse}\label{para:Messenger}
Wie schon erw�hnt, sollte eine Trennung von Oberfl�chen- und Datenlogik erstrebt werden.
Um diese Trennung zu erreichen, wurde die \texttt{Messenger}-Klasse verwendet \cite{MessengerClass:14}. 
Mit dieser Klasse ist die Implementierung eines \verb!gebundenen Service! einfacher als eine mit einer
\verb!AIDL!-Schnittstelle, erf�llt aber alle Vorraussetzungen die f�r dieses Programm witchtig sind. 
\\*
Der Aufbau der \emph{Messenger}-Schnittstelle ist �bersichtlich und mit wenigen Schritten erreicht. Zun�chst wird eine
\texttt{MessengerService}-Klasse erstellt, die von der \texttt{Service}-Klasse erbt. Somit muss die Methode 
\textsl{onBind(\ldots{})} (vgl. Zeile \ref{line:MsgBinder}) implementiert werden, welche als \verb!Binder!
eine Instanz der inneren Klasse \texttt{IncomingHandler} zur�ckgibt (vgl. Zeile \ref{line:newMessngr}).
Der \texttt{IncomingHandler} arbeitet die eingehenden Anfragen 
seriell ab und f�hrt mittels einer \verb!switch-case!-Anweisung die erw�nschten
Operationen aus (wie in Zeile \ref{line:handleMsg} ff zu sehen).

\begin{figure}[H]
  \centering
  \begin{lstlisting}[language=java,escapechar=|]
[|\ldots{}|]
  private final Messenger mMessenger = new Messenger(new IncomingHandler()); |\label{line:newMessngr}|
[|\ldots{}|]
@Override
public IBinder onBind(Intent intent) { |\label{line:MsgBinder}|
  return mMessenger.getBinder();
}
public class IncomingHandler extends Handler {

  public static final String TAG = Constants.TAG_PREFIX + "IncomingHandler";

  @Override
  public void handleMessage(Message msg) { |\label{line:handleMsg}|
    // TODO
    String[] tablesToSync;
    int ppAufgId;
    Bundle _bundle;
    Map<String, String> params;
    int remItemId, shoppingListId;
    switch (msg.what) {
        case MessageConstants.MSG_UNREBIND:
            reService = null;
            reBound = false;
            break;
      {|\ldots{}|}
     }
     {|\ldots{}|}
   }
   {|\ldots{}|}
}
[|\ldots{}|]
  \end{lstlisting}
  \caption{Auszug aus MessengerService.java}
  \label{java:MessengerService}
\end{figure}

Neben den erw�hnten Methoden enth�lt die Klasse \texttt{MessengerService} au�erdem noch einige Hilfsmethoden, die zum
Beispiel zum Entpacken der \verb!Bundles! verwendet werden.

\section{Putzplan}\label{section:Putzplan}

\subsection{Umsetzung}\label{subSec:PPUmsetz}

\subsection{Probleme und L�sungen}\label{subSec:PPSol}



\section{Einkaufsliste}\label{section:Einkaufsliste}
Die Einkaufsliste ist eine vom Benutzer sortierte Liste von Einkaufsgegenst�nden. Es k�nnen von jedem WG Mitglied Gegenst�nde hinzugef�gt und entfernt werden. Abschlie�end soll dadurch die Abrechnung vereinfacht werden, da den
gekauften Artikel jeweils der K�ufer, als auch der bezahlte Preis zugeordnet werden kann.

\subsection{Implementierung}\label{subSec:EkLImpl}
Wie die Aufgaben wurde auch f�r die Einkaufliste eine Klasse \texttt{ShoppingListFragment} von der Klasse
\verb!Fragment! abgeleitet. Da hierbei bis einschlie�lich zum Einsatz des \texttt{ShoppingListAdapter} analog zum
\texttt{ChorePlanFragment} vorgegangen wurde, ist eine erneute Ausf�hrung nicht notwendig. Im Vergleich dazu wurde
jedoch ein \verb!AutoCompleteTextView! verwendet. Da meist die gleichen Artikel hinzugef�gt werden m�ssen, ist der
Einsatz der Autovervollst�ndigung hier eine gro�e Erleichterung. Dazu werden zun�chst die gew�nschten Vorschl�ge aus der
Datenbank in ein \verb!String-Array! geladen (vgl. Zeile \ref{line:getGroceries}). Danach wird ein \verb!ArrayAdapter!
erzeugt, dem das \verb!Array! mit den Vorschl�gen, sowie ein Layout �bergeben werden. Dieser Adapter wird dem
vorbereiteten \verb!AutoCompleteTextView! auf dem Einkaufslisten-\verb!Fragment! gesetzt. Als weitere Erleichterung
wurde ein \verb!Listener! zur automatischen Eingabebest�tigung implementiert (Zeile \ref{line:onEditorListener}),
welcher beim Hinzuf�gen von Waren zum Einsatz kommt, die noch nicht als Vorschlag verf�gbar hinterlegt sind.
Dazu kommt der \texttt{TextView.OnEditorActionListener()} zum Einsatz und f�ngt das Dr�cken der Tasten auf der Tastatur
ab und �berpr�ft ob es sich dabei um die \verb!Senden!-Taste handelt (siehe Zeile \ref{line:actionIdVergleich}).
In diesem Fall wird die Weitergabe des Events unterbrochen und die \textsl{handleItemAddAction(\ldots{})} aufgerufen
(vgl. Zeile \ref{line:handleItemAdd}, der der eingegebene Text �bergeben wird und diesen als neue Auswahl zur Verf�gung
stellt, sowie ein Dialog �ffnet mit dem die ben�tigte Anzahl f�r die Einkaufsliste �bergeben werden kann.\\

\noindent
\begin{minipage}{\linewidth}
\begin{lstlisting}[caption=ShoppingListFragment.java, captionpos=below, label=java:ShoppingListFrag,
language=java, escapechar=|]
[|\ldots{}|]
final String[] grocieries = dbHandler.getGrocieries(); |\label{line:getGroceries}|
ArrayAdapter<String> adapter = new ArrayAdapter<String>(getActivity(),
    android.R.layout.simple_list_item_1, grocieries);
final AutoCompleteTextView editTextNew =
    (AutoCompleteTextView) rootView.findViewById(R.id.autoCompleteShoppingListNewItem);
editTextNew.setAdapter(adapter);

editTextNew.setOnEditorActionListener(new TextView.OnEditorActionListener() { |\label{line:onEditorListener}|
  @Override
  public boolean onEditorAction(TextView v, int actionId, KeyEvent event) {
    boolean handled = false;
    if (actionId == EditorInfo.IME_ACTION_SEND) { |\label{line:actionIdVergleich}|
      handleItemAddAction(v.getText().toString()); |\label{line:handleItemAdd}|
      handled = true;
    }
    return handled;
} });
[|\ldots{}|]
\end{lstlisting}  
\end{minipage}      

Dieser Teil der App konnte ohne weitere Probleme gel�st werden, weswegen auf die hier �blichen Paragraphen \textsc{Probleme}
und \textsc{L�sungen} verzichtet wird.


\section{Blackboard}\label{section:Blackboard}
Eine einfache M�glichkeit alle WG-Mitglieder zu erreichen bietet ein Blackboard auf dem jeder Nachrichten hinterlassen
kann. Obwohl der Einsatz von Zugriffsbeschr�nkungen auf die Nachrichten leicht umsetzbar w�re,
wurde bewusst darauf verzichtet, um die Eigentschaften eines physikalischen Blackboards gerecht zu werden.

\subsection{Implementierung}\label{subSec:BbImpl}
Die Klassenhierarchie der Fragmente ist aus den vorhergehenden Beispielen schon bekannt und wurde auch in beim
\texttt{BlackboardFragment} beibehalten. Dabei wird in der �berschriebenen Methode \textsl{onCreateView(\ldots{})} durch
ein SQL-Statement die Nachrichten aus der \verb!SQLite!-Datenbank gelesen, welche in einem \verb!Cursor! vorgehalten
werden. Da es sich bei dem Zeilenlayout der \verb!ListView! um kein Standard-Layout handelt, muss zun�chst der
\verb!Cursor! schrittweise durchgearbeitet werden und die Werte in eine \verb!HashMap! �bertragen werden. Auf eine
\verb!HashMap! wurde zur�ckgegriffen, um beim L�schen des Eintrags einfach �ber die \textit{BlackboardId} an die
Nachricht zu gelangen und aus der Liste zu l�schen, das spart einen direkten Datenbankzugriff f�r das L�schen und
weitere Zugriffe beim Aktualisieren der Liste, sowie dem etwaigen Wiederherstellen der Nachricht.
�ber eine \verb!EditText!-Feld kann die neue, mehrzeilige Nachricht eingegeben werden und durch den
\textit{+}-Button dem schwarzen Brett hinzugef�gt werden. Das L�schen der einzelnen Nachrichten kann durch ein Klick auf
das L�schen-Symbol ausgel�st werden und ist durch die \verb!UndoBar!-Funktion revidierbar.
Zum Bearbeiten wurde ein \verb!onLongClickListener! an die \verb!View! der Zeile gebunden. Wird dieser ausgel�st, so
generiert er einen Dialog mit dem aktuellen Inhalt der Nachricht und zeigt diesen an. Gespeichert wird dann die
Nachricht mit der ID des Bearbeiters. Eine Erweiterung um die \verb!UndoBar!-Funktion sollte hier noch erg�nzt werden.

\paragraph{Probleme}\label{para:BbProbl}
Bei der Umsetzung des Blackboards kam es beim Anzeigen der Nachrichten zu einem komplexen Problem, was zun�chst nicht
nachvollziehbar war. In der \verb!ListView! war zwar die Anzahl der Eintr�ge richtig, aber der Anfang der Liste wurde am 
Ende der Liste, also bei Zeilen die ausserhalb des anf�nglich darstellbaren Bereichs lagen, wiederholt.

\paragraph{L�sungen}\label{para:BbSol}
Um die Datenquelle als Fehler auszuschlie�en wurde zun�chst ein \verb!DISTINCT! in das \verb!SQL!-Statement eingef�gt.
Das bewirkt, dass doppelte Eintr�ge ausgefiltert werden. Das war aber nicht die Ursache des Problems, da die erwarteten
Daten im \verb!Cursor! waren und auch an den \texttt{BlackboardAdapter} weitergegeben wurden. Somit lies sich das
Problem auf die Anzeige, beziehungsweise das die Vorbereitung der Daten zu Anzeige, eingrenzen. Demnach muss sich der
Fehler im erw�hnten \texttt{BlackboardAdapter} befinden. Nachdem weitere Gedanken �ber die Funktion des Adapters gemacht
wurden, kam die Erkenntnis, dass die Zeile \ref{line:WrongLineBbAdap} im alten Quelltext \ref{java:BbAdapOld} nicht
funktionieren kann, sobald es mehr Eintr�ge gibt, als auf anhieb anzeigbar sind.

\begin{lstlisting}[float, language=java, caption=Alte BlackboardAdapter.java, captionpos=below, label=java:BbAdapOld,
escapechar=|]
[|\ldots{}|]
@Override
public View getView(int position, View convertView, ViewGroup parent) { |\label{line:getView}|
  if (convertView &=&  null) { |\label{line:WrongLineBbAdap}|
    final BlackboardMessage bbMsg =
        (BlackboardMessage) blackboardMessages.values().toArray()[position];
  [|\ldots{}|] }
}
[|\ldots{}|]
\end{lstlisting}

Mit der \verb!if!-Abfrage, ob die �bergebene \verb!View! noch \verb!null! ist, wird verhindert, dass wenn die
\verb!ListView! gescrollt wird, die neue Zeile �berschrieben werden kann. In der \verb!ListView! befinden sich n�mlich
immer gleich viele \verb!Views! als Zeilen, vorrausgesetzt dass mehr Eintr�ge dargestellt werden sollen als auf den
sichtbaren Bereich passen. Tritt der Fall ein dass neue Zeilen angezeigt werden m�ssen, sprich es wird gescrollt, werden
die angezeigten \verb!View!s der \textsl{getView(int position, View convertView, ViewGroup parent)} (vgl. Zeile
\ref{line:getView} im \nameref{java:BbAdapOld}~\ref{java:BbAdapOld}) als \textit{converView} �bergeben. Somit kann diese
Variable nicht \verb!null! sein und die Bedingung der \verb!if!-Abfrage ist falsch. Demzufolge k�nnen die alten 
Daten nicht in den vorhandenen \verb!View!s durch die neuen ersetzt werden und die selbe Nachricht wird noch mal
angezeigt. Die Wiederholung des Anfangs wird dadurch erzeugt, dass die Zeilen-\verb!View!s wiederverwendet werden, die
aus dem angezeigten Bereich geschoben werden, also die zuvor erste \verb!View!.\\*
Durch das Entfernen der \verb!if!-Abfrage wurde die erw�nschte Funktion erreicht und das Scrollen der Liste war
m�glich.

\section{Weboberf�che}\label{section:WebUI}
Der Administrator einer WG muss in der Lage sein, die Eigenschaften der WG zu konfigurieren und die Funktionen zu verwalten. Wir haben uns dazu entschieden, eine extern jederzeit erreichbare Weboberfl�che daf�r bereit zu stellen. Wir h�tten uns genauso gut f�r die Implementierung in die Android App entscheiden k�nnen, haben uns jedoch bewusst dagegen entschieden. Wir m�chten mit den unterschiedlichen Programmiersprachen und den daraus resultierenden Herangehensweisen eine m�glichst gro�es Vielfalt der Informatik widerspiegeln und die Arbeiten auf die Mitglieder verteilen, die bisher noch keine Erfahrung mit der Programmierung f�r die Android Plattform sammeln konnten.\\
Im folgenden Kapitel wird die Weboberfl�che mit Hilfe von Codesnippets erl�utert.

\subsection{Umsetzung}\label{subSec:WebUmsetz}
Als Programmiersprache haben wir uns f�r die sehr beliebte und weit verbreitete Skriptsprache PHP entschieden. PHP bietet durch den Einfluss von Java, C++ und Perl einen leichten Einstieg f�r diejenigen, die bereits erste Erfahrungen mit einer der Programmiersprachen sammeln konnten. Ausserdem bietet PHP die einfache Umsetzung von dynamischen Webseiten und eine sehr gute Unterst�tzung von Datenbankverbindungen. Mit PHP ist automatisch sichergestellt, dass die Weboberfl�che von jedem g�ngigen Browser aus in deren Desktop- sowie Mobilversion genutzt werden kann.\\
\begin{description}
\item Die Weboberfl�che gliedert sich in 11 Seiten:
\begin{itemize}
	\item admin.php
	\item benutzer.php
	\item blackboard.php
	\item einkaufsliste.php
	\item login.php
	\item logout.php
	\item admin.php
	\item putzplan.php
	\item regist.php
	\item admin.php
	\item system.php
\end{itemize}
\end{description}
Jede dieser PHP Dateien beinhaltet gew�hnlichen HTML Code um dem Browser die Daten pr�sentieren und PHP Code f�r den dynamischen Teil der Datenabfrage von der Datenbank.

\chapter{Implementierung von \textsl{\textbf{DorMApp}}}\label{chp:Impl}
Die App wurde zum Gro�teil in der aktuell von Google empfohlenen Umgebung \emph{Android Studio} \cite{androidStudio:15} umgesetzt. Zu Beginn fand die Entwicklung noch in \emph{Eclipse} \cite{eclipse:14} mit dem entsprechenden Plug-In \emph{Android Developer Tools} \cite{eclipsePlugIn:14} statt. Jedoch erschwerten die Bugs und die Beh�bigkeit der Eclipse-IDE den z�gigen 
Fortschritt. Aus diesem Grund wurde nach dem Legen des Grundsteins das Projekt auf die
neue Entwicklungsumgebung migriert, wo es auch fertiggestellt wurde. \\*

\section{Entwicklungsgrundlagen}\label{section:ImplGrundlagen}
Durch die Prototyp-Entwicklung konnte zu Beginn der Implementierung schon auf einige Bausteine zur�ckgegriffen werden.
Zun�chst wurde die Kommunikation mit der Datenbank nicht ver�ndert. Somit ruft die App weiterhin ein PHP-Skript auf,
welches dann die Tabelleninhalte ausliest, aufbereitet und im \emph{JSON}-Format zur�cksendet. 
Die von der \textsl{\textbf{DorMApp}}-App ben�tigten Skripte, sowie der Zugriff auf die Datenbank werden im Kapitel \ref{section:AppDbInterface} - \nameref{section:AppDbInterface} genauer erl�utert. Sollte es das Verst�ndnis eines Sachverhaltes 
fordern, so wird auf das entsprechende Kapitel vorgegriffen und mit in die Erkl�rung aufgenommen.

\paragraph*{Externe Bibliotheken}\label{para:ExtLibs}
Um Entwicklungszeit zu sparen, wurden vier externe Bibliotheken eingesetzt. Um eine bessere Sicherheit zu erreichen und
die Logindaten der Benutzer unverschl�sselt im \verb!SharedPreferences!-Bereich der App abzulegen, wurde die Bibliothek
\emph{secure-preferences} \cite{scottyab:14} eingebunden. Diese Bibliothek stellt eine Schnittstelle f�r die
eigentlichen \verb!SharedPreferences! bereit, die zwar die gleichen Methoden bereitstellt, aber alle Daten
verschl�sselt im App-Speicherbereich ablegt. Das bringt, neben der gleichen Verwendung, den Vorteil, dass die vom
Android-Betriebssystem bereitgestellten \verb!SharedPreferences! weiterhin f�r unsensible Daten genutzt werden k�nnen.
\\*
Weiterhin wurde auf die Bibliothek \emph{UndoBar} \cite{soarcn:13} zur�ckgegriffen. Dadurch wird eine einfache und
schnelle M�glichkeit zum Erstellen der \texttt{Toasts} mit einem \verb!Undo!-Button integriert. Neben der visuellen Komponente
stellt \emph{UndoBar} auch die Logik hinter dieser intuitiven Art der Benutzerf�hrung dem Entwickler zur Verf�gung.
Bei Bedarf kann nach dem Verschwinden des \texttt{Toasts} direkt die �nderung gespeichert oder, wenn der Benutzer die
�nderung r�ckg�ngig machen will, bei Buttonklick den vorherigen Zustand wiederhergestellt werden. 
\\*
Zu Testzwecken wurden die Bibliotheken \emph{android-test-kit} \cite{atk:13}, die auch als \emph{Espresso}-Testkit
bekannt ist, und \emph{mockwebserver} \cite{okhttp:14} in das Projekt aufgenommen. In dieser Kombination lassen sich
ohne gro�en Aufwand die Abl�ufe auf der GUI testen, sowie durch gestellte Serverantworten, die Verarbeitung der Daten in
verschiedenen Testf�llen testen. 
\\*
Abgesehen von diesen externen L�sungen, flossen noch Teile der Android-Samples \cite{google:14} mit in die Entwicklung
ein. Eine anf�ngliche Hilfestellung bei der Entwicklung bot die Seite \emph{learn2crack.com} \cite{Learn2Crack:13}, die eine Vielzahl an Beispielprojekten und L�sungsans�tzen aufweist.

\begin{comment}
\paragraph{Projektstruktur}\label{para:ProjektStrukt}
Da das Projekt in die neue Entwicklungsumgebung f�r Android, dem \textbf{Android Studio}, migriert wurde, haben sich
einige �nderungen an der Projekt- und Dateistruktur ergeben. In Abbildung \ref{dirtree:AndroidStudio} kann man erkennen,
dass es im Vergleich zur Projektstruktur von \textbf{Eclipse} \ref{dirtree:dbProto} einige Unterschiede gibt. Zum einen
ist die Anzahl der Ordner in den Projekten auf drei geschrumpft, was die �bersichtlichkeit enorm 
\end{comment}

\section{App}\label{section:App}
Dieses Kapitel umfasst die Implementierung der App, auf der die weiteren Teile aufbauen. Dabei wird n�her auf die
service-�hnliche Struktur und deren Umsetzung, sowie die sichere Speicherung von Benutzerinformationen eingegangen. 
Anfangs wird noch der Ablauf beim ersten Start, den folgenden Starts sowohl mit als auch ohne angemeldetem
Benutzer beschrieben.
\\*
Da eine Verwendung der App ohne Anmeldung, sprich ohne personalisierte Daten, nicht vorgesehen ist, wird zun�chst der
App-Speicher auf vorhandene Login-Informationen �berpr�ft. Aufgrund der Annahme, dass es sich um die erste Verwendung
nach der Installation handelt, k�nnen noch keine Benutzerdaten vorhanden sein und es wird direkt in die
\textsl{LoginActivity} weitergeleitet, die ohne Anmeldung nicht verlassen werden kann. Hat sich der Benutzer erfolgreich
authentifiziert, startet die Synchronisierung aller Tabellen um die aktuellsten Daten zu erhalten. Ein Beenden ohne
Logout hat zur folge, dass E-Mailadresse und Passwort �ber die \texttt{Secure-Preferences}-Schnittstelle verschl�sselt
im Speicherbereich abgelegt werden. Die erfolgt in der \textsl{onPause()}-Methode, um den Verlust der Daten beim
Zerst�ren der App aufgrund von Ressourcenknappheit, zu verhindern. Beim darauffolgenden Start wird in der
\textsl{onResume()}-Methode gepr�ft, ob verschl�sselte Credentials im Speicher hinterlegt sind. Nach dem Auslesen werden
damit direkt die aktuellen Bewegungsdaten aus der Datenbank abgerufen und lokal gespeichert. Hat sich der Benutzer vor
dem Beenden der App abgemeldet, wurden beim Schlie�en keine Informationen im App-Speicher abgelegt. Dadurch landet der
Benutzer, wie beim Erststart, wieder in der Loginmaske.

\subsection{Probleme}\label{subsec:AppProbl}
Neben denen im Prototyp \ref{chp:Prototyp} gel�sten Problemen, die schon vor dem Beginn der Implementierung erkannt 
wurden, sind auch w�hrend der Umsetzung einige Stolperfallen aufgetreten. Dazu geh�rt zun�chst das Problem, die
Logininformationen so sicher wie m�glich zu speichern. Desweiteren sollte die Server- und Datenbankkommunikation
m�glichst unabh�ngig vom Darstellungsteil der App gehalten werden.

\paragraph*{Unsichere App-Speicher}\label{para:unsafeStorage}
Da der \verb!SharedPreference!-Speicher nicht verschl�sselt ist und durch einfache Mittel ausgelesen werden kann, k�nnen
dort keine sensiblen Daten ohne weiteres gespeichert werden. Die sicherste Art w�re es, die Logininformationen nicht zu
speichern, was aber dazu f�hren w�rde, dass sich der Benutzer bei jedem Start der App neu anmelden m�sste. Das ist dem
Benutzer aber unter keinerlei Umst�nden zumutbar.

\paragraph*{Datenbank-Service}\label{para:dbService}
Das Trennen der Oberfl�che von der Datenhaltung hat mehrere Vorteile und wurde deswegen auch in diesem Projekt
umgesetzt. Zun�chst ist es dadurch m�glich die Kommunikation zwischen App und Datenbank zu �ndern, ohne dass die
Oberfl�che angepasst werden muss. Weiterhin ist es dadurch einfacher m�glich die Umsetzung auf verschiedene Entwickler
aufzuteilen, da keine st�ndige R�cksprache n�tig ist, sondern nur zu Beginn die Schnittstelle definiert werden muss.

\subsection{L�sungen}\label{subsec:AppSol}
Zur L�sung der vorangegangenen Probleme, sind nachfolgend kurze Ausz�ge aus dem Quelltext mit einer knappen 
Erl�uterung der Funktion

\paragraph*{Secure-Preferences}\label{para:secPrefs}
Wie zuvor erw�hnt, ist der App-Speicher nicht verschl�sselt und somit eigentlich f�r die Ablage von Passw�rtern
ungeeignet. Die L�sung dieses Problems bringt der Einsatz einer Verschl�sselung beim Schreiben der \verb!SharedPreferences!.
Dabei kommt die schon erw�hnte Bibliothek \emph{Secure-Preferences} ins Spiel, wodurch die Informationen vor dem
Schreiben in den App-Speicher verschl�sselt werden. Dabei wird einfach die schon vorhandene \verb!SharedPreferences!-Funktion
von Android mit einer extra Schnittstelle dazwischen verwendet, die beim Schreiben ver- und beim Lesen entschl�sselt.
Die Verwendung der \emph{Secure-Preferences} ist einfach und analog zur Verwendung der
Standard-\verb!SharedPreferences!. Das physikalische erstellen der Datei auf dem Datentr�ger geschieht durch das
Instanziieren der Klasse \texttt{SecurePreferences} im \verb!Context! der App. Wie auch bei den \verb!SharedPreferences!
kann auch hier optional ein eigener Name f�r die Datei �bergeben werden, ist hier jedoch nicht notwendig. Hat sich ein
Benutzer angemeldet und wird die App geschlossen, wird die Funktion \textsl{storeCredentials(\ldots{})}
\ref{line:secPrefsStore} aufgerufen und die Referenz zum \emph{SecurePreferences}-Objekt �bergeben, sowie die
\texttt{AuthCredentials}. Damit �berhaupt Daten geschrieben werden k�nnen, muss zun�chst ein \texttt{Editor}-Objekt
erstellt werden, was durch \verb!.edit()! in Zeile \ref{line:secPrefsEdit} geschieht. Damit einen eventuell verwaister Eintrag
keine Probleme bereitet, wird der Speicher in der folgenden Zeile sicherheitshalber komplett gel�scht. Danach werden die
Attribute aus dem \texttt{AuthCredentials}-Objekt ausgelesen und zusammen mit einem eindeutigen Bezeichner durch den
\verb!.put(\ldots{})!-Befehl dem \verb!Editor! zum Speichern �bergeben (vgl. Zeile \ref{line:secPrefsPut} ff). Um die
Daten nun physikalisch zu schreiben, wird auf dem \verb!Editor! der \verb!.commit()! (vgl. Zeile
\ref{line:secPrefsCommit})
ausgef�hrt. Ausgelesen werden die Daten einfach in der umgekehrten Reihenfolge, mit dem einzigen Unterschied, dass
hierzu kein \verb!Editor! ben�tigt wird. Wird zum Beispiel die App gestartet, ruft die
\textsl{onResume()}-Methode die \textsl{loggedInUser(\ldots{})} in Zeile \ref{line:secPrefsLogged} auf.
Darin wird nach dem Sicherstellen ob Zugangsdaten vorhanden sind, die einzelnen Schl�ssel-Wert-Paare wieder ausgelesen
(siehe Zeile \ref{line:secPrefsGet} ff). Ist keiner der Werte \verb!null!, werden sie in einem \texttt{AuthCredentials}
verpackt zur�ckgegeben.

\begin{figure}
  \centering
  
  \begin{lstlisting}[language=java,escapechar=|]
[|\ldots{}|]
public void resetCredentials(final SecurePreferences secPrefs) {
  Editor secPrefEditor = secPrefs.edit();
  secPrefEditor.clear();
  secPrefEditor.commit();
}
public static void storeCredentials(final SecurePreferences secPrefs,
  AuthCredentials _creds) { |\label{line:secPrefsStore}|
  Editor secPrefEditor = secPrefs.edit(); |\label{line:secPrefsEdit}|
  secPrefEditor.clear();
  secPrefEditor.putString(EnumSqLite.KEY_UID.getName(), 
      _creds.getUid()); |\label{line:secPrefsPut}|
  secPrefEditor.putString(EnumSqLite.KEY_PASSWORD.getName(),
      _creds.getPassword());
  secPrefEditor.putString(EnumSqLite.KEY_EMAIL.getName(), 
      _creds.getEmail());
  secPrefEditor.commit(); |\label{line:secPrefsCommit}|
}
public static AuthCredentials loggedInUser(final SecurePreferences secPrefs) { |\label{line:secPrefsLogged}|
  String uid = null, uname = null, upassword = null, email = null;
  if (!secPrefs.getAll().isEmpty()) {
    uid = secPrefs.getString(EnumSqLite.KEY_UID.getName(), null); |\label{line:secPrefsGet}|
    upassword = secPrefs.getString(EnumSqLite.KEY_PASSWORD.getName(), null);
    email = secPrefs.getString(EnumSqLite.KEY_EMAIL.getName(), null);
  }
  if (uid != null & upassword != null & email != null ) {
    AuthCredentials creds = new AuthCredentials(uid, email, upassword); |\label{line:secPrefsAuth}|
    return creds;
  }
  return null;
}
[|\ldots{}|]
  \end{lstlisting}
  \caption{Verwendung von Secure-Preferences}
  \label{java:UsingSecPrefs}
\end{figure}

Hat man schon mit den \verb!SharedPreferences! gearbeitet, kann man klar die analoge Vorgehenseweise erkennen.
Obwohl die Entwickler keine hunderprozentige Sicherheit garantieren k�nnen und m�chten, ist es dennoch dem
unverschl�sselten Ablegen vorzuziehen.

\paragraph{Messenger-Klasse}\label{para:Messenger}
Wie schon erw�hnt, sollte eine Trennung von Oberfl�chen- und Datenlogik erstrebt werden.
Um diese Trennung zu erreichen, wurde die \texttt{Messenger}-Klasse verwendet \cite{MessengerClass:14}. 
Mit dieser Klasse ist die Implementierung eines \verb!gebundenen Service! einfacher als eine mit einer
\verb!AIDL!-Schnittstelle, erf�llt aber alle Vorraussetzungen die f�r dieses Programm witchtig sind. 
\\*
Der Aufbau der \emph{Messenger}-Schnittstelle ist �bersichtlich und mit wenigen Schritten erreicht. Zun�chst wird eine
\texttt{MessengerService}-Klasse erstellt, die von der \texttt{Service}-Klasse erbt. Somit muss die Methode 
\textsl{onBind(\ldots{})} (vgl. Zeile \ref{line:MsgBinder}) implementiert werden, welche als \verb!Binder!
eine Instanz der inneren Klasse \texttt{IncomingHandler} zur�ckgibt (vgl. Zeile \ref{line:newMessngr}.
Der \texttt{IncomingHandler} arbeitet die eingehenden Anfragen 
seriell ab und f�hrt mittels einer \verb!switch-case! die erw�nschten
Operationen aus wie in Zeile \ref{line:handleMsg} ff zu sehen ist.

\begin{figure}
  \centering
  \begin{lstlisting}[language=java,escapechar=|]
[|\ldots{}|]
  private final Messenger mMessenger = new Messenger(new IncomingHandler()); |\label{line:newMessngr}|
[|\ldots{}|]
@Override
public IBinder onBind(Intent intent) { |\label{line:MsgBinder}|
  return mMessenger.getBinder();
}
public class IncomingHandler extends Handler {

  public static final String TAG = Constants.TAG_PREFIX + "IncomingHandler";

  @Override
  public void handleMessage(Message msg) { |\label{line:handleMsg}|
    // TODO
    String[] tablesToSync;
    int ppAufgId;
    Bundle _bundle;
    Map<String, String> params;
    int remItemId, shoppingListId;
    switch (msg.what) {
        case MessageConstants.MSG_UNREBIND:
            reService = null;
            reBound = false;
            break;
      {|\ldots{}|}
     }
     {|\ldots{}|}
   }
   {|\ldots{}|}
}
[|\ldots{}|]
  \end{lstlisting}
  \caption{Auszug aus MessengerService.java}
  \label{java:MessengerService}
\end{figure}

Neben den erw�hnten Methoden enth�lt die Klasse \texttt{MessengerService} au�erdem noch einige Hilfsmethoden, die zum
Beispiel zum Entpacken der \verb!Bundles! verwendet werden.

\section{Putzplan}\label{section:Putzplan}
Der Putzplan soll den Benutzern zeigen, wer als n�chstes f�r eine Aufgabe an der Reihe ist. Dabei werden bei der
Synchronisierung die f�r die gesamte WG anfallenden Aufgaben kopiert. Somit ist es auch m�glich die Arbeiten eines
anderen WG-Mitbewohners abzuarbeiten. F�r die einzelnen Aufgaben, wie K�che oder Badezimmer, k�nnen noch Schritte
definiert werden. Diese sind zu erledigen, bevor die Aufgabe als erledigt angesehen wird. Sobald alle Schritte markiert 
sind, wird beim �bertragen automatisch die Aufgabe auf erledigt gesetzt.

\subsection{Umsetzung}\label{subSec:PPUmsetz}
Die Anzeige der Aufgaben wurde durch ein, von \verb!Fragment! abgeleitetes, \\* 
\texttt{ChorePlanFragment} realisiert. Wechselt man zu dem Besagten
\verb!Fragment!, wird zun�chst das passende Layout geladen (vgl. Zeile
\ref{line:inflateFragChore}). Danach wird aus der \textsl{onViewCreated(\ldots{})} der Inhalt des Fragments
initialisiert. Dabei wird sowohl f�r die Aufgaben als auch die Schritte eine \verb!SQL!-Statement auf der
\verb!SQLite!-Datenbank ausgef�hrt und in einen Cursor geladen. Die aus der Datenbank geladenen Daten werden
anschlie�end in eine \verb!ArrayList! hinzugef�gt, beziehungsweise in eine \verb!HashMap! gesetzt. Diese k�nnen dann dem
\texttt{ChorePlanAdapter} �bergeben werden, der die Daten f�r die Anzeige aufbereitet und dem \texttt{ListView}-Element
im \texttt{ChorePlanFragment} anh�ngt. 
Schlussendlich wird dem \textit{Erledigt}-Button die Funktion hinterlegt, die Schritte mit ge�nderter Markierung aus dem
\texttt{ChorePlanAdapter} auszulesen und in der lokalen Datenbank zu speichern.

\begin{lstlisting}[float,caption=ChorePlanFragment.java,label=java:ChorePlanFrag2,language=java,escapechar=|,captionpos=below]
[|\ldots{}|]
@Override
public View onCreateView(LayoutInflater inflater, ViewGroup container, Bundle savedInstanceState) {
  rootView = inflater.inflate(R.layout.fragment_chores, container, false);  |\label{line:inflateFragChore}|
  return rootView;
}
[|\ldots{}|]
cpAdapter = new ChorePlanAdapter(getActivity(), chores, steps);
lvChorePlan.setAdapter(cpAdapter);
query = generateStepsQueryString();

try {
    result = dbHandler.getCursorForQuery(query, null);
}
catch (SQLiteException sqe) {
    sqe.printStackTrace();
    Log.e(TAG, sqe.getLocalizedMessage());
}
if (result != null && result.moveToFirst()) {
[|\ldots{}|]
final Button btnChoreDone = (Button) rootView.findViewById(R.id.btnChoreDone);
btnChoreDone.setOnClickListener(new View.OnClickListener() {
  private ArrayList<ChoreStepItem> selectedSteps = new ArrayList<ChoreStepItem>();

  @Override
  public void onClick(View v) {
    Map<Integer, ChoreStepItem> steps = cpAdapter.getSelectedSteps();
    ChorePlanStep commitStep = new ChorePlanStep(getActivity());

    ArrayList<Integer> ids = new ArrayList<Integer>();
    ArrayList<Integer> choreStepIds = new ArrayList<Integer>();
    for (ChoreStepItem step : steps.values()) {
        Long longId = commitStep.doChoreStep(step.getChorePlanId(), step.getChoreStepId(), step.getChoreId()) ? step.getChoreStepId() : 0L;
        ids.add(Integer.parseInt(String.valueOf(longId)));
        choreStepIds.add(step.getChoreStepId());
        selectedSteps.add(step);
    }
    Bundle token = new Bundle();
    token.putIntegerArrayList("insertedRowIds", ids);
    token.putIntegerArrayList("choreStepId", choreStepIds); |\label{line:lastChorePlanLine}|
[|\ldots{}|]
\end{lstlisting}

\subsection{Probleme und L�sungen}\label{subSec:PPSol}
Bei diesem Teil der App kam es an zwei Stellen zu leichten Problemen. Dazu z�hlt zum einen das Aufklappen der
\verb!ListView!-Zeilen, zum anderen war das nicht-persistente �ndern eine kleine Herausforderung.

\paragraph*{Probleme}\label{para:ProbChorePlanFrag}
Um die Liste der anfallenden Aufgaben �bersichtlicht zu halten, aber trotzdem die Schritte bei den zugeh�rigen Aufgaben
anzuzeigen, bewerkstelligen zu k�nnen, wurden die einzelnen \verb!ListView!-Elemente klappbar gemacht. Somit klappt beim
klick auf eine Zeile der Teil mit den Aufgabeschritten auf. Womit die Zusammengeh�rigkeit symbolisiert wird und die
Bedienbarkeit intuitiv ist.\\*
Wie im Sourcecode-Auszug \ref{java:ChorePlanFrag2} in der letzten Zeile zu erkennen ist, werden dort noch keine Daten
geschrieben, sondern nur ein \verb!Bundle! mit den ge�nderten Daten erzeugt. An diese Stelle soll die Funktion des
\emph{Undo}-Buttons kommen. Das bedeutet, eine Art \verb!Toast!, der erzeugt wird sobald der Benutzer Daten ge�ndert
hat, und �ber einen Button verf�gt, mit dem die �nderungen wieder r�ckg�ngig gemacht werden k�nnen. Bis dato stellt die
\textsc{android}-API keine derartige Funktion direkt bereit, noch wird auf der Developer-Homepage \cite{google:14} ein L�sung f�r das
Problem angeboten. 

\paragraph*{L�sungen}\label{para:ChorePlanSol}
Um ein Aufklappen des Eintrags zu simulieren wurde zun�chst eine Animations-Klasse \ref{java:ExpandAnim} geschrieben.
Die sorgt daf�r, dass die �bergebene \verb!View! �ber eine angegebene Zeitspann hinweg aufgeklappt wird (Zeile
\ref{line:ConstructExpandAnim}). 

\begin{lstlisting}[float, caption=ExpandAnimation.java, captionpos=below, label=java:ExpandAnim, language=java,
escapechar=|]
\centering
public ExpandAnimation(View view, int duration) { |\label{line:ConstructExpandAnim}|
  setDuration(duration);
  mAnimatedView = view;
  mViewLayoutParams = (LayoutParams) view.getLayoutParams();

  mIsVisibleAfter = (view.getVisibility() == View.VISIBLE);

  mMarginStart = mViewLayoutParams.bottomMargin;
  mMarginEnd = (mMarginStart == 0 ? (0- view.getHeight()) : 0);

  view.setVisibility(View.VISIBLE);
}
\end{lstlisting}

Diese Animation wird bei einem Klick auf einen beliebigen Punkt in der Zeile der \verb!ListView! ausgel�st. Der daf�r
ben�tigten \texttt{onClickListener(\ldots{})} wird im \textsl{ChorePlanAdapter} an die einzelnen \verb!Views! geh�ngt
(Zeile \ref{line:onClickChorePlanAdapter}).
Da es immer nur eine aufgeklapte Zeile geben soll, muss noch �berpr�ft werden ob, es eine offene gibt, die geklickte
die offene oder ob noch keine ge�ffnet Zeile in der Liste war (vgl. Zeile \ref{line:AnimLogic}).
Au�erdem wird im Falle eines Zeilenwechsels die Auswahl der Schritte zur�ckgesetzt, um beim sp�teren Speichern der
�nderungen nicht die falschen Schritte auf 'erledigt' zu setzen, wozu in Zeile \ref{line:resetSel} ein
\textsl{resetSelection()} aufgerufen wird. In Zeile \ref{line:startAnim} wird dann die Animation auf dem entsprechenden
Element ausfgef�hrt.

\begin{lstlisting}[float, caption=ChorePlanAdapter.java, captionpos=below, label=java:ChorePlanAdap, language=java,
escapechar=|]
convertView.setOnClickListener(new View.OnClickListener() { |\label{line:onClickChorePlanAdapter}|

@Override
public void onClick(View v) {
  View toolbar = v.findViewById(R.id.lstViewChoreSteps);

  if (prevToolbar != null   |\label{line:AnimLogic}|
          && prevToolbar.getVisibility() == ListView.VISIBLE
          && toolbar.getVisibility() != ListView.VISIBLE) {
    ExpandAnimation tmpAnimation = new ExpandAnimation(prevToolbar, 0);
    prevToolbar.startAnimation(tmpAnimation);

    TableLayout lvChoreSteps = 
      (TableLayout) prevToolbar.findViewById(R.id.lstViewChoreSteps);
    for(int i= 0; i < lvChoreSteps.getChildCount(); i++) {
        TableRow row = (TableRow) lvChoreSteps.getChildAt(i);
        CheckedTextView chdTxtView =
          (CheckedTextView) row.findViewById(R.id.chkTxtViewStep);

        if (!selectedSteps.isEmpty()
                && selectedSteps.containsValue(
                  steps.get(Integer.parseInt(row.getTag().toString()))
                )) {
            chdTxtView.setChecked(false);
        }
    }

    resetSelection(); |\label{line:resetSel}|
  }

  ExpandAnimation expandAni = new ExpandAnimation(toolbar, 0);
  toolbar.startAnimation(expandAni); |\label{line:startAnim}|
  prevToolbar = toolbar;
}
});
\end{lstlisting}

Das Problem mit dem r�ckg�ngig machen der letzten �nderung �ber ein \verb!Toast! zieht sich durch das gesamte Projekt
und wird hier nun beispielweise erkl�rt.
Nachdem in \ref{java:ChorePlanFrag2} in Zeile \ref{line:lastChorePlanLine} die Daten soweit aufbereitet wurden, dass ein
umsetzen der �nderungen m�glich ist, folgt nun der Einsatz der \emph{UndoBar} \ref{java:UndoChorePlan}. 
Zun�chst wird ein neues Objekt des \texttt{UndoBarController.UndoBar} erstellt. Dieses wird keiner Variablen zugewiesen,
da kein weiterer Zugriff darauf erfolgt. Nacheinander werden der Schaltfl�che nun die Eigenschaften zugewiesen. Zun�chst
das in \ref{java:ChorePlanFrag2} in Zeile \ref{line:lastChorePlanLine} erstellte \verb!Bundle!, in dem die zu �ndernden
Objekte stecken. In Zeile \ref{line:msgUndoBar} wird die anzuzeigende Nachricht gesetzt, in diesem Fall ein aus den
Ressourcen geladener String. Nun folgen die besonderen Eigenschaften, der \verb!Listener! zum Persistieren (Zeile
\ref{line:onHideUndoBar}), beziehungsweise in Zeile \ref{line:onUndoUndoBar} die Undo-Funktion. 
M�chte der Benutzer die �nderung nicht r�ckg�ngig machen, werden in diesem Fall mit der 
\textsl{onHide(\ldots{})}-Methode die ausgew�hlten Schritte �ber das im Kapitel
\ref{para:Messenger} erw�hnte \verb!Messenger!-Interface an den Datenbank-Service �bergeben und synchronisiert.
W�nscht der Benutzer jedoch, die get�tigten �nderunge zu annulieren, kann er das durch einen klicken des
\emph{Undo}-Buttons ansto�en. Dadurch wird die erw�hnte \textsl{onUndo(\ldots{})}-Methode in Zeile
\ref{line:onUndoUndoBar} ausgef�hrt. Darin werden zuerst die lokal ge�nderten Tabelleneintr�ge aus dem Token gelesen
(Zeile \ref{line:readToken}). Darauf folgt die L�schung der Eintr�ge aus der \verb!SQLite!-Datenbank in Zeile
\ref{line:delRows} sowie die Entfernung der Markierung auf der grafischen Oberfl�che (vgl. Zeile \ref{line:setChecked}).
Abschlie�end wird noch durch einen simulierten Klick auf die \verb!ListView!-Zeile das Schlie�en der aufgeklappten
Schritte (Zeile \ref{line:closeToggledLine}) erwirkt.

\begin{lstlisting}[float, caption=ChorePlanFragment.java, captionpos=below, label=java:UndoChorePlan, language=java,
escapechar=|]
[|\ldots{}|]
new UndoBarController.UndoBar(getActivity())
    .token(token)
    .message(getString(R.string.textChoreStepSaved)) |\label{line:msgUndoBar}|
    .listener(new UndoBarController.AdvancedUndoListener() { 
      @Override
      public void onHide(Parcelable _token) { |\label{line:onHideUndoBar}|
        if (_token != null) {
          if (selectedSteps != null && selectedSteps.size() >= 1) {
            for (ChoreStepItem cStepItm : selectedSteps) {
              Message msg = Message.obtain(null,
                MessageConstants.MSG_COMMIT_CHORE_STEP_DONE, 
                cStepItm.getChorePlanId(), cStepItm.getChoreStepId());
              ((MainActivity) getActivity()).sendMessage(msg);
        } } } }
[|\ldots{}|]
      @Override
      public void onUndo(Parcelable _token) { |\label{line:onUndoUndoBar}|
          if (_token != null) {
              ArrayList<Integer> arrayList = |\label{line:readToken}|
                ((Bundle) _token).getIntegerArrayList("insertedRowIds");
              ArrayList<Integer> cPlChIds = 
                ((Bundle) _token).getIntegerArrayList("choreStepId");
              for (int rowId : arrayList) {
                  ChorePlanStep commitStep = new ChorePlanStep(getActivity());
                  commitStep.undoChoreStep(rowId);  |\label{line:delRows}|

                  for (Integer id : cPlChIds) {
                      CheckBox chkBoxChoreDone =
                        ((CheckBox) rootView.findViewWithTag(id));
                      chkBoxChoreDone.setChecked(false); |\label{line:setChecked}|
                      chkBoxChoreDone.setEnabled(false);
                  }
              }
              rootView.findViewWithTag(selectedSteps.get(0).
                getChorePlanId()).performClick(); |\label{line:closeToggledLine}|
    } } }).show();
[|\ldots{}|]
\end{lstlisting}



\section{Einkaufsliste}\label{section:Einkaufsliste}
Die Einkaufsliste ist eine vom Benutzer sortierte Liste von Einkaufsgegenst�nden. Es k�nnen von jedem WG Mitglied Gegenst�nde hinzugef�gt und entfernt werden. Abschlie�end soll dadurch die Abrechnung vereinfacht werden, da den
gekauften Artikel jeweils der K�ufer, als auch der bezahlte Preis zugeordnet werden kann.

\subsection{Implementierung}\label{subSec:EkLImpl}
Wie die Aufgaben wurde auch f�r die Einkaufliste eine Klasse \texttt{ShoppingListFragment} von der Klasse
\verb!Fragment! abgeleitet. Da hierbei bis einschlie�lich zum Einsatz des \texttt{ShoppingListAdapter} analog zum
\texttt{ChorePlanFragment} vorgegangen wurde, ist eine erneute Ausf�hrung nicht notwendig. Im Vergleich dazu wurde
jedoch ein \verb!AutoCompleteTextView! verwendet. Da meist die gleichen Artikel hinzugef�gt werden m�ssen, ist der
Einsatz der Autovervollst�ndigung hier eine gro�e Erleichterung. Dazu werden zun�chst die gew�nschten Vorschl�ge aus der
Datenbank in ein \verb!String-Array! geladen (vgl. Zeile \ref{line:getGroceries}). Danach wird ein \verb!ArrayAdapter!
erzeugt, dem das \verb!Array! mit den Vorschl�gen, sowie ein Layout �bergeben werden. Dieser Adapter wird dem
vorbereiteten \verb!AutoCompleteTextView! auf dem Einkaufslisten-\verb!Fragment! gesetzt. Als weitere Erleichterung
wurde ein \verb!Listener! zur automatischen Eingabebest�tigung implementiert (Zeile \ref{line:onEditorListener}),
welcher beim Hinzuf�gen von Waren zum Einsatz kommt, die noch nicht als Vorschlag verf�gbar hinterlegt sind.
Dazu kommt der \texttt{TextView.OnEditorActionListener()} zum Einsatz und f�ngt das Dr�cken der Tasten auf der Tastatur
ab und �berpr�ft ob es sich dabei um die \verb!Senden!-Taste handelt (siehe Zeile \ref{line:actionIdVergleich}).
In diesem Fall wird die Weitergabe des Events unterbrochen und die \textsl{handleItemAddAction(\ldots{})} aufgerufen
(vgl. Zeile \ref{line:handleItemAdd}, der der eingegebene Text �bergeben wird und diesen als neue Auswahl zur Verf�gung
stellt, sowie ein Dialog �ffnet mit dem die ben�tigte Anzahl f�r die Einkaufsliste �bergeben werden kann.\\

\noindent
\begin{minipage}{\linewidth}
\begin{lstlisting}[caption=ShoppingListFragment.java, captionpos=below, label=java:ShoppingListFrag,
language=java, escapechar=|]
[|\ldots{}|]
final String[] grocieries = dbHandler.getGrocieries(); |\label{line:getGroceries}|
ArrayAdapter<String> adapter = new ArrayAdapter<String>(getActivity(),
    android.R.layout.simple_list_item_1, grocieries);
final AutoCompleteTextView editTextNew =
    (AutoCompleteTextView) rootView.findViewById(R.id.autoCompleteShoppingListNewItem);
editTextNew.setAdapter(adapter);

editTextNew.setOnEditorActionListener(new TextView.OnEditorActionListener() { |\label{line:onEditorListener}|
  @Override
  public boolean onEditorAction(TextView v, int actionId, KeyEvent event) {
    boolean handled = false;
    if (actionId == EditorInfo.IME_ACTION_SEND) { |\label{line:actionIdVergleich}|
      handleItemAddAction(v.getText().toString()); |\label{line:handleItemAdd}|
      handled = true;
    }
    return handled;
} });
[|\ldots{}|]
\end{lstlisting}  
\end{minipage}      

Dieser Teil der App konnte ohne weitere Probleme gel�st werden, weswegen auf die hier �blichen Paragraphen \textsc{Probleme}
und \textsc{L�sungen} verzichtet wird.


\section{Blackboard}\label{section:Blackboard}
Eine einfache M�glichkeit alle WG-Mitglieder zu erreichen bietet ein Blackboard auf dem jeder Nachrichten hinterlassen
kann. Obwohl der Einsatz von Zugriffsbeschr�nkungen auf die Nachrichten leicht umsetzbar w�re,
wurde bewusst darauf verzichtet, um die Eigentschaften eines physikalischen Blackboards gerecht zu werden.

\subsection{Implementierung}\label{subSec:BbImpl}
Die Klassenhierarchie der Fragmente ist aus den vorhergehenden Beispielen schon bekannt und wurde auch in beim
\texttt{BlackboardFragment} beibehalten. Dabei wird in der �berschriebenen Methode \textsl{onCreateView(\ldots{})} durch
ein SQL-Statement die Nachrichten aus der \verb!SQLite!-Datenbank gelesen, welche in einem \verb!Cursor! vorgehalten
werden. Da es sich bei dem Zeilenlayout der \verb!ListView! um kein Standard-Layout handelt, muss zun�chst der
\verb!Cursor! schrittweise durchgearbeitet werden und die Werte in eine \verb!HashMap! �bertragen werden. Auf eine
\verb!HashMap! wurde zur�ckgegriffen, um beim L�schen des Eintrags einfach �ber die \textit{BlackboardId} an die
Nachricht zu gelangen und aus der Liste zu l�schen, das spart einen direkten Datenbankzugriff f�r das L�schen und
weitere Zugriffe beim Aktualisieren der Liste, sowie dem etwaigen Wiederherstellen der Nachricht.
�ber eine \verb!EditText!-Feld kann die neue, mehrzeilige Nachricht eingegeben werden und durch den
\textit{+}-Button dem schwarzen Brett hinzugef�gt werden. Das L�schen der einzelnen Nachrichten kann durch ein Klick auf
das L�schen-Symbol ausgel�st werden und ist durch die \verb!UndoBar!-Funktion revidierbar.
Zum Bearbeiten wurde ein \verb!onLongClickListener! an die \verb!View! der Zeile gebunden. Wird dieser ausgel�st, so
generiert er einen Dialog mit dem aktuellen Inhalt der Nachricht und zeigt diesen an. Gespeichert wird dann die
Nachricht mit der ID des Bearbeiters. Eine Erweiterung um die \verb!UndoBar!-Funktion sollte hier noch erg�nzt werden.

\subsection{Probleme}\label{subSec:BbProbl}
Bei der Umsetzung des Blackboards kam es beim Anzeigen der Nachrichten zu einem komplexen Problem, was zun�chst nicht
nachvollziehbar war. In der \verb!ListView! war zwar die Anzahl der Eintr�ge richtig, aber der Anfang der Liste wurde am 
Ende der Liste, also bei Zeilen die ausserhalb des anf�nglich darstellbaren Bereichs lagen, wiederholt.

\subsection{L�sungen}\label{subSec:BbSol}
Um die Datenquelle als Fehler auszuschlie�en wurde zun�chst ein \verb!DISTINCT! in das \verb!SQL!-Statement eingef�gt.
Das bewirkt, dass doppelte Eintr�ge ausgefiltert werden. Das war aber nicht die Ursache des Problems, da die erwarteten
Daten im \verb!Cursor! waren und auch an den \texttt{BlackboardAdapter} weitergegeben wurden. Somit lies sich das
Problem auf die Anzeige, beziehungsweise das die Vorbereitung der Daten zu Anzeige, eingrenzen. Demnach muss sich der
Fehler im erw�hnten \texttt{BlackboardAdapter} befinden. Nachdem weitere Gedanken �ber die Funktion des Adapters gemacht
wurden, kam die Erkenntnis, dass die Zeile \ref{line:WrongLineBbAdap} im alten Quelltext \ref{java:BbAdapOld} nicht
funktionieren kann, sobald es mehr Eintr�ge gibt, als auf anhieb anzeigbar sind.

\begin{lstlisting}[float, language=java, caption=Alte BlackboardAdapter.java, captionpos=below, label=java:BbAdapOld,
escapechar=|]
[|\ldots{}|]
@Override
public View getView(int position, View convertView, ViewGroup parent) { |\label{line:getView}|
  if (convertView &=&  null) { |\label{line:WrongLineBbAdap}|
    final BlackboardMessage bbMsg =
        (BlackboardMessage) blackboardMessages.values().toArray()[position];
  [|\ldots{}|] }
}
[|\ldots{}|]
\end{lstlisting}

Mit der \verb!if!-Abfrage, ob die �bergebene \verb!View! noch \verb!null! ist, wird verhindert, dass wenn die
\verb!ListView! gescrollt wird, die neue Zeile �berschrieben werden kann. In der \verb!ListView! befinden sich n�mlich
immer gleich viele \verb!Views! als Zeilen, vorrausgesetzt dass mehr Eintr�ge dargestellt werden sollen als auf den
sichtbaren Bereich passen. Tritt der Fall ein dass neue Zeilen angezeigt werden m�ssen, sprich es wird gescrollt, werden
die angezeigten \verb!View!s der \textsl{getView(int position, View convertView, ViewGroup parent)} (vgl. Zeile
\ref{line:getView} im \nameref{java:BbAdapOld}~\ref{java:BbAdapOld}) als \textit{converView} �bergeben. Somit kann diese
Variable nicht \verb!null! sein und die Bedingung der \verb!if!-Abfrage ist falsch. Demzufolge k�nnen die alten 
Daten nicht in den vorhandenen \verb!View!s durch die neuen ersetzt werden und die selbe Nachricht wird noch mal
angezeigt. Die Wiederholung des Anfangs wird dadurch erzeugt, dass die Zeilen-\verb!View!s wiederverwendet werden, die
aus dem angezeigten Bereich geschoben werden, also die zuvor erste \verb!View!.\\*
Durch das Entfernen der \verb!if!-Abfrage wurde die erw�nschte Funktion erreicht und das Scrollen der Liste war
m�glich.

\section{App/Datenbank Schnittstelle}\label{section:AppDbInterface}
In Kapitel \nameref{chp:Prototyp}~\ref{chp:Prototyp}, als auch in Kapitel \nameref{section:App}~\ref{section:App} wurde
bereits die Problematik aufgegriffen, dass es f�r eine Android-App nicht m�glich ist �ber einen \verb!JDBC!-Treiber eine
Verbindung mit einer Datenbank aufzubauen. Neben der Verwendung eines \textsc{REST}full Webservice zum Abrufen von
Datenbankinhalten, gibt es die hier verwendete Methode �ber \textsc{PHP}-Dokumente die �ber eine 
\textsc{HTTP}-Verbindungen abgerufen werden und einen \textsc{JSON}-formatierte Antwort liefern. Diese Vorgehensweise
kann mit einfachen Mitteln realisiert werden und Bedarf keiner komplexen Serverkonfiguration.
Der Aufbau wird ausgehend vom aufgerufenen \textsc{PHP}-Skript, �ber die \textsc{PHP}-Datenbankschnittstelle bis
schlie�lich zur Datenbank hin erkl�rt.

\subsection{PHP-Skript}\label{subSec:PhpDbKomm}
Der Aufruf des Skripts erfolgt �ber eine einfache \textsc{HTTP-POST}-Anfrage an die Server-\textsc{URL}. Der Aufbau
wurde bereits im Kapitel \nameref{subsub:DBConLibProto}~\ref{subsub:DBConLibProto} Paragraph
\ref{para:SyncRemoteDBProto} erl�utert. \\*
Wurden die notwendigen Parameter angegeben, so erreicht die Abarbeitung die Zeile \ref{line:getTag} in der die
gew�nschte Operation dem \textsc{HTTP}-Request entnommen wird. Nach der erfolgreichen Authentifizierung der
�bermittelten Zugangsdaten in Zeile \ref{line:authCreds}, findet die Auswertung der \textit{tag}-Parameters statt. Dabei
gibt es drei m�gliche Zust�nde. Entweder wurde eine Synchronisierung angesto�en, ein Datensatz soll in eine Tabelle
geschrieben werden oder ein �nderung wird committed. Der zweite Zustand, schreiben eines Datensatz, wurde nur f�r den
Prototyp ben�tigt und kann hier vernachl�ssigt werden.
Beim Synchonisieren wird dann aus den Parametern die gew�nschte Tabelle gelesen (Zeile \ref{line:getTablePara}).
Die Parametervariable wird mit der User- und WG-Id der \textsc{PHP}-Datenbankschnittstelle �bergeben,
welche die Daten in einem \verb!Array! zur�ckliefert (siehe Zeile \ref{line:getTable}). Wurde kein leeres \verb!Array!
erhalten, kann es dem \textit{response}-\verb!Array! angeh�ngt werden (vgl. Zeile \ref{line:result}.
Danach wird die Anfrage mit einem \nameref{line:echoJson}~\ref{line:echoJson} an den Client zur�ckgeschickt.
Befindet sich weder \emph{sync} noch \emph{write} im \textit{tag}, so handelt es sich um den Commit einer 
Benutzer�nderungen. In diesem Fall wird der \textit{tag} durch eine \verb!switch-case!-Unterscheidung ausgewertet (Zeile
\ref{line:switchTag}). Da die Datenbankfunktion entweder WG-�bergreifend oder benutzerspezifische �nderungen vornehmen
werden in Zeile \ref{line:userWgId4Commit} die �bertragene WG-, sowie User-ID aus dem \textit{\$user}-Objekt gelesen.
Hier wurden nur die F�lle zum Erstellen und Bearbeiten von Blackboard-Nachrichten dargestellt. Dieser Ausschnitt gen�gt,
um zu erkennen, dass je nach Fall unterschiedliche Parameter aus dem \textsc{HTTP-POST} gelesen werden. Diese werden
dann in einem \verb!Array!, Zeile \ref{line:params1}, \ref{line:params2} und \ref{line:params3}, zusammengepackt und der
\texttt{commitFunc}-Methode der Datenbankschnittstelle zur Ausf�hrung �bergeben (Zeile \ref{line:commitFunc}). Der Grund
f�r die Verwendung des \verb!Arrays! waren die unterschiedlichen Parameter, welche dadurch einfacher einer Funktion �bergeben
werden konnten. Das Ergebnis der Funktion wird wieder als \verb!Array! geliefert und, wie schon beim \textit{write}, dem
R�ckgabeobjekt \textit{\$response} als \textit{result} angeh�ngt (Zeile \ref{line:responseResult}).

\begin{lstlisting}[language=php, float=ph, caption=db.php, captionpos=below, label=php:dbPhpSkript, escapechar=|]
[|\ldots{}|]
$tag = $_POST['tag']; |\label{line:getTag}|
$response = array("tag" => $tag, "success" => 0);

$email = $_POST['email'];
$password = $_POST['password'];
$user = $db->getUserByEmailAndPassword($email, $password); |\label{line:authCreds}|
if ($user) {
 if ($tag == 'sync') {
  $table = $_POST['table']; |\label{line:getTablePara}|
  $result = $db->getTable($table, $user['user_id'], $user['wg_id']); |\label{line:getTable}|
  // put result in array
 if (isset($result) && count($result) > 0) {
      $response["result"] = $result; |\label{line:result}|
      $response["success"]=count($result);
      echo json_encode($response); |\label{line:echoJson}|
 } else {
      $response["success"] = 0;
      $response["error_msg"] = "Got no results back";
 } else if ($tag &=& 'write') {
  [|\ldots{}|] 
 } else { |\label{line:elseToCommit}|
  $userId = $user['user_id']; |\label{line:userWgId4Commit}|
  $wgId = $user['wg_id'];
  $sqlFunc = "";
  $params = "";
  $result = 0;
  switch ($tag) {   |\label{line:switchTag}|
  [|\ldots{}|]
    case "commitBlackboardMessageAdd":
      $bbMsg = $_POST['nachricht'];
      $sqlFunc = "funcBlackboardMessageAdd";
      $params = array($wgId, $userId, $bbMsg); |\label{line:params1}|
    break;
    case "commitBlackboardMessageEdit":
      $newMsg = $_POST['nachricht'];
      $bbId = $_POST['blackboard_id'];
      $sqlFunc = "funcBlackboardMessageEdit";
      $params = array($bbId, $newMsg, $userId); |\label{line:params2}|
    break;
    case "commitBlackboardMessageRemove":
      $blackboardId = $_POST['blackboard_id'];
      $sqlFunc = "funcBlackboardMessageRemove";
      $params = array($blackboardId); |\label{line:params3}|
    break;
  [|\ldots{}|]
    default:
      $result = -1;
  }
  if ($result != -1 && 
    ($result = $db->commitFunc($sqlFunc, $params)) >= 1) { |\label{line:commitFunc}|
    $response["success"] = 1;
    $response["result"] = $result; |\label{line:responseResult}|
    echo json_encode($response);
  } else {
    $response["error_msg"] = "Fehler beim Commit!";
    $response["error"] = $result;
 } }
[|\ldots{}|]
\end{lstlisting}

\subsection{PHP-Datenbankschnittstelle}\label{subSec:PhpDbInterface}
Im wesentlichen ist die \textsc{PHP}-Datenbankschnittstelle ebenso auf \textsc{PHP}-Skripten basierend wie die im
Kapitel \ref{subSec:PhpDbKomm} beschriebene Verarbeitung der \textsc{HTTP}-Anfragen.
Am Anfang der �ffentlichen \textsc{PHP}-Skripte wird die Datenbankschnittstelle intitialisiert (siehe
\nameref{php:dbFuncInit} Zeile \ref{line:reqOnce} f).

\begin{lstlisting}[language=php, float=ph, caption=Datenbankschnittstelle initialisieren, captionpos=below,
label=php:dbFuncInit, escapechar=|]
  require_once 'include/DB_Functions.php'; |\label{line:reqOnce}|
  $db = new DB_Functions();
\end{lstlisting}

Durch das Initialiseren der \texttt{DB\_Functions}-Klasse wird dessen Konstruktor \ref{php:dbFuncConstructor}
aufgerufen. Darin wird dann die Verbindung zur Datenbank aufgebaut, indem ein Objekt der Klasse
\texttt{DB\_Connect} erstellt und darin die Methode \textsl{connect} \ref{php:dbConnConstructor} aufgerufen
wird.

\begin{lstlisting}[language=php, float=ph, caption=Konstruktor der DB\_Functions.php, captionpos=below,
label=php:dbFuncConstructor, escapechar=|]
<?php
class DB_Functions {
  private $db;
  function __construct() {
      require_once 'DB_Connect.php';
      $this->db = new DB_Connect();
      $this->db->connect();
[|\ldots{}|]
\end{lstlisting}

\begin{lstlisting}[language=php, float=ph, caption=Konstruktor der DB\_Connect.php, captionpos=below,
label=php:dbConnConstructor, escapechar=|]
[|\ldots{}|]
public function connect() {
  require_once 'include/config.php';
  $con = mysql_connect(DB_HOST, DB_USER, DB_PASSWORD);
  mysql_select_db(DB_DATABASE);
  return $con;
[|\ldots{}|]
\end{lstlisting}

Sind beide Klassen erfolgreich instanziiert worden, kann mit den Methoden der \texttt{DB\_Functions} gearbeitet werden.
Eine besondere Methode ist die \textsl{getTable(\ldots{})} \ref{php:getTableMethod}. Das Besondere daran ist, dass
dieser Methode sowohl Tabellennamen als auch Prozeduren �bergeben werden k�nnen. Dazu wird, wie in Zeile
\ref{line:strrposTableProc} zu sehen, im �bergebenen Attribute der String \emph{proc} gesucht. Da die Ergebnisse die
selben sind, kann damit eine Redundanz des Codes vermieden werden.

\begin{lstlisting}[language=php, float=tph, caption=getTable() aus DB\_Functions.php, captionpos=below,
label=php:getTableMethod, escapechar=|]
public function getTable($table_name, $userId, $wgId) {
  if (strrpos($table_name, "proc") === false) { |\label{line:strrposTableProc}|
    $query = "SELECT * FROM $table_name;";
  } else {
    $query = "CALL $table_name('$userId','$wgId')";
{|\ldots{}|}
\end{lstlisting}

Eine weitere interessante Methode ist die \textsl{commitFunc(\ldots{})} \ref{php:commitFunc}. Damit kann eine \verb!SQL!-Funktion mit einer
variablen Anzahl an Paramtern aufgerufen werden. Wie in \nameref{php:dbPhpSkript}~\ref{php:dbPhpSkript} in Zeile
\ref{line:commitFunc} zu sehen, wird der Methode ein \verb!Array! mit den passenden Parametern �bergeben, 
sowie der Name der gew�nschten Methode. Zun�chst wird die \verb!SQL-Query! mit einem \verb!SELECT $dbFunc! ge�ffnet (siehe Zeile
\ref{line:comFuncOpenQuery}), wobei f�r \verb!$dbFunc! der �bergebene Funktionsname steht. Danach wird gepr�ft, ob
�berhaupt Parameter im \verb!Array! �bergeben wurden. Da die Parameter mit Kommata aneinandergereiht werden,
wird der erste Parameter vor der Iteration extrahiert (vgl. Zeile \ref{line:comFuncExtract1stParam}), was die
Kommasetzung erleichtert und eine Fallabfrage in der \verb!for!-Schleife, ob es sich um den letzten Parameter handelt,
spart. In der Schleife werden dann einfach die Parameter mit Komma und dem vorhandenen String konkateniert (Zeile
\ref{line:comFuncExtractNthparam}) und mit einer Klammer wird die Abfrage geschlossen (Zeile
\ref{line:comFuncPuttingTogether}). Mit dem Befehl \verb!mysql_query(\ldots{})! wird dann das Statement auf der
Datenbank ausgef�hrt und per \verb!return! zur�ckgegeben (vgl. Zeile \ref{line:comFuncReturn}).

\begin{lstlisting}[language=php, float=ph, caption=commitFunc() aus DB\_Functions.php, captionpos=below,
label=php:commitFunc, escapechar=|]
public function commitFunc($dbFunc, $params) {
  $query = "SELECT $dbFunc ( '"; �|\label{line:comFuncOpenQuery}|

  if (sizeOf($params) >= 1) {
    $paramStr = $params[0]; |\label{line:comFuncExtract1stParam}|
    for ($i = 1; $i < sizeOf($params); $i++ ) {
      $paramStr .= "', '" . $params[$i]; |\label{line:comFuncExtractNthparam}|
    }
    $query .= $paramStr . "' ) "; |\label{line:comFuncPuttingTogether}|
  }
  return mysql_query($query) or die(mysql_error()); |\label{line:comFuncReturn}|
}
\end{lstlisting}

Neben den zwei gerade erl�uterten Methoden besitzt die \texttt{DB\_Functions} noch einige weitere Funktionen, bei denen
von einer detailierten Ausarbeitung abgesehen wird. Einige Funktionen, wie zum Beispiel die Registrierung, Passwort�nderung oder 
Erzeugung eines zuf�lligen Strings, wurden aus dem Programmierbeispiel \textit{\textbf{Android Login and Registration}}� \cite{AndroidHive:12},
als auch \textit{\textbf{Android Programming Samples}} \cite{Learn2Crack:13} �bernommen. 

\subsection{Datenbank}\label{subSec:Datenbank}
Da eine seperate Tabellenstruktur f�r jeden WG sehr umst�ndlich und, bei vielen angemeldeten Wohngemeinschaften, sehr
ressourcenhungrig w�re, werden die Daten in den Tabellen zusammengefasst. Damit m�ssen aber auch die Daten dynamisch an
den gerade anfragenden Benutzer angepasst werden. Desweiteren ist eine sinnvolle Aufarbeitung und Zusammenstellung der
Daten notwendig, um die �bertragene Menge so gering wie m�glich zu halten.

\paragraph{\textit{Views} - Datenbanksichten}\label{para:Sichten}
Da zum einen die �bertragungsgeschwindigkeit bei mobilen Endger�ten meist etwas geringer ausf�llt und zum anderen das
Volumen begrenzt ist, empfiehlt es sich auf das komplette Synchronisieren der Tabellen zu verzichten. Statt die Abfragen
direkt auf den Tabellen auszuf�hren, werden zuerst sogenannte \verb!View!s oder \verb!Sicht!en zwischengeschaltet. Dabei
handelt es sich um gespeicherte Abfragen, die wie Tabellen verwendet werden k�nnen. Da solche \verb!Sicht!en jedoch statisch
sind, werden sie nur zum Zusammenfassen von verschiedenen Tabellen benutzt. Ein einfaches Beispiel einer solchen
\verb!View! ist die \nameref{sql:ViewBlackboard}~\ref{sql:ViewBlackboard}. Darin werden die \textit{user}- und
\textit{blackboards}-Tabellen so vereinigt, dass eine Tabelle entsteht in der zu jeder Nachricht die Benutzer- als auch
die WG-Id angezeigt werden (siehe Zeile \ref{line:ViewBbIds}) und ausgeblendete Nachrichten herausgefiltert sind
(Zeile \ref{line:ViewBbWhere}).

\begin{lstlisting}[language=sql, float=tph, caption=ViewBlackboard, captionpos=below, label=sql:ViewBlackboard,
escapechar=|]
VIEW `ViewBlackboard` AS 
SELECT DISTINCT `b`.`blackboard_id` AS `blackboard_id`,
 `b`.`nachricht` AS `nachricht`,`b`.`crea` AS `crea`,
 `u`.`wg_id` AS `wg_id`,`b`.`creator_id` AS `creator_id` |\label{line:ViewBbIds}|
FROM (`user` `u` JOIN `blackboards` `b` ON((`u`.`wg_id` = `b`.`wg_id`))) 
WHERE (`b`.`anzeigen` > 0); |\label{line:ViewBbWhere}|
\end{lstlisting}

W�rde die App diese \verb!View! abfragen, so m�ssten entweder im \verb!PHP!-Skript oder sp�ter in der App die, f�r die
WG des angemeldeten Benutzers, relevanten Nachrichten herausgefiltert werden. Das w�rde entweder die Laufzeit auf dem
Server drastisch erh�hen oder ein sehr gro�es Sicherheitsrisiko darstellen, da zun�chst die Nachrichten s�mtlicher WGs
�bertragen werden w�rden. Abgesehen vom unn�tig gro�en �betragungsvolumen, wurden hier Prozeduren eingesetzt. 
Wie in \nameref{sql:funcBbMsgRem}~\ref{sql:funcBbMsgRem} zu sehen, wird �hnlich einer \textsc{java}-Methode, mit dem 
Methodenkopf definiert welche Parameter erwartete werden und was sie f�r einen R�ckgabetyp hat (vgl. Zeile
\ref{line:funcBbRemMeth}). Anschlie�end wird mit \textsc{begin} in Zeile \ref{line:funcBbRemBegin} der Anfang des 
Funktionsk�rper markiert, der mit \textsc{RETURN} und \textsc{END} beendet wird (vgl. Zeile \ref{line:funcBbRemEnd}).
Dazwischen kann jeder beliebige, valide \verb!SQL!-Code stehen, in diesem Fall wird beim Blackboardeintrag mit der
\textsl{blackboard\_id} gleich der �bergebenen \texttt{\$blackboardId} die Spalte \textsc{anzeigen} auf 0 gesetzt. Das
f�hrt dazu, dass die Nachricht beim Synchronisieren ausgefiltert wird (siehe Zeile \ref{line:ViewBbWhere} im Listing
\ref{sql:ViewBlackboard}).

\begin{lstlisting}[language=sql, float=ph, caption=Funktion \textsl{funcBlackboardMessageRemove(\ldots{})},
captionpos=below, label=sql:funcBbMsgRem, escapechar=|]
FUNCTION `funcBlackboardMessageRemove`(`$blackboardId` INT(8)) RETURNS tinyint(1) |\label{line:funcBbRemMeth}|
    MODIFIES SQL DATA
BEGIN |\label{line:funcBbRemBegin}|
  UPDATE LOW_PRIORITY blackboards
  SET anzeigen = 0
  WHERE blackboard_id = $blackboardId
  LIMIT 1;
RETURN 1; |\label{line:funcBbRemEnd}|
END
\end{lstlisting}

Das komplette Datenbankschema inklusiv der hier beispielhaft aufgef�hrten \verb!View! und \verb!Funktion! sind im Anhang
einzusehen.

\section{Weboberfl�che}\label{section:WebUI}
Der Administrator einer WG muss in der Lage sein, die Eigenschaften der WG zu konfigurieren und die Funktionen zu verwalten. Wir haben uns dazu entschieden, eine extern jederzeit erreichbare Weboberfl�che daf�r bereitzustellen. Wir h�tten uns genauso gut f�r die Implementierung in die Android App entscheiden k�nnen, haben uns jedoch bewusst dagegen entschieden. Wir m�chten mit den unterschiedlichen Programmiersprachen und den daraus resultierenden Herangehensweisen eine m�glichst gro�e Vielfalt der Informatik widerspiegeln und die Aufgaben fair auf die Teammitglieder verteilen, die bisher noch keine Erfahrung mit der Programmierung f�r die Android Plattform sammeln konnten.\\
Im folgenden Kapitel wird die Struktur der Weboberfl�che mit Hilfe von einzelnen Ausz�gen aus dem Quellcode erl�utert.

\subsection{Umsetzung}\label{subSec:WebUmsetz}
Als Programmiersprache haben wir uns f�r die sehr beliebte und weit verbreitete Skriptsprache PHP entschieden. PHP bietet durch den Einfluss von Java, C++ und Perl einen leichten Einstieg f�r diejenigen, die bereits erste Erfahrungen mit einer der Programmiersprachen sammeln konnten. Ausserdem bietet PHP die einfache Umsetzung von dynamischen Webseiten und eine sehr gute Unterst�tzung von Datenbankverbindungen. Mit PHP ist automatisch sichergestellt, dass die Weboberfl�che von jedem g�ngigen Browser aus in deren Desktop- sowie Mobilversion angezeigt werden kann.

\begin{figure}[H]
Die Weboberfl�che gliedert sich in neun Seiten:
\dirtree{%
.0 /.
.2 admin.php.
.2 benutzer.php.
.2 blackboard.php.
.2 einkaufsliste.php.
.2 login.php.
.2 logout.php.
.2 putzplan.php.
.2 regist.php.
.2 system.php.
}
\end{figure}

Jede dieser PHP Dateien stellt eine Seite der Weboberfl�che dar. Jede Datei beinhaltet gew�hnlichen HTML Code f�r die
Anzeige im Browser und PHP Code f�r die dynamische Datenabfrage von der Datenbank.\\
Mehrzeilige Abfragen in PHP sind in externe Dateien ausgelagert. So ist zum Beispiel die Abfrage zum L�schen eines WG
Mitglieds in der Datei \textit{benutzer\textunderscore edit \textunderscore delete.php} zu finden. Dadurch wird die
�bersichtlichkeit erh�ht und wurde deshalb bei allen weiteren Abfragen so beibehalten.

\begin{figure}[H]
Daraus ergibt sich folgende Struktur f�r die Weboberfl�che mit ihren Seiten inklusive aller ausgelagerten PHP Abfragen:
\dirtree{%
.1 /.
.2 admin.php.
.2 benutzer.php.
.3 benutzer\textunderscore aktivierung.php.
.3 benutzer\textunderscore edit\textunderscore delete.php.
.3 benutzer\textunderscore script.php.
.3 benutzer\textunderscore update.php.
.2 blackboard.php.
.3 blackboard\textunderscore add.php.
.3 blackboard\textunderscore delete\textunderscore edit.php.
.3 blackboard\textunderscore update.php.
.2 einkaufsliste.
.3 einkaufsliste\textunderscore add.php.
.3 einkaufsliste\textunderscore delete\textunderscore edit.php.
.3 waren\textunderscore add\textunderscore delete.php.
.3 waren\textunderscore add.php.
.2 login.
.3 login\textunderscore script.php.
.2 logout.php.
.2 putzplan.php.
.3 putzplan\textunderscore add.php.
.3 putzplan\textunderscore delete\textunderscore edit.php.
.3 putzplan\textunderscore unteraufgaben\textunderscore add.php.
.3 putzplan\textunderscore unteraufgaben\textunderscore update.php.
.3 putzplan\textunderscore update.php.
.2 regist.php.
.3 regist\textunderscore script.php.
.2 system.php.
.3 system\textunderscore delete\textunderscore edit.php.
.3 system\textunderscore update.php.
.2 style.css.
}
\end{figure}

Alle Eingaben die der Benutzer auf der Weboberfl�che machen kann, werden durch Textfelder oder Checkboxen erfasst. Wir
verwenden f�r alle zu �bertragenden Daten die Methode POST. Damit werden die Daten f�r die Benutzer unsichtbar im Rumpf
des HTTP-Requests gesendet. Die Methode GET, die die Daten f�r alle sichtbar an die URI anh�ngen w�rde, k�nnte
theoretisch ebenfalls genutzt werden. Jedoch w�rde das in unserem Fall ein erh�htes Sicherheitsrisiko darstellen. Darum
verzichten wir, bis auf wenige Ausnahmen, auf die GET-Methode.
\\

Auf jeder HTML Seite und in jedem PHP Skript wird zu Beginn �berpr�ft, ob der Nutzer im System angemeldet ist. Falls die Pr�fung fehlschl�gt wird ihm mit dem Hinweis \textit{Bitte loggen Sie sich erst ein!} die Anzeige der Seite verwehrt und das PHP Script wird mit \textit{exit;} gestoppt.

\begin{figure}
\begin{lstlisting}[language=PHP, caption=Login des Benutzers �berpr�fen, captionpos=below, label=WEBLogin]
if(!isset($_SESSION["email"])) {
  echo("<a href=\"login.php\" />Bitte loggen sie sich erst ein!</a>");
  exit;
}
\end{lstlisting}
\end{figure}

Ist der Benutzer eingeloggt, wird das PHP Skript nicht gestoppt, sondern weiter verarbeitet. Mit einem \textit{require\textunderscore once 'db\textunderscore inc.php} wird die Datei db\textunderscore inc.php eingebunden und ausgef�hrt. Das passiert nur, wenn die Datei nicht schon im vorhergehenden Teil des Codes eingebunden wurde.

\begin{figure}
\begin{lstlisting}[language=PHP, caption=Verbindung aufbauen\, falls noch nicht geschehen, captionpos=below,
label=OpenConn]
require_once 'db_inc.php';	
\end{lstlisting}
\end{figure}

Im Anschluss k�nnen die �ber die Methoden POST und GET �bergebenen Variablen abgefragt und zwischengespeichert werden.
In dem Beispiel wird mit einer if-Abfrage gepr�ft, ob eine Variable "einkaufsliste\textunderscore id" an das PHP Skript
�bertragen wurde. Falls dies der Fall ist, wird der Inhalt der Variablen lokal zwischengespeichert.

\begin{figure}
\begin{lstlisting}[language=PHP, caption=Variable \textit{einkaufsliste\textunderscore id} lokal zwischenspeichern,
captionpos=below, label=Variable]
if(isset($_POST['einkaufsliste_id'])) {
	$einkaufsliste_id = $_POST['einkaufsliste_id'];
}
\end{lstlisting}
\end{figure}
 


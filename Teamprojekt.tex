%%%%%%%%%%%%%%%%%%% vorlage.tex %%%%%%%%%%%%%%%%%%%%%%%%%%%%%
%
% LaTeX-Vorlage zur Erstellung von Projekt-Dokumentationen
% im Fachbereich Informatik der Hochschule Trier
%
% Basis: Vorlage svmono des Springer Verlags
%
%%%%%%%%%%%%%%%%%%%%%%%%%%%%%%%%%%%%%%%%%%%%%%%%%%%%%%%%%%%%%

\documentclass[envcountsame,envcountchap, deutsch]{i-studis}

\usepackage{makeidx}         	% Index
\usepackage{multicol}        	% Zweispaltiger Index
\usepackage{dirtree}            % Um Ordnerstruktur nachzubilden
%\usepackage{lineno}             % Referenzieren von Linien
\usepackage{verbatim}
%\usepackage[bottom]{footmisc}	% Erzeugung von Fu�noten
\usepackage{float}


%%-----------------------------------------------------
%\newif\ifpdf
%\ifx\pdfoutput\undefined
%\pdffalse
%\else
%\pdfoutput=1
%\pdftrue
%\fi
%%--------------------------------------------------------
%\ifpdf
\usepackage[pdftex]{graphicx}
\usepackage[pdftex,plainpages=false]{hyperref}
\usepackage{nameref}
%\else
%\usepackage{graphicx}
%\usepackage[plainpages=false]{hyperref}
%\fi

%%-----------------------------------------------------
\usepackage{color}				% Farbverwaltung
%\usepackage{ngerman} 			% Neue deutsche Rechtsschreibung
\usepackage[english, ngerman]{babel}
\usepackage[latin1]{inputenc} 	% Erm�glicht Umlaute-Darstellung
%\usepackage[utf8]{inputenc}  	% Erm�glicht Umlaute-Darstellung unter Linux (je nach verwendetem Format)
\usepackage[T1]{fontenc}

%-----------------------------------------------------
\usepackage{listings} 			% Code-Darstellung
\lstset
{
	basicstyle=\scriptsize, 	% print whole listing small
	keywordstyle=\color{blue}\bfseries,
								% underlined bold black keywords
	identifierstyle=, 			% nothing happens
	commentstyle=\color{red}, 	% white comments
	stringstyle=\ttfamily, 		% typewriter type for strings
	showstringspaces=false, 	% no special string spaces
	framexleftmargin=7mm, 
	tabsize=3,
	showtabs=false,
	frame=single, 
	rulesepcolor=\color{blue},
	numbers=left,
	linewidth=146mm,
	xleftmargin=8mm,
	breaklines=true
}
\usepackage{textcomp} 			% Celsius-Darstellung
\usepackage{amssymb,amsfonts,amstext,amsmath}	% Mathematische Symbole
\usepackage[german, ruled, vlined]{algorithm2e}
\usepackage[a4paper]{geometry} % Andere Formatierung
\usepackage{bibgerm}
\usepackage{array}
\hyphenation{Ele-men-tar-ob-jek-te  ab-ge-tas-tet Aus-wer-tung House-holder-Matrix Le-ast-Squa-res-Al-go-ri-th-men} 		% Weitere Silbentrennung bei Bedarf angeben
\setlength{\textheight}{1.1\textheight}
\pagestyle{myheadings} 			% Erzeugt selbstdefinierte Kopfzeile
\makeindex 						% Index-Erstellung


%--------------------------------------------------------------------------
\begin{document}
%------------------------- Titelblatt -------------------------------------
\title{Entwicklung mobiler Applikation zur zentralen Verwaltung WG-typischer Aufgaben}
\subtitle{Development of an mobile application for central administration to manage flat sharing tasks}
%---- Die Art der Dokumentation kann hier ausgew�hlt werden---------------
\project{Bachelor-Projektarbeit}
%\project{Bachelor-Abschlussarbeit}
%\project{Master-Projektstudium}
%\project{Master-Abschlussarbeit}
%\project{Seminar zur Vorlesung ...}
%\project{Hausarbeit zur Vorlesung ...}
%--------------------------------------------------------------------------
\supervisor{Prof. Dr. Georg Rock} 		% Betreuer der Arbeit
\author{Tobias Barwig, Robert Raschel, Simon Ritzel} 							% Autor der Arbeit
\address{Trier,} 							% Im Zusammenhang mit dem Datum wird hinter dem Ort ein Komma angegeben
\submitdate{18.12.2014} 				% Abgabedatum
%\begingroup
%  \renewcommand{\thepage}{title}
%  \mytitlepage
%  \newpage
%\endgroup
\begingroup
  \renewcommand{\thepage}{Titel}
  \mytitlepage
  \newpage
\endgroup
%--------------------------------------------------------------------------
\frontmatter 
%--------------------------------------------------------------------------
%\input{chapters/Vorwort}				% Vorwort (optional)
\kurzfassung

%% deutsch
\paragraph*{}

Mit der steigenden Anzahl von Studierenden an den Bildungseinrichtungen Deutschlands erfreuen sich die Wohngemeinschaften immer gr��erer Beliebtheit.
Mit den vielen neuen Studenten vergr��ern sich gleichzeitig die Wohngemeinschaften und die Probleme die mit solch zusammengew�rfelten Personengruppen einhergehen. 
Viele Studierende kennen das Problem einer chaotischen Zettelwirtschaft in K�che, Bad und Flur.\\
In dieser Projektarbeit wird die Problematik der Verwaltung von Haushaltsaufgaben in einer Wohngemeinschaft mit meist jungen Erwachsenen behandelt.
Ziel ist das oftmals auftretende Chaos (durch Zettelwirtschaft und schlechter Kommunikation) m�glichst gering zu halten 
und die anstehenden Aufgaben und Termine jederzeit klar definiert jedem Mitbewohner zug�nglich zu machen.
Dieses Ziel soll durch eine mobile Applikation f�r Smartphones auf Basis des von Google entwickelten Betriebssystems Android erreicht werden.


%% englisch
\paragraph*{}
\begin{comment}
This project thesis displays the often problematic administration of the various household tasks within a flatsharing student community.
Objective is to reduce the mostly chaotic atmosphere (be it due to jumble of bits of paper or bad communication) by defining due tasks and deadlines and publishing them to every flatmate in an easy manner.
This should be reached by using a mobile application based in an operation system from Google, called ?Android?. Carrier medium should be a common smartphone.
The application out every roommate into position to manage and create different notes and calender appointments. Furthermore it is possible to manage a shopping list where articles can easily be added, deleted or marked.  In addition to this shows
\end{comment}

Due to increasing numbers of students on German universities, living communities getting more and more important and
popular. This results in larger communities and therefore more problems. Chaotic pin boards and mess all over the place
is well known by every student.
\\
This project deals with the issue of managing household chores in those living communities mainly inhabited by young
adults. Diminishing and containing nuisance by untidy pin boards and poor communication, as well as providing upcoming
chores and tasks for everybody on demand are the main goals. For meeting those goals we decided to develop and implement
an mobile application for Google Android based smartphones.
 			% Kurzfassung Deutsch/English
\tableofcontents 						% Inhaltsverzeichnis
\listoffigures 							% Abbildungsverzeichnis (optional)
%\listoftables 							% Tabellenverzeichnis (optional)
%--------------------------------------------------------------------------
\mainmatter                        		% Hauptteil (ab hier arab. Seitenzahlen)
%--------------------------------------------------------------------------
% Die Kapitel werden in separaten .tex-Dateien abgelegt und hier eingebunden.
\chapter{Einleitung}

\begin{quote}
\textbf{Haushaltsaufgaben m�ssen sinnvoll und einfach auf alle Mitbewohner verteilt werden und der aktuelle Stand zu jedem beliebigen Zeitpunkt f�r jeden einsehbar sein.}
\end{quote}
Die App bietet jedem Mitbewohner die M�glichkeit Notizen einzusehen und zu erstellen. Au�erdem kann eine Einkaufsliste verwaltet werden. In dieser k�nnen Artikel hinzugef�gt, gel�scht und als gekauft markiert werden. Des Weiteren zeigt ein Putzplan die noch anstehenden bzw. bereits erledigten Aufgaben aus dem WG Haushalt an. Jedes WG-Mitglied kann sich in der App einloggen. Dass ein Smartphone als neues Tr�germedium dient, liegt nahe, da nahezu alle der durchweg jungen erwachsenen Personen der Zielgruppe solch ein Ger�t besitzen. Au�erdem bietet es mit der hohen Konnektivit�t die perfekte Grundlage alle Mitbewohner jederzeit auf dem gleichen Wissensstand zu halten. Als Betriebssystem kommt die von Google entwickelte Android Plattform zum Einsatz. Durch deren hohen Marktanteil von 68,2\% wird hiermit die gr��te Anzahl an potentiellen Nutzern erreicht (siehe Abb. \ref{fig:AndroidMarktanteil}).
\clearpage
\begin{figure}[h]
\includegraphics[width=\linewidth]{images/statistic_id170408_marktanteile-der-betriebssysteme-an-der-smartphone-nutzung-in-deutschland-bis-2014}
\caption{Marktanteile der Betriebssysteme an der Smartphone-Nutzung in Deutschland von Dezember 2011 bis Juni 2014}
\label{fig:AndroidMarktanteil}
\end{figure}
\section{Zielsetzung}\index{Zielsetzung}
Einer der WG-Mitbewohner erkl�rt sich dazu bereit die Aufgaben des WG-Adminstrators zu �bernehmen. Dieser WG-Administrator registriert sich als erster und legt dabei eine neue WG f�r sich an. Zu seinen Aufgaben geh�rt unter anderem die Verwaltung der Mitbewohner sowie das Pflegen des Putzplans. Neue Mitbewohner m�ssen vom WG-Administrator per E-Mail in die WG eingeladen werden. F�r den Putzplan definiert er Aufgaben, die in einem einstellbaren Rhythmus wiederholt werden. Zu jeder Aufgabe wird ein Mitbewohner ausgew�hlt. Nun beginnt der Rhythmus zu laufen und die Aufgaben wechseln nach Erledigung automatisch zu der n�chsten Person aus der WG. Auf dem Schwarzen Brett k�nnen Eintr�ge angezeigt, erstellt und gel�scht werden. In der Einkaufsliste k�nnen Artikel hinzugef�gt und entfernt werden. Wurde ein Artikel gekauft, kann derjenige der den Artikel gekauft hat, den Artikel als \textit{gekauft} markieren.
Alle �nderungen eines Mitbewohners sind f�r alle anderen Mitglieder der WG nach einer kurzen Synchronisation sofort sichtbar.
Alle Informationen einer WG werden serverseitig in einer Datenbank und clientseitig auf dem Smartphone des Benutzers gespeichert. Bei jedem Start der App wird ein Datenabgleich der auf Client- und Serverseite gespeicherten Informationen durchgef�hrt und alle voneinander abweichenden Daten auf einer �bersichtsseite dem Benutzer als \textit{Neu} aufgelistet.
\section{�hnliche Apps}\index{�hnlicheApps}
Es gibt bereits eine Auswahl an Apps, die sich jeweils an einem kleinen Teilbereich unseres Funktionsumfanges
orientieren und dies gut umsetzen. Hierbei sind einzelne Apps f�r Einkaufslisten wie 
,,\/Shopping List``\footnote{\label{foot:shoppingList}\href{https://play.google.com/store/apps/details?id=com.fivefly.android.shoppinglist&hl=en}
{,,Shopping List`` in Google Play Store}} oder Putzpl�ne wie ,,Roomboard - Cleaning
Roster``\footnote{\label{foot:roomboard}\href{https://play.google.com/store/apps/details?id=com.roomboardapp.android&hl=en}{,,Roomboard-
Cleaning Roster`` in Google Play Store}}. Jedoch gibt es eine weitere App die sich stark an unserer Idee mit dem
Funktionumfang orientiert. Die App ,,Flatastic: Die
WG-App``\footnote{\label{foot:flatastic}\href{https://play.google.com/store/apps/details?id=com.flatastic.app}{,,Flatastic:
Die WG-App`` in Google Play Store}} bietet neben einer Einkaufsliste, einem Putzplan und einer Pinnwand zus�tzlich einen Ausgabenrechner, womit alle f�r die WG get�tigten Eink�ufe zusammengerechnet werden k�nnen.
Wir beschr�nken uns in dieser Ausarbeitung dennoch weiter auf unseren festgelegten Funktionsumfang und k�nnen uns nach der Fertigstellung nach wie vor dazu entscheiden weitere Zusatzfunktionen zu implementieren.
\inputencoding{latin1}
\chapter{Vorgehensweise}
Um m�gliche Probleme bei der Implementierung der App bereits fr�h zu identifizieren und den Arbeitsumfang der einzelnen Funktionen besser ermitteln zu k�nnen, haben wir vor Beginn der Implementierung die Risiken analysiert und uns f�r ein geeignetes Vorgehensmodell im Projektmanagement entschieden.
\section{Projektmanagement}\index{Projektmanagement}
Horizontales Prototyping mit Einfl�ssen aus Wasserfallmodell (dem Lastenheft)


1. Lastenheft
2. Mockups
3. Prototyp Datenbankverbindung
4. Implementierung



Horizontales Prototyping
Prototyp f�r Datenbankverbindung
Mockups f�r alle wichtigen Screens

Prototyp 
\chapter{Anforderungen}\index{Anforderungen}\label{chp:Anforderungen}
In der Softwareentwicklung erl�utern Anforderungen alle Funktionen eines Softwareprojekts, die ein Auftraggeber vom
fertigen Produkt erwartet. Spezifiziert werden die Anforderungen im Lastenheft, welches direkt vom Auftraggeber an den
Auftragnehmer �bergeben wird.\\
Die Anforderungen werden f�r gew�hnlich in die zwei Kategorien \textit{funktional} und \textit{nichtfunktional} unterteilt. 

\begin{quote}
\textbf{Funktionale Anforderungen} \textit{beschreiben die Merkmale einer Funktion, also was das Produkt tun soll.}\\
\textbf{Nichtfunktionale Anforderungen} \textit{beschreiben die Eigenschaften des Produktes, also wie das Produkt etwas tun soll.}
\end{quote}

Als Beispiel nehmen wir wieder unser bekanntes Blackboard, welches wir bereits detailliert in den Kapiteln \ref{ssec:UseCases} \nameref{ssec:UseCases} sowie \ref{ssec:Mockups} \nameref{ssec:Mockups} kennengelernt haben. Bei unserem Blackboard haben wir die drei funktionalen Anforderungen \textit{Notiz hinzuf�gen}, \textit{Notiz bearbeiten}, \textit{Notiz l�schen} und die nichtfunktionale Anforderung \textit{Design}. Betrachten wir uns in Abb. \ref{fig:LastenheftAnforderungF1} die funktionale Anforderung \textit{Notiz hinzuf�gen}.

\begin{figure}[H]
\centering
\includegraphics[width=\linewidth]{images/vorgehensweise/lastenheft/LastenheftAnforderungenBlackBoard.png}
\caption{Anforderung F1 - Notiz hinzuf�gen (Blackboard)}
\label{fig:LastenheftAnforderungF1}
\end{figure}

Zu Beginn wird die Eigenschaft jeder Anforderung in ein bis zwei kurzen S�tzen erl�utert. Daraufhin folgt eine Tabelle mit charakteristischen Merkmalen der Anforderung. Jeder Anforderung wird eine eindeutige, fortlaufende  ID zugeteilt. Au�erdem wird ein nichttechnischer Titel vergeben. Als Quelle der Anforderung dient ein anderes Dokument der Projektarbeit, im besten Fall bereits verf�gbar und f�r alle abrufbar. Meistens wird ein Dokument aus den Mockups oder Use-Cases als Quelle herangezogen. Die Verweise geben alle weiterf�hrenden Dokumente an, in denen die Anforderung weiter ausgef�hrt wird oder welche, die auf die Zusammenarbeit mit dieser Anforderung unverzichtbar sind. Zuletzt kann eine Priorit�t vergeben werden. Wir haben uns, mit Ausnahmen, bei den meisten unserer funktionalen Anforderungen f�r eine hohe Priorit�t und bei vielen nichtfunktionalen Anforderungen f�r eine mittlere Priorit�t entschieden.\\
Nach der Tabelle folgt eine stichwortartige Beschreibung des grunds�tzlich vorgesehenen Ablaufes. Der Ablauf wird durch uns festgelegt. Hierbei wird festgehalten, wie das System sowie der Benutzer agieren muss und welche Konsequenz die Akteure erwarten haben.

Zuletzt werden alle bekannten Risiken aufgez�hlt und stichwortartig beschrieben.\\

\section{Anforderungen f�r DorMApp}\label{AnforderungenDorMApp}
Alle funktionalen Anforderungen werden als \textit{Anforderung F\#} und alle nichtfunktionalen Anforderungen werden als \textit{Anforderung NF\#} gekennzeichnet.\\
Es folgen nun alle Anforderungen unserer App, sortiert nach den Hauptfunktionen.\\
\begin{itemize}
	\item \textbf{Blackboard}
		\begin{itemize}
			\item \textit{Anforderung F1 - Notiz hinzuf�gen}
			\item \textit{Anforderung F2 - Notiz bearbeiten}
			\item \textit{Anforderung F3 - Notiz l�schen}
			\item \textit{Anforderung NF4 - Design}			
		\end{itemize}
\end{itemize}

\begin{itemize}
	\item \textbf{Putzplan}
		\begin{itemize}
			\item \textit{Anforderung F5 - Aufgabe erledigen}
			\item \textit{Anforderung NF6 - Design}
		\end{itemize}
\end{itemize}

\begin{itemize}
	\item \textbf{Einkaufsliste}
		\begin{itemize}
			\item \textit{Anforderung F7 - Artikel hinzuf�gen}
			\item \textit{Anforderung F8 - Artikel l�schen}
			\item \textit{Anforderung F9 - Artikel gekauft}
			\item \textit{Anforderung NF10 - Design}
		\end{itemize}
\end{itemize}

\begin{itemize}
	\item \textbf{Kalender}
		\begin{itemize}
			\item \textit{Anforderung F11 - Kalendereintrag eintragen}
			\item \textit{Anforderung F12 - Kalendereintrag bearbeiten}
			\item \textit{Anforderung F13 - Kalendereintrag l�schen}
			\item \textit{Anforderung NF14 - Design}
		\end{itemize}
\end{itemize}

\begin{itemize}
	\item \textbf{Web Login}
		\begin{itemize}
			\item \textit{Anforderung F15 - Web Login}
			\item \textit{Anforderung F16 - Eingabekontrolle}
			\item \textit{Anforderung F17 - Passwort vergessen}
			\item \textit{Anforderung NF18 - Design}
		\end{itemize}
\end{itemize}

\begin{itemize}
	\item \textbf{Web Oberfl�che}
		\begin{itemize}
			\item \textit{Anforderung F19 - Benutzerverwaltung}
			\item \textit{Anforderung F20 - Terminkalenderverwaltung}
			\item \textit{Anforderung F21 - Putzplanverwaltung}
			\item \textit{Anforderung F22 - Systemeinstellungen}
			\item \textit{Anforderung NF23 - Design}
		\end{itemize}
\end{itemize}
\newpage

\section{Risiken}\index{Risiken}
Risiken k�nnen den Ablauf in einer Softwarearchitektur erheblich beeinflussen. Jedes Risiko, welche vor der
Implementierung nicht ber�cksichtigt wurde, kann beim Benutzer des Systems ein unkontrollierbares Verhalten verursachen.
Um dies bestm�glich zu vermeiden haben wir uns bereits fr�hzeitig Gedanken um M�gliche Risiken gemacht. In einem
Brainstorming kamen so eine Menge an Risikopunkten zusammen, die zu einem sp�teren Zeitpunkt Probleme machen k�nnten. In
unserem Lastenheft, werden zu jeder Anforderung alle Risiken aufgelistet.
\\
\begin{itemize}
	\item \textbf{Blackboard}
		\begin{itemize}
			\item \textit{Anforderung F1 - Notiz hinzuf�gen}
				\begin{itemize}
					\item Benutzer bricht Vorgang ab
					\item Speichern nicht m�glich
					\item Synchronisieren nicht m�glich
					\\
				\end{itemize}
			\item \textit{Anforderung F2 - Notiz bearbeiten}
				\begin{itemize}
					\item Benutzer bricht Vorgang ab
					\item Speichern nicht m�glich
					\item Synchronisieren nicht m�glich
					\\
				\end{itemize}
			\item \textit{Anforderung F3 - Notiz l�schen}
				\begin{itemize}
					\item Ungewolltes L�schen
					\\
				\end{itemize}
			\item \textit{Anforderung NF4 - Design}			
				\begin{itemize}
					\item Un�bersichtliches Layout
					\item Missverst�ndliche Abl�ufe
					\\
				\end{itemize}
		\end{itemize}
\end{itemize}

\begin{itemize}
	\item \textbf{Putzplan}
		\begin{itemize}
			\item \textit{Anforderung F5 - Aufgabe erledigen}
				\begin{itemize}
					\item Fehlerhafte Synchronisierung
					\item Falsche Aufgabe ausgew�hlt
					\\
				\end{itemize}
			\item \textit{Anforderung NF6 - Design}
				\begin{itemize}
					\item Namen des Verantwortlichen zu Lang und/oder in nicht darstellbarem Zeichensatz
					\item Rhythmus funktioniert nicht wie Benutzer es erwartet
					\\
				\end{itemize}
		\end{itemize}
\end{itemize}

\begin{itemize}
	\item \textbf{Einkaufsliste}
		\begin{itemize}
			\item \textit{Anforderung F7 - Artikel hinzuf�gen}
				\begin{itemize}
					\item Fehler beim Speichern in die Datenbank
					\item Der Artikel steht bereits in der Einkaufsliste, allerdings mit anderer Buchstabenformatierung (Gro�- und Kleinschreibung, Bindestrich, etc.)
					\\
				\end{itemize}
			\item \textit{Anforderung F8 - Artikel l�schen}
				\begin{itemize}
					\item Fehler beim Speichern in die Datenbank
					\item Der Artikel existiert bereits nicht mehr in der Datenbank, wird aber in der Applikation auf dem Endger�t noch angezeigt.
					\\
				\end{itemize}
			\item \textit{Anforderung F9 - Artikel gekauft}
				\begin{itemize}
					\item Fehler beim speichern in die Datenbank
					\item Der Artikel existiert bereits nicht mehr in der Datenbank, wird aber in der Applikation auf dem Endger�t noch angezeigt.
					\\
				\end{itemize}
			\item \textit{Anforderung NF10 - Design}
				\begin{itemize}
					\item Sortierung der Artikel
					\item Un�bersichtliche Aufz�hlung
					\item Schlecht leserliche Schlichtart und Schriftgr��e
					\\
				\end{itemize}
		\end{itemize}
\end{itemize}

\begin{itemize}
	\item \textbf{Kalender}
		\begin{itemize}
			\item \textit{Anforderung F11 - Kalendereintrag eintragen}
				\begin{itemize}
					\item Fehler beim Speichern in die Datenbank
					\item Titel ist zu lang
					\item Datum liegt in der Vergangenheit
					\item Datum liegt weit in der Zukunft (>50 Jahre)
					\\
				\end{itemize}
			\item \textit{Anforderung F12 - Kalendereintrag bearbeiten}
				\begin{itemize}
					\item Fehler beim Speichern in die Datenbank
					\item Titel ist zu lang
					\item Datum liegt in der Vergangenheit
					\item Datum liegt weit in der Zukunft (>50 Jahre)
					\\
				\end{itemize}
			\item \textit{Anforderung F13 - Kalendereintrag l�schen}
				\begin{itemize}
					\item Fehler beim speichern in die Datenbank
					\item Kalendereintrag existiert nicht mehr in Datenbank, wird aber noch auf dem Endger�t angezeigt
					\\
				\end{itemize}
			\item \textit{Anforderung NF14 - Design}
				\begin{itemize}
					\item Benutzer ben�tigt zu lange um sich mit dem Design zurecht zu finden
					\item Zu wenig Platz um alle Kalendereintragungen anzuzeigen
					\item Titel zu lang f�r Gesamtansicht
					\\
				\end{itemize}
		\end{itemize}
\end{itemize}

\begin{itemize}
	\item \textbf{Web Login}
		\begin{itemize}
			\item \textit{Anforderung F15 - Web Login}
				\begin{itemize}
					\item E-Mailadresse existiert nicht
					\item E-Mailadresse und/oder Passwort falsch
					\item Passwort vergessen
					\item Verbindung zum Server kann nicht hergestellt werden
					\\
				\end{itemize}
			\item \textit{Anforderung F16 - Eingabekontrolle}
				\begin{itemize}
					\item\textit{ Eingabe falscher Daten}
					\\
				\end{itemize}
			\item \textit{Anforderung F17 - Passwort vergessen}
				\begin{itemize}
					\item Missbrauch von Daten
					\\
				\end{itemize}
			\item \textit{Anforderung NF18 - Design}
				\begin{itemize}
					\item Design Gesetze nicht eingehalten (Gesetz der N�he, Gleichheit, Geschlossenheit, etc.)
					\item Gr��e der Textfelder nicht optimal
					\\
				\end{itemize}
		\end{itemize}
\end{itemize}


\begin{itemize}
	\item \textbf{Web Oberfl�che}
		\begin{itemize}
			\item \textit{Anforderung F19 - Benutzerverwaltung}
				\begin{itemize}
					\item Datenbankverbindung schl�gt fehl
					\\
				\end{itemize}
			\item \textit{Anforderung F20 - Terminkalenderverwaltung}
				\begin{itemize}
					\item Datenbankverbindung
					\\
				\end{itemize}
			\item \textit{Anforderung F21 - Putzplanverwaltung}
				\begin{itemize}
					\item Datenbankverbindung
					\\
				\end{itemize}
			\item \textit{Anforderung F22 - Systemeinstellungen}
				\begin{itemize}
					\item Datenbankverbindung
					\\
				\end{itemize}
			\item \textit{Anforderung NF23 - Design}
				\begin{itemize}
					\item Un�bersichtlich
					\\
				\end{itemize}
		\end{itemize}
\end{itemize}
Die mit Abstand anf�lligste Funktion unserer App wird die Verbindung zur Datenbank auf einem externen Server sein. Dabei kann zu jedem Zeitpunkt ein Fehler auftreten, der vom System m�glichst gut behandelt werden soll. Darum hat die Verbindung zur Datenbank mehr Aufmerksamkeit bekommen und wir haben uns daf�r entschieden einen lauff�higen Prototypen zu entwickeln. Wir haben uns bereits jetzt so stark mit der Mechanik besch�ftigt, um anstelle eines einfachen Wegwerf-Prototypen, einen im sp�teren Verlauf der Implementierung weiter zu verwendenden Prototypen entwerfen k�nnen.
Folgende Risiken k�nnen bei unserer Arbeit mit einer extern erreichbaren Datenbank auftreten:
\begin{itemize}
	\item Server nicht erreichbar
	\item Verbindungsabbruch durch Zusammenbruch der Internetverbindung
	\item Benutzer killt die App w�hrend der aktiven Nutzung der Verbindung
\end{itemize}
Eine genaue Erl�uterung der Implementierung des gesamten Prototypen finden Sie im folgenden Kapitel \ref{chp:Prototyp} \nameref{chp:Prototyp}.

\chapter{Prototyp}\label{chp:Prototyp}
Um vermeidbare Verz�gerungen in der sp�teren Entwicklung m�glichst auszuschlie�en, wurde der Einsatz von Prototypen entschieden. 
Dazu werden die ben�tigten Komponenten genauer betrachtet und auf ihre Umsetzbarkeit hin untersucht. 
Bei einer Android-App, die auf Inhalte aus einer Datenbank zugreift, 
ist das die Schnittstelle zur Datenbank, beziehungsweise die Netzwerkverbindung.\\
Da eine direkte Verbindung zum Datenbankserver aus dem Android Betriebssystem nicht m�glich ist,
wurde die Authentifizierung des Benutzers als weitere Problemstelle markiert.
Den die Verwendung von Datenbankbenutzern f�llt dadurch weg und muss andersweitig gel�st werden.
Um bei diesen kritischen Stellen nicht in bredouille zu geraten, wurde daf�r ein Prototyp entwicklt.

\section{Implementierung}\label{sect:ImplPrototyp}
Als eine einfache und schnelle Art der Umsetzung haben wir uns f�r die PHP-Variante entschieden.
Dabei werden die Daten von einem PHP-Skript aus der Datenbank gelesen und in eine JSON-Datenformat gebracht.
Dieses Objekt wird in einem HTTP-Paket an die App �bertragen.
In der App sorgt dann ein JSON-Parser f�r das Auslesen der Daten, welche anschlie�end direkt verwertet werden oder
zun�chst in der lokalen SQLite-Datenbank vorgehalten werden. \\
Alternativ zu dieser L�sung w�re ein RESTful Web Service gewesen. 
Dieser L�sungsansatz w�re jedoch mit einem gr��eren programmiertechnischen Aufwand, sowie einer umfangreicheren
Serverkonfiguration verbunden gewesen. \\
Da beide Schwerpunkte in einem Prototyp getestet wurden, wird auf eine weitere Trennung verzichtet.
\pagebreak

\subsection{Datenbankstruktur}\label{subsect:DBStruktPrototyp}
Begonnen wurde die Umsetzung mit der Definition der Datenbankstruktur sowie deren Umsetzung.
Dazu wurden zwei einfache Tabellen angelegt. Die Tabelle \verb!tp_test!\ref{sql:tptest} wird f�r die Schreib- und Lesevorg�nge
verwendet. Um alle n�tigen Datentypen testen zu k�nnen wurden verschiedene Spalten verwendet. 
Dadurch konnte auch der sp�tere Einsatz besser simuliert werden.

\begin{figure} %[hbtp]
  \centering
  \begin{minipage}{1.0\textwidth} \small
    \begin{lstlisting}[language=sql]
      CREATE TABLE IF NOT EXISTS `test_tp` (
          `uid` VARCHAR(23) NOT NULL,
          `msg` TEXT,
          `nmbr` INT(11) DEFAULT NULL,
          `created_at` TIMESTAMP NOT NULL DEFAULT CURRENT_TIMESTAMP
      ) ENGINE=MyISAM DEFAULT CHARSET=utf8;
    \end{lstlisting}
  \end{minipage}
  \caption{Aufbau der Tabelle tp\_test} 
  \label{sql:tptest}
\end{figure}

Um die Benutzerverwaltung f�r den Prototypen zu simulieren wurden au�erdem noch eine Tabelle \verb!users!
\ref{sql:usersTest} angelegt.
Darin wurden Informationen zu den Benutzern hinterlegt. Zum Beispiel eine eindeutige ID, sowie Vor- und Nachname.
Zum Authentifizieren wurde die E-Mailadresse und ein beliebiges Passwort verwendet.
Um das Passwort nicht im Klartext zu speichern, wurde es zusammen mit einem Salt als Hash-Wert abgelegt.

\begin{figure} %[hbtp]
  \centering
    \begin{minipage}{1.0\textwidth} \small
      \begin{lstlisting}[language=sql]
        CREATE TABLE IF NOT EXISTS `users` (
          `uid` int(11) NOT NULL AUTO_INCREMENT,
          `unique_id` varchar(23) NOT NULL,
          `firstname` varchar(50) NOT NULL,
          `lastname` varchar(50) NOT NULL,
          `username` varchar(20) NOT NULL,
          `wgId` int(11) NOT NULL,
          `email` varchar(100) NOT NULL,
          `encrypted_password` varchar(80) NOT NULL,
          `salt` varchar(10) NOT NULL,
          `created_at` datetime DEFAULT NULL,
          PRIMARY KEY (`uid`)
        ) ENGINE=MyISAM  DEFAULT CHARSET=utf8 AUTO_INCREMENT=5 ;
        \end{lstlisting}
    \end{minipage}
  \caption{Aufbau der Tabelle users}
  \label{sql:usersTest}
\end{figure}

  %Nachdem die Struktur der Datenbank stand, wurde die \kill

\subsection{PHP-Skripte}\label{subsect:PHPrototyp}
Der Aufbau der PHP-Schnittstelle ist simpel umgesetzt, da nicht viele Funktionen f�r den Prototyp ben�tigt werden.
Trotzdem wurde auf eine �bersichtliche Datei- und Ordnerstruktur, als auch auf einen modularen Aufbau geachtet.
Wie in Abbildung \ref{dirtree:phpPrototyp} \nameref{dirtree:phpPrototyp} zu sehen, wurden der Aufbau zun�chst in zwei Kategorien aufgeteilt.
Funktionen, die direkt auf der Datenbank ausgef�hrt werden, sowie das Verbinden und Bereitstellen des Datenbankobjekts
�bernehmen, sind im Ordner \verb!php/include! untergebracht.
Alle weiteren Funktionen die aus der App heraus erreichbar sein sollen, befinden sich im Hauptordner \verb!php!.

\begin{figure} %[hbtp]
  \centering
  \begin{minipage}{1.0\textwidth} \small
    \dirtree{%
      .1 php.
        .2 include.
          .3 config.php \ldots{} \begin{minipage}[t]{5cm}
                                Verbindungsparameter f�r die Datenbank{.}
                               \end{minipage}.
          .3 DB\_Connect.php \ldots{}
                               \begin{minipage}[t]{5cm}
                                 Funktionen zum Aufbau und dem Bereitstellen der Datenbankverbindung{.}
                               \end{minipage}.
          .3 DB\_Functions.php \ldots{} \begin{minipage}[t]{5cm}
                                  Methoden f�r Datenbankoperationen; zum Beispiel getTable{.}
                                \end{minipage}.
        .2 db.php \ldots{} \begin{minipage}[t]{5cm}
                                  Verarbeitung des HTTP-Requests, auswerten der Anfrage, sammeln der Daten und
                                  abschlie�end Daten verpacken und an Client zur�cksenden{.}
                                \end{minipage}.
        .2 login.php \ldots{} \begin{minipage}[t]{5cm}
                                  Authentifizieren des Benutzers{.}
                                \end{minipage}.
    }
  \end{minipage}
  \caption{Struktur der PHP-Skripte im Dateisystem}
  \label{dirtree:phpPrototyp}
\end{figure}

Auf die einzelnen Skripte, sowie die Funktionen der Methoden, wird im Kapitel \ref{section:App} noch
detailiert eingegangen, da diese nahezu identisch �bernommen wurden.

\subsection{Prototyp-App}\label{subsect:PrototypApp}


\chapter{Implementierung}\label{chp:Impl}
Die App wurde zum Gro�teil in der aktuell von Google empfohlenen Umgebung Android Studio umgesetzt. Zu Beginn fand die
Entwicklung noch in Eclipse mit dem entsprechenden Plug-In statt. Jedoch erschwerten die Bugs und die Beh�bigkeit der
Eclipse-IDE den z�gigen Fortschritt. Aus diesem Grund wurde nach dem Legen des Grundsteins das Projekt auf die neue
Entwicklungsumgebung migriert, wo es auch fertiggestellt wurde.
\ldots

\section{App}\label{section:App}
Dieses Kapitel umfasst die Implementierung der App, auf der die weiteren Teile aufbauen. Dabei wird n�her auf die
service-�hnliche Struktur und deren Umsetzung, sowie die sichere Speicherung von Benutzerinformationen eingegangen. 
Anfangs wird der Ablauf beim ersten Start, den folgenden Starts sowohl mit als auch ohne angemeldetem
Benutzer beschrieben.
\\*
Da eine Verwendung der App ohne Anmeldung, sprich ohne personalisierte Daten, nicht vorgesehen ist, wird zun�chst der
App-Speicher auf vorhandene Login-Informationen �berpr�ft. Aufgrund der Annahme, dass es sich um die erste Verwendung
nach der Installation handelt, k�nnen noch keine Benutzerdaten vorhanden sein und es wird direkt in die
\textsl{LoginActivity} weitergeleitet, die ohne Anmeldung nicht verlassen werden kann. Hat sich der Benutzer erfolgreich
authentifiziert, startet die Synchronisierung aller Tabellen, um die aktuellsten Daten zu erhalten. Ein Beenden ohne
Logout hat zur Folge, dass E-Mailadresse und Passwort �ber die \texttt{Secure-Preferences}-Schnittstelle verschl�sselt
im Speicherbereich abgelegt werden. Dies erfolgt in der \textsl{onPause()}-Methode, um den Verlust der Daten beim
Zerst�ren der App aufgrund von Ressourcenknappheit zu verhindern. Beim darauffolgenden Start wird in der
\textsl{onResume()}-Methode gepr�ft, ob verschl�sselte Credentials im Speicher hinterlegt sind. Nach dem Auslesen werden
damit die aktuellen Bewegungsdaten aus der Datenbank abgerufen und lokal gespeichert. Hat sich der Benutzer vor
dem Beenden der App abgemeldet, wurden beim Schlie�en keine Informationen im App-Speicher abgelegt. Dadurch landet der
Benutzer, wie beim Erststart, wieder in der Loginmaske.

\subsection{Probleme}\label{subsec:AppProbl}
Neben denen in Kapitel \ref{chp:Prototyp} - \nameref{chp:Prototyp} gel�sten Problemen, die schon vor dem Beginn der Implementierung erkannt 
wurden, sind auch w�hrend der Umsetzung einige Stolperfallen aufgetreten. Dazu geh�rt zun�chst die Herausforderung, die
Logininformationen so sicher wie m�glich zu speichern. Des Weiteren sollte die Server- und Datenbankkommunikation
m�glichst unabh�ngig vom Darstellungsteil der App gehalten werden.

\paragraph*{Unsichere App-Speicher}\label{para:unsafeStorage}
Da der \verb!SharedPreference!-Speicher nicht verschl�sselt ist und durch einfache Mittel ausgelesen werden kann, d�rfen
dort keine sensiblen Daten ohne weiteres gespeichert werden. Die sicherste Art w�re es, die Logininformationen nicht zu
speichern, was aber dazu f�hren w�rde, dass sich der Benutzer bei jedem Start der App neu anmelden m�sste. Das ist dem
Benutzer aber unter keinerlei Umst�nden zumutbar.

\paragraph*{Datenbank-Service}\label{para:dbService}
Das Trennen der Oberfl�che von der Datenhaltung hat mehrere Vorteile und wurde deswegen in diesem Projekt
umgesetzt. Zun�chst ist es dadurch m�glich die Kommunikation zwischen App und Datenbank zu �ndern, ohne dass die
Oberfl�che angepasst werden muss. Weiterhin ist es dadurch einfacher die Umsetzung auf verschiedene Entwickler
aufzuteilen, da keine st�ndige R�cksprache n�tig ist, sondern nur zu Beginn die Schnittstelle bereits definiert sein muss.

\subsection{L�sungen}\label{subsec:AppSol}
Zur L�sung der vorangegangenen Probleme, sind nachfolgend kurze Ausz�ge aus dem Quelltext mit einer knappen 
Erl�uterung der Funktion.

\paragraph*{Secure-Preferences}\label{para:secPrefs}
Wie zuvor erw�hnt, ist der App-Speicher nicht verschl�sselt und somit eigentlich f�r die Ablage von Passw�rtern
ungeeignet. Die L�sung dieses Problems bringt der Einsatz einer Verschl�sselung beim Schreiben der \verb!SharedPreferences!.
Dabei kommt die schon erw�hnte Bibliothek \emph{Secure-Preferences} ins Spiel, wodurch die Informationen vor dem
Schreiben in den App-Speicher verschl�sselt werden. Dabei wird einfach die schon vorhandene \verb!SharedPreferences!-Funktion
von Android mit einer extra Schnittstelle dazwischen verwendet, die beim Schreiben ver- und beim Lesen entschl�sselt.
Die Verwendung der \emph{Secure-Preferences} ist einfach und analog zur Verwendung der
Standard-\verb!SharedPreferences!. Das physikalische erstellen der Datei auf dem Datentr�ger geschieht durch das
Instanziieren der Klasse \texttt{SecurePreferences} im \verb!Context! der App. Wie auch bei den \verb!SharedPreferences!
kann auch hier optional ein eigener Name f�r die Datei �bergeben werden, ist hier jedoch nicht notwendig. Hat sich ein
Benutzer angemeldet und wird die App geschlossen, wird die Funktion \textsl{storeCredentials(\ldots{})}
(Zeile \ref{line:secPrefsStore}) aufgerufen und die Referenz zum \emph{SecurePreferences}-Objekt sowie die
\texttt{AuthCredentials} �bergeben. Damit �berhaupt Daten geschrieben werden k�nnen, muss zun�chst ein \texttt{Editor}-Objekt
erstellt werden, was durch \verb!.edit()! in Zeile \ref{line:secPrefsEdit} geschieht. Damit einen eventuell verwaister Eintrag
keine Probleme bereitet, wird der Speicher in der folgenden Zeile sicherheitshalber komplett gel�scht. Danach werden die
Attribute aus dem \texttt{AuthCredentials}-Objekt ausgelesen und zusammen mit einem eindeutigen Bezeichner durch den
\verb!.put(\ldots{})!-Befehl dem \verb!Editor! zum Speichern �bergeben (vgl. Zeile \ref{line:secPrefsPut} ff). Um die
Daten nun physikalisch zu schreiben, wird auf dem \verb!Editor! \verb!.commit()! (vgl. Zeile
\ref{line:secPrefsCommit})
ausgef�hrt. Ausgelesen werden die Daten einfach in der umgekehrten Reihenfolge, mit dem einzigen Unterschied, dass
hierzu kein \verb!Editor! ben�tigt wird. Wird zum Beispiel die App gestartet, ruft die
\textsl{onResume()}-Methode die Methode \textsl{loggedInUser(\ldots{})} in Zeile \ref{line:secPrefsLogged} auf.
Darin wird nach dem Sicherstellen ob Zugangsdaten vorhanden sind, die einzelnen Schl�ssel-Wert-Paare wieder ausgelesen
(siehe Zeile \ref{line:secPrefsGet} ff). Ist keiner der Werte \verb!null!, werden sie in einem \texttt{AuthCredentials}
verpackt zur�ckgegeben.

\begin{figure}[H]
  \centering
  
  \begin{lstlisting}[language=java,escapechar=|]
[|\ldots{}|]
public void resetCredentials(final SecurePreferences secPrefs) {
  Editor secPrefEditor = secPrefs.edit();
  secPrefEditor.clear();
  secPrefEditor.commit();
}
public static void storeCredentials(final SecurePreferences secPrefs,
  AuthCredentials _creds) { |\label{line:secPrefsStore}|
  Editor secPrefEditor = secPrefs.edit(); |\label{line:secPrefsEdit}|
  secPrefEditor.clear();
  secPrefEditor.putString(EnumSqLite.KEY_UID.getName(), 
      _creds.getUid()); |\label{line:secPrefsPut}|
  secPrefEditor.putString(EnumSqLite.KEY_PASSWORD.getName(),
      _creds.getPassword());
  secPrefEditor.putString(EnumSqLite.KEY_EMAIL.getName(), 
      _creds.getEmail());
  secPrefEditor.commit(); |\label{line:secPrefsCommit}|
}
public static AuthCredentials loggedInUser(final SecurePreferences secPrefs) { |\label{line:secPrefsLogged}|
  String uid = null, uname = null, upassword = null, email = null;
  if (!secPrefs.getAll().isEmpty()) {
    uid = secPrefs.getString(EnumSqLite.KEY_UID.getName(), null); |\label{line:secPrefsGet}|
    upassword = secPrefs.getString(EnumSqLite.KEY_PASSWORD.getName(), null);
    email = secPrefs.getString(EnumSqLite.KEY_EMAIL.getName(), null);
  }
  if (uid != null & upassword != null & email != null ) {
    AuthCredentials creds = new AuthCredentials(uid, email, upassword); |\label{line:secPrefsAuth}|
    return creds;
  }
  return null;
}
[|\ldots{}|]
  \end{lstlisting}
  \caption{Verwendung von Secure-Preferences}
  \label{java:UsingSecPrefs}
\end{figure}

Hat man schon mit den \verb!SharedPreferences! gearbeitet, kann man klar die analoge Vorgehenseweise erkennen.
Obwohl die Entwickler keine hunderprozentige Sicherheit garantieren k�nnen und m�chten, ist es dennoch dem
unverschl�sselten Ablegen vorzuziehen.

\paragraph{Messenger-Klasse}\label{para:Messenger}
Wie schon erw�hnt, sollte eine Trennung von Oberfl�chen- und Datenlogik erstrebt werden.
Um diese Trennung zu erreichen, wurde die \texttt{Messenger}-Klasse verwendet \cite{MessengerClass:14}. 
Mit dieser Klasse ist die Implementierung eines \verb!gebundenen Service! einfacher als eine mit einer
\verb!AIDL!-Schnittstelle, erf�llt aber alle Vorraussetzungen die f�r dieses Programm witchtig sind. 
\\*
Der Aufbau der \emph{Messenger}-Schnittstelle ist �bersichtlich und mit wenigen Schritten erreicht. Zun�chst wird eine
\texttt{MessengerService}-Klasse erstellt, die von der \texttt{Service}-Klasse erbt. Somit muss die Methode 
\textsl{onBind(\ldots{})} (vgl. Zeile \ref{line:MsgBinder}) implementiert werden, welche als \verb!Binder!
eine Instanz der inneren Klasse \texttt{IncomingHandler} zur�ckgibt (vgl. Zeile \ref{line:newMessngr}).
Der \texttt{IncomingHandler} arbeitet die eingehenden Anfragen 
seriell ab und f�hrt mittels einer \verb!switch-case!-Anweisung die erw�nschten
Operationen aus (wie in Zeile \ref{line:handleMsg} ff zu sehen).

\begin{figure}[H]
  \centering
  \begin{lstlisting}[language=java,escapechar=|]
[|\ldots{}|]
  private final Messenger mMessenger = new Messenger(new IncomingHandler()); |\label{line:newMessngr}|
[|\ldots{}|]
@Override
public IBinder onBind(Intent intent) { |\label{line:MsgBinder}|
  return mMessenger.getBinder();
}
public class IncomingHandler extends Handler {

  public static final String TAG = Constants.TAG_PREFIX + "IncomingHandler";

  @Override
  public void handleMessage(Message msg) { |\label{line:handleMsg}|
    // TODO
    String[] tablesToSync;
    int ppAufgId;
    Bundle _bundle;
    Map<String, String> params;
    int remItemId, shoppingListId;
    switch (msg.what) {
        case MessageConstants.MSG_UNREBIND:
            reService = null;
            reBound = false;
            break;
      {|\ldots{}|}
     }
     {|\ldots{}|}
   }
   {|\ldots{}|}
}
[|\ldots{}|]
  \end{lstlisting}
  \caption{Auszug aus MessengerService.java}
  \label{java:MessengerService}
\end{figure}

Neben den erw�hnten Methoden enth�lt die Klasse \texttt{MessengerService} au�erdem noch einige Hilfsmethoden, die zum
Beispiel zum Entpacken der \verb!Bundles! verwendet werden.

\section{Putzplan}\label{section:Putzplan}

\subsection{Umsetzung}\label{subSec:PPUmsetz}

\subsection{Probleme und L�sungen}\label{subSec:PPSol}



\section{Einkaufsliste}\label{section:Einkaufsliste}
Die Einkaufsliste ist eine vom Benutzer sortierte Liste von Einkaufsgegenst�nden. Es k�nnen von jedem WG Mitglied Gegenst�nde hinzugef�gt und entfernt werden. Abschlie�end soll dadurch die Abrechnung vereinfacht werden, da den
gekauften Artikel jeweils der K�ufer, als auch der bezahlte Preis zugeordnet werden kann.

\subsection{Implementierung}\label{subSec:EkLImpl}
Wie die Aufgaben wurde auch f�r die Einkaufliste eine Klasse \texttt{ShoppingListFragment} von der Klasse
\verb!Fragment! abgeleitet. Da hierbei bis einschlie�lich zum Einsatz des \texttt{ShoppingListAdapter} analog zum
\texttt{ChorePlanFragment} vorgegangen wurde, ist eine erneute Ausf�hrung nicht notwendig. Im Vergleich dazu wurde
jedoch ein \verb!AutoCompleteTextView! verwendet. Da meist die gleichen Artikel hinzugef�gt werden m�ssen, ist der
Einsatz der Autovervollst�ndigung hier eine gro�e Erleichterung. Dazu werden zun�chst die gew�nschten Vorschl�ge aus der
Datenbank in ein \verb!String-Array! geladen (vgl. Zeile \ref{line:getGroceries}). Danach wird ein \verb!ArrayAdapter!
erzeugt, dem das \verb!Array! mit den Vorschl�gen, sowie ein Layout �bergeben werden. Dieser Adapter wird dem
vorbereiteten \verb!AutoCompleteTextView! auf dem Einkaufslisten-\verb!Fragment! gesetzt. Als weitere Erleichterung
wurde ein \verb!Listener! zur automatischen Eingabebest�tigung implementiert (Zeile \ref{line:onEditorListener}),
welcher beim Hinzuf�gen von Waren zum Einsatz kommt, die noch nicht als Vorschlag verf�gbar hinterlegt sind.
Dazu kommt der \texttt{TextView.OnEditorActionListener()} zum Einsatz und f�ngt das Dr�cken der Tasten auf der Tastatur
ab und �berpr�ft ob es sich dabei um die \verb!Senden!-Taste handelt (siehe Zeile \ref{line:actionIdVergleich}).
In diesem Fall wird die Weitergabe des Events unterbrochen und die \textsl{handleItemAddAction(\ldots{})} aufgerufen
(vgl. Zeile \ref{line:handleItemAdd}, der der eingegebene Text �bergeben wird und diesen als neue Auswahl zur Verf�gung
stellt, sowie ein Dialog �ffnet mit dem die ben�tigte Anzahl f�r die Einkaufsliste �bergeben werden kann.\\

\noindent
\begin{minipage}{\linewidth}
\begin{lstlisting}[caption=ShoppingListFragment.java, captionpos=below, label=java:ShoppingListFrag,
language=java, escapechar=|]
[|\ldots{}|]
final String[] grocieries = dbHandler.getGrocieries(); |\label{line:getGroceries}|
ArrayAdapter<String> adapter = new ArrayAdapter<String>(getActivity(),
    android.R.layout.simple_list_item_1, grocieries);
final AutoCompleteTextView editTextNew =
    (AutoCompleteTextView) rootView.findViewById(R.id.autoCompleteShoppingListNewItem);
editTextNew.setAdapter(adapter);

editTextNew.setOnEditorActionListener(new TextView.OnEditorActionListener() { |\label{line:onEditorListener}|
  @Override
  public boolean onEditorAction(TextView v, int actionId, KeyEvent event) {
    boolean handled = false;
    if (actionId == EditorInfo.IME_ACTION_SEND) { |\label{line:actionIdVergleich}|
      handleItemAddAction(v.getText().toString()); |\label{line:handleItemAdd}|
      handled = true;
    }
    return handled;
} });
[|\ldots{}|]
\end{lstlisting}  
\end{minipage}      

Dieser Teil der App konnte ohne weitere Probleme gel�st werden, weswegen auf die hier �blichen Paragraphen \textsc{Probleme}
und \textsc{L�sungen} verzichtet wird.


\section{Blackboard}\label{section:Blackboard}
Eine einfache M�glichkeit alle WG-Mitglieder zu erreichen bietet ein Blackboard auf dem jeder Nachrichten hinterlassen
kann. Obwohl der Einsatz von Zugriffsbeschr�nkungen auf die Nachrichten leicht umsetzbar w�re,
wurde bewusst darauf verzichtet, um die Eigentschaften eines physikalischen Blackboards gerecht zu werden.

\subsection{Implementierung}\label{subSec:BbImpl}
Die Klassenhierarchie der Fragmente ist aus den vorhergehenden Beispielen schon bekannt und wurde auch in beim
\texttt{BlackboardFragment} beibehalten. Dabei wird in der �berschriebenen Methode \textsl{onCreateView(\ldots{})} durch
ein SQL-Statement die Nachrichten aus der \verb!SQLite!-Datenbank gelesen, welche in einem \verb!Cursor! vorgehalten
werden. Da es sich bei dem Zeilenlayout der \verb!ListView! um kein Standard-Layout handelt, muss zun�chst der
\verb!Cursor! schrittweise durchgearbeitet werden und die Werte in eine \verb!HashMap! �bertragen werden. Auf eine
\verb!HashMap! wurde zur�ckgegriffen, um beim L�schen des Eintrags einfach �ber die \textit{BlackboardId} an die
Nachricht zu gelangen und aus der Liste zu l�schen, das spart einen direkten Datenbankzugriff f�r das L�schen und
weitere Zugriffe beim Aktualisieren der Liste, sowie dem etwaigen Wiederherstellen der Nachricht.
�ber eine \verb!EditText!-Feld kann die neue, mehrzeilige Nachricht eingegeben werden und durch den
\textit{+}-Button dem schwarzen Brett hinzugef�gt werden. Das L�schen der einzelnen Nachrichten kann durch ein Klick auf
das L�schen-Symbol ausgel�st werden und ist durch die \verb!UndoBar!-Funktion revidierbar.
Zum Bearbeiten wurde ein \verb!onLongClickListener! an die \verb!View! der Zeile gebunden. Wird dieser ausgel�st, so
generiert er einen Dialog mit dem aktuellen Inhalt der Nachricht und zeigt diesen an. Gespeichert wird dann die
Nachricht mit der ID des Bearbeiters. Eine Erweiterung um die \verb!UndoBar!-Funktion sollte hier noch erg�nzt werden.

\paragraph{Probleme}\label{para:BbProbl}
Bei der Umsetzung des Blackboards kam es beim Anzeigen der Nachrichten zu einem komplexen Problem, was zun�chst nicht
nachvollziehbar war. In der \verb!ListView! war zwar die Anzahl der Eintr�ge richtig, aber der Anfang der Liste wurde am 
Ende der Liste, also bei Zeilen die ausserhalb des anf�nglich darstellbaren Bereichs lagen, wiederholt.

\paragraph{L�sungen}\label{para:BbSol}
Um die Datenquelle als Fehler auszuschlie�en wurde zun�chst ein \verb!DISTINCT! in das \verb!SQL!-Statement eingef�gt.
Das bewirkt, dass doppelte Eintr�ge ausgefiltert werden. Das war aber nicht die Ursache des Problems, da die erwarteten
Daten im \verb!Cursor! waren und auch an den \texttt{BlackboardAdapter} weitergegeben wurden. Somit lies sich das
Problem auf die Anzeige, beziehungsweise das die Vorbereitung der Daten zu Anzeige, eingrenzen. Demnach muss sich der
Fehler im erw�hnten \texttt{BlackboardAdapter} befinden. Nachdem weitere Gedanken �ber die Funktion des Adapters gemacht
wurden, kam die Erkenntnis, dass die Zeile \ref{line:WrongLineBbAdap} im alten Quelltext \ref{java:BbAdapOld} nicht
funktionieren kann, sobald es mehr Eintr�ge gibt, als auf anhieb anzeigbar sind.

\begin{lstlisting}[float, language=java, caption=Alte BlackboardAdapter.java, captionpos=below, label=java:BbAdapOld,
escapechar=|]
[|\ldots{}|]
@Override
public View getView(int position, View convertView, ViewGroup parent) { |\label{line:getView}|
  if (convertView &=&  null) { |\label{line:WrongLineBbAdap}|
    final BlackboardMessage bbMsg =
        (BlackboardMessage) blackboardMessages.values().toArray()[position];
  [|\ldots{}|] }
}
[|\ldots{}|]
\end{lstlisting}

Mit der \verb!if!-Abfrage, ob die �bergebene \verb!View! noch \verb!null! ist, wird verhindert, dass wenn die
\verb!ListView! gescrollt wird, die neue Zeile �berschrieben werden kann. In der \verb!ListView! befinden sich n�mlich
immer gleich viele \verb!Views! als Zeilen, vorrausgesetzt dass mehr Eintr�ge dargestellt werden sollen als auf den
sichtbaren Bereich passen. Tritt der Fall ein dass neue Zeilen angezeigt werden m�ssen, sprich es wird gescrollt, werden
die angezeigten \verb!View!s der \textsl{getView(int position, View convertView, ViewGroup parent)} (vgl. Zeile
\ref{line:getView} im \nameref{java:BbAdapOld}~\ref{java:BbAdapOld}) als \textit{converView} �bergeben. Somit kann diese
Variable nicht \verb!null! sein und die Bedingung der \verb!if!-Abfrage ist falsch. Demzufolge k�nnen die alten 
Daten nicht in den vorhandenen \verb!View!s durch die neuen ersetzt werden und die selbe Nachricht wird noch mal
angezeigt. Die Wiederholung des Anfangs wird dadurch erzeugt, dass die Zeilen-\verb!View!s wiederverwendet werden, die
aus dem angezeigten Bereich geschoben werden, also die zuvor erste \verb!View!.\\*
Durch das Entfernen der \verb!if!-Abfrage wurde die erw�nschte Funktion erreicht und das Scrollen der Liste war
m�glich.

\section{Weboberf�che}\label{section:WebUI}
Der Administrator einer WG muss in der Lage sein, die Eigenschaften der WG zu konfigurieren und die Funktionen zu verwalten. Wir haben uns dazu entschieden, eine extern jederzeit erreichbare Weboberfl�che daf�r bereit zu stellen. Wir h�tten uns genauso gut f�r die Implementierung in die Android App entscheiden k�nnen, haben uns jedoch bewusst dagegen entschieden. Wir m�chten mit den unterschiedlichen Programmiersprachen und den daraus resultierenden Herangehensweisen eine m�glichst gro�es Vielfalt der Informatik widerspiegeln und die Arbeiten auf die Mitglieder verteilen, die bisher noch keine Erfahrung mit der Programmierung f�r die Android Plattform sammeln konnten.\\
Im folgenden Kapitel wird die Weboberfl�che mit Hilfe von Codesnippets erl�utert.

\subsection{Umsetzung}\label{subSec:WebUmsetz}
Als Programmiersprache haben wir uns f�r die sehr beliebte und weit verbreitete Skriptsprache PHP entschieden. PHP bietet durch den Einfluss von Java, C++ und Perl einen leichten Einstieg f�r diejenigen, die bereits erste Erfahrungen mit einer der Programmiersprachen sammeln konnten. Ausserdem bietet PHP die einfache Umsetzung von dynamischen Webseiten und eine sehr gute Unterst�tzung von Datenbankverbindungen. Mit PHP ist automatisch sichergestellt, dass die Weboberfl�che von jedem g�ngigen Browser aus in deren Desktop- sowie Mobilversion genutzt werden kann.\\
\begin{description}
\item Die Weboberfl�che gliedert sich in 11 Seiten:
\begin{itemize}
	\item admin.php
	\item benutzer.php
	\item blackboard.php
	\item einkaufsliste.php
	\item login.php
	\item logout.php
	\item admin.php
	\item putzplan.php
	\item regist.php
	\item admin.php
	\item system.php
\end{itemize}
\end{description}
Jede dieser PHP Dateien beinhaltet gew�hnlichen HTML Code um dem Browser die Daten pr�sentieren und PHP Code f�r den dynamischen Teil der Datenabfrage von der Datenbank.
\chapter{Zusammenfassung und Ausblick}
In dieser Projektarbeit wurde die Umsetzung einer mobilen Applikation zur zentralen Verwaltung WG-typischer Aufgaben erl�utert. Dabei wurde mit Hilfe der Entwicklungsumgebung \textit{Android Studio} und der Programmiersprache \textit{Java} eine Applikation implementiert, die auf allen Smartphones mit Android 4.2 oder h�heren Versionen des Betriebssystems lauff�hig sind. Die Daten der Nutzer werden in einer externen, �ber eine Internetverbindung erreichbare Datenbank gespeichert. Bei jedem Start der App werden die Daten des Benutzers aus der Datenbank geladen und bei jedem Beenden der App wieder in diese Datenbank geschrieben.\\

Die im Vorfeld get�tigten umfangreichen Vorbereitungen, von Use-Cases-Digrammen, �ber Mockups bis hin zum Lastenheft, haben uns im nach hinein in der Implementierungsphase sehr gut weitergeholfen. Insbesondere durch das Lastenheft waren wir zu jederzeit in der Lage den �berblick �ber die Funktionen und deren Anforderungen zu behalten. Damit hatten wir zur finalen Version der Applikation eine gute Referenz zum vergleichen.\\

Die Datenbankverbindung hat im Vergleich zu anderen Anforderungen die h�chste Aufmerksamkeit bekommen. Bereits in der ersten Vorbesprechung war uns klar, dass es bei der Arbeit mit externen Quellen zu den verschiedensten Fehlern kommen kann. Die Entwicklung eines Prototypen vor Beginn der finalen Implementierung der App, hat uns vermutlich im sp�teren Verlauf erheblich viel Zeit und �rger erspart.\\

Leider gestaltete sich der Einstieg in die Programmierung f�r die Plattform \textit{Android} schwieriger als erwartet. Am Anfang gab es Kompatitiblit�tssprobleme bei der Konfiguration von \textit{Eclipse} mit dem \textit{Android Studio Plugin}, weshalb wir kurzerhand auf die Standalone Variante\textit{ Android Studio }umgestiegen sind. Doch auch bei dieser Software geh�ren bei Konfiguration und Implementierung verwirrende Mechaniken, sich wiedersprechende Anleitungen und undeutliche Fehlermeldungen fast zur Tagesordnung. Die Tastsache dass lediglich 1/3 des Teams erste grundlegende Erfahrungen in der Programmierung f�r \textit{Android} sammeln konnte, hat die Situation w�hrend der Implementierungsphase weiter erschwert. F�r weitere Projekte w�re zu Beginn ein Teammeeting zu empfehlen, in dem alle Mitglieder auf den gleichen Stand der Technik und des Wissens gebrachten werden. \\

Eine Ver�ffentlichung in den \textit{Google Play Store} und dem damit verbundenen erh�hten Arbeitsaufwand der Qualit�tssicherung, dem Support der Nutzer und der Erwartung der Nutzer auf immer weitere Steigerung des Umfangs, ist zu dem jetzigen Zeitpunkt nicht geplant.
% ...
%--------------------------------------------------------------------------
\backmatter                        		% Anhang
%-------------------------------------------------------------------------
\bibliographystyle{geralpha}			% Literaturverzeichnis
\bibliography{literatur}     			% BibTeX-File literatur.bib
%--------------------------------------------------------------------------
\printindex 							% Index (optional)
%--------------------------------------------------------------------------
\begin{appendix}						% Anh�nge sind i.d.R. optional
   \include{chapters/Glossar}			% Glossar   
   \include{chapters/Selbststaendigkeitserklaerung}	% Selbstst�ndigkeitserkl�rung
\end{appendix}

\end{document}
